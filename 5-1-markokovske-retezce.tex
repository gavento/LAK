\subsection{Markovovské řetězce}


\df Markovovský řetězec je orientovaný graf s váženými hranami takový, že
výstupní stupeň každého vrcholu je 1. Markovoský řetězec často reprezentujeme
maticí přechodu $P$, kde $P_{ij}$ udává pravděpodobnost, že ze stavu $i$
přejdeme do stavu $j$.

\df Distribuce $\pi$ je vektor, jehož součet je 1 a kde $p_i$ určuje
pravděpodobnost, že se nacházíme ve stavu $i$.

\pzn Máme-li distribuci $\pi$ a provedeme jeden krok na Markovovském řetězci s
maticí přechodu $P$, dostaneme novou distribuci $\pi\cdot P$.

\df Markovovský řetězec je reversibilní, existuje-li distribuce $\pi$
t. že $\pi_i\cdot P_{ij} = \pi_j\cdot P_{ji}$.

\lm Markovovský řetězec je reversibilní $\Leftrightarrow$ je odvozen z váženého neorientovaného grafu.

\dk 
\begin{itemize}

\item[\uv{$\Leftarrow$}]
Zvolíme si $\pi$ následovně a ukážeme, že splňuje reversibilní podmínku:
\begin{align*}
& \pi_v = {\deg v\over \sum_{u\in V(G)} \deg u} & P_{ij} = {w_G(i,j)\over \deg i} \\ 
\end{align*}
\begin{align*}
\pi_i P_{ij} &= \pi_i {w_G(i,j)\over \deg i} = {w_G(i,j)\over \sum_{u\in V(G)}\deg u} \\
\pi_j P_{ji} &= \pi_j {w_G(j,i)\over \deg j} = {w_G(j,i)\over \sum_{u\in V(G)}\deg u}
\end{align*}

\item[\uv{$\Rightarrow$}]
Zvolíme váhu $w(i,j) = P_{i,j}\pi_i = P_{j,i}\pi_j = w(j,i)$ a dostaneme vážený
neorientovaný graf.
\qed
\end{itemize}

\df $\pi$ je stabilní distribuce\footnote{Někdy též zvaná \uv{stacionární}.},
je-li $\pi\cdot P = \pi$. Jinak řečeno, stabilní distribuce se po provedení
kroku nezmění.

\vt Pro $G$ neorientovaný souvislý platí: $\forall \rho$ počáteční distribuci
$\set{ \rho P_G^k }_k$ konverguje $\Leftrightarrow$ $G$ není bipartitní.

\dk \begin{description}
	\item \uv{$\Rightarrow$} Pokud je $G$ bipartitní, stačí jako protipříklad vzít distribuci, která začíná jenom v jedné partitě. Pak každým pronásobením matice se celá distribuce přesune do druhé partity, protože nemá kam jinam. Zjevně tedy nekonverguje k jedinému rozložení.
	\item \uv{$\Leftarrow$} Prvně si vyjádříme distribuci jako lineární kombinaci vlastních vektorů matice $P_G$ (to lze, protože tvoří ortonormální bázi). Tedy $\rho = \sum_i a_ip_i$. Dále si vyjádříme distribuci po $k$ iteracích:
  \todo{distribucí násobíme zleva, dále (levým) vlastním vektorem 1 není vektor jedniček,
        ale stabilní distribuce $\pi$ \dots}
	\begin{align}
		P_G^k\rho = P_G^k\sum_ia_ip_i = \sum_iP_G^ka_ip_i
	\end{align}
	Protože $p_i$ je vlastní vektor $P_G$, tak $P_Gp_i = \lambda_ip_i$:
	\begin{align}
		\sum_i\lambda_i^ka_ip_i
	\end{align}
  \todo{$P$ není matice sousednosti\dots}
	Nyní si všimneme, že protože graf není bipartitní, tak $\lambda_1 \neq -\lambda_n$ a největší vlastní číslo distribuce je $1$, protože matice $P_G$ má řádkové i sloupcové součty konstantní $1$ a zároveň je $1$ má vlastní vektor samých jedniček. Tedy pro $i > 1$ platí $|\lambda_i| < 1$. Dejme nyní výraz do limity a všimneme si, že suma jde k nule díky tomu, že jediný člen závislý na $k$ je $\lambda_i$:
	\begin{align}
		\lim_{k\to \infty}\left(\lambda_1^ka_1p_1 + \sum_{i>1}\lambda_i^ka_ip_i\right) = a_1p_1 = \pi
	\end{align}
	Tedy máme stabilní distribuci, protože $a_1p_1$ jsou po celou dobu konstantní.
\end{description}

\vt Nechť $\rho$ je distribuce na vrcholech grafu a $\mu=\max\{\lambda_i,-\lambda_n\}$. Pak po $t$ krocích platí, že $\|P_G^t\rho - \pi\|_1 \leq \mu^t\sqrt{n}$, tedy distribuce konverguje relativně rychle.

\dk Z předchozího důkazu víme, že $\rho = p_ia_i + \sum_{i>1} \lambda_i^ta_ip_i$ a \todo{vec}.

Pusťme se do odhadu naší odchylky, prozatím však v $L_2$ normě.

\begin{align}
	\|P_G^t\rho - \pi\|_2^2 = \left\|\sum_{i > 1} \lambda_i^ta_ip_i\right\|_2^2 = \sum_{i>1}\lambda_i^{2t}\|a_ip_i\|_2^2
\end{align}
Nyní si zjednodušíme práci a do sumy zahrneme i první člen. Navíc odhadneme $\lambda_i$ největšším vlastním číslem $\mu$ (mocnina u $\lambda_i$ je sudá!).
\begin{align}
	\leq \mu^{2t}\sum_i\|a_ip_i\|_2^2 = \mu^{2t}\|\rho\|_2^2 \leq \mu^{2t}
\end{align}
Nyní stačí výraz odmocnit a vzpomenout si na analýzu, čímž víme, že $\|x\|_1 \leq \|x\|_2 \cdot \sqrt n$ a máme nerovnost:
\begin{align}
	\|P_G^t\rho - \pi\|_2 &\leq \mu^{t} \\
	\|P_G^t\rho - \pi\|_1 &\leq \mu^{t} \sqrt n
\end{align}
Což jsme chtěli dokázat. \qed

