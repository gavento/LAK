\subsection{Silně regulární grafy}
% Stanislava Tlustá

% úvod
Dalším typem regulárních grafů jsou silně regulární grafy.

\df $r$-regulární graf $G$ se nazývá silně regulární, pokud existují $e, f \in \N$ taková, že:
\begin{itemize}
	\item každá hrana $uv \in E(G)$ se vyskytuje právě v $e$ trojúhelnících (tj. $|N(u) \cap N(v)| = e$) a zároveň
	\item každá nehrana $uv \not\in E(G)$ se vyskytuje právě v $f$ třešničkách (tj. $|N(u) \cap N(v)| = f$).
\end{itemize}

\pzn Abychom mohli zanedbat triviální případy, dodáváme $f>0$ a $G\neq K_n$.
Příkladem silně regulárního grafu je úplný bipartitní graf se stejně velkými
partitami ($e=0$, $f$ velikost partity). Nejmenším nebipartitním silně regulárním grafem je
pětiúhelník ($e=0$, $f=1$).

\vt Nechť $G$ je silně regulární graf s parametry $(r,e,f)$ na $n$ vrcholech. Potom:
\begin{enumerate}
	\item[(a)] $f-e = 1$, $n = 2r +1$,  $r = 2f$ nebo
%	\item[] {\it nebo}
	\item[(b)] $\exists s \in \Z$ takové, že platí $(e-f)^2-4(f-r) = s^2$ \\
	a výraz \ ${r\over 2fs}((r-1+f-e)(s+f-e)-2f)$ je přirozené číslo.
\end{enumerate}

\dk
% jak na to
Technika tohoto důkazu je stejná jako u předchozí věty o Mooreových grafech.

% sestavení polynomu
Matici souslednosti grafu $G$ označíme $A$. Její druhá mocnina má na diagonále $r$. Mimo diagonálu má buď hodnotu $e$ pro případ kdy v $A$ byla jednička ($e$ trojúhelníků), nebo hodnotu $f$, pokud v $A$ byla nula ($f$ třešniček). Vidíme tedy vztah:
\begin{align}
A^2 = rI + eA + f(J-I-A),
\end{align}
kde $E$ je jednotková matice a $J$ je matice samých jedniček. Úpravou dostaneme polynomiální vztah
\begin{align}
p(A) = A^2 + (f-e)A + (f-r)E = fJ.
\end{align}

% spektra
Dále platí, že pro $\lambda \in \Sp(A)$ je $p(\lambda) \in Sp(fJ)$. Spektrum $fJ$ známe: obsahuje $fn$ s násobností $1$ a $0$ s násobností $n-1$.

% % \lambda = r
Protože $G$ je $r$-regulární, tak je $r$ také jeho vlastním číslem. Dosadíme tedy $r$ do $p$:
\begin{align}
p(r) &= r^2 + (f-e)r + (f-r) \\
p(r) &= r^2 +fr -er +f -r +1 -1 \\
p(r) &= (r^2 -er +1) + f(r+1) -(r+1) \\
p(r) &= (r^2 -er +1) + (r+1)(f-1).
\end{align}
Protože $f>0$, tak platí $(r+1)(f-1) \geq 0$.
Navíc zřejmě $e<r$, tudíž $r^2 -er +1 >0$. Jediné vlastní číslo, které toto splňuje je $fn$, proto
\begin{align}
\label{4-3:vztah-pro-vrcholy}
p(r) = fn
\end{align}

%%%%%%%%
% % p(x)=0
Nyní zbývá vyřešit případ, kdy $p(\lambda)=0$.
Řešení této kvadratické rovnice je 
\begin{align}
\lambda_{1,2} = {e-f \pm s \over 2}, \qquad s = \sqrt{(f-e)^2 -4(f-r)}.
\end{align}
Násobnosti těchto kořenů označíme $m_1, m_2$. Jejich součet je zřejmě roven $n-1$.
% % stopa
Využitím faktu, že stopa matice je suma vlastních čísel 
včetně násobností, získáme rovnici
\begin{align}
	0 = \Tr(A) = r + m_1\lambda_1 + m_2\lambda_2,
\end{align}
kterou upravíme
\begin{align}
0 &= r + {m_1 \over 2}(e -f +s) + {m_2 \over 2}(e -f -s) \\
0 &= 2r + (e-f)(m_1 +m_2) +s(m_1 -m_2). \label{4-3:rovnice-pro-s}
\end{align}

% % rozbor odmocniny
Řešení rozdělíme na dva případy.
\begin{enumerate}
% % % \not\in \Q
\item $s \not\in \Q$: potom $m_1 = m_2$. Potom se rovnice zjednoduší
\begin{align}
0 &= 2r +2m_{1}(e-f) \\
m_1 &= {r \over f-e}.
\end{align}
Z toho vidíme, že
\begin{align}
\label{4-3:rovnice-ef}
	(f-e)|r, \quad f-e > 0, \quad n = 1 + 2m_1 = 1 + {2r\over f-e}.
\end{align}
Pokud $f-e = 1$, tak jsme hotovi. \\
Pokud $f-e = 2$, pak $n = 1+r$ a $G = K_{r+1}$, ale úplné grafy jsme si zakázali. \\
Pokud $f-e > 2$, pak $n < 1+r$, což je nesmysl.

Po dosazení $f-e=1$ do poslední rovnice \ref{4-3:rovnice-ef} vidíme, že $n = 2r+1$. \\
Po dosazení téhož do polynomu $p$ a použitím vztahu \ref{4-3:vztah-pro-vrcholy} dostáváme
\begin{align}
r^2 + r + (f-r) = f(2r+1).
\end{align}
Z toho již snadnou úpravou získáme hledané rovnosti
\begin{align}
r = 2f, \qquad n = 4f+1.
\end{align}

%%%%%%
% % %  \in \Q
\item $s \in \Q$, což implikuje $s \in \N$.
Vezmeme vztah pro násobnosti vlastních čísel a vztah \ref{4-3:vztah-pro-vrcholy} pro $n$
\begin{align}
m_2 = n - 1 - m_1 = {r^2 -er +1 +(r+1)(f-1) \over f} -1 -m_1.
\end{align}
Dosadíme do rovnice \ref{4-3:rovnice-pro-s}. Dostaneme:
\begin{align}
0 = 2r +m_1(e-f+s) + ({r^2 -er +1 +(r+1)(f-1) \over f} -1 -m_1)(e-f-s) 
\end{align}
Což dále upravíme
\begin{align}
& m_1(-(e-f+s)+(e-f-s)) = 2r + {1 \over f}((r^2 -er +rf -r +f)-f)(e-f-s), \\
& m_1(-2s) = 2r +{1 \over f}(r^2 -er +fr -r)(e-f-s), \\
& m_1 = {1 \over 2sf}(-2rf +r(r-e+f-1)(-e+f+s)), \\
& m_1 = {r \over 2sf}(r-1+f-e)(s+f-e) -2f).
\end{align}
Protože $m_1$ je násobnost vlastního čísla $\lambda_1$, tak se jedná o přirozené číslo.
\end{enumerate}
\qed

\vt(Friendship theorem) Nechť každí dva lidé mají právě jednoho společného známého. Pak existuje jeden (starosta), který zná všechny.

Neboli: nechť pro každé dva různé vrcholy $u, v \in V(G)$ platí $|N(u) \cap N(v)| = 1$. Potom existuje vrchol $c \in V$ takový, že $N(c) \cup \{c\} = V$.

\pzn Friendship theorem tvrdí, že takový graf musí vypadat jako
mlýn (hromádka trojúhelníků, které se stýkají v jednom centrálním vrcholu), viz obrázek \ref{4-3:obr-mlyn}.
\begin{align}
\label{4-3:obr-mlyn}
\begin{tikzpicture}[thick,scale=1.1]
% obrázek: šestilopatkový mlýn
\draw \foreach \x in {15,75,...,315}
    {
        (0,0) -- (\x:2) \vrchol -- (\x+30:2) \vrchol -- cycle
    };
\draw (0,0) \vrchol;
% popisek
\draw (5,0) node {šestilopatkový mlýn};
\end{tikzpicture}
\end{align}

\dk
% převzato z:
% http://math.mit.edu/~fox/MAT307-lecture20.pdf
% jak na to - sporem
Pro spor předpokládejme, že takový vrchol $c$ neexistuje. Nejprve si všimněme, že podmínka na množství sousedů implikuje, že $G$ neobsahuje čtyřúhelník. 

% G regulární
% % u,v nesousedí
Nejdříve ukážeme, že $G$ je regulárním grafem. Vezměme libovolné dva vrcholy $u$, $v$, které spolu nesousedí a označme $w_1, ..., w_k$ sousedy vrcholu $u$. Každý vrchol $w_i$ musí mít jednoho společného souseda $z_i$ s vrcholem $v$. Vrcholy $z_i$ musí být různé, jinak by vznikl čtyřúhelník ($u, w_i,z_i =z_j, w_j$). Vrchol $v$ má tedy také alespoň $k$ sousedů. Symetrickou úvahou pak dostáváme rovnost $deg(u) = deg(v)$. 

% % ostatní vrcholy
Vrcholy $u$ a $v$ mají právě jednoho společného souseda $c$.
Bez újmy na obecnosti můžeme předpokládat, že je to $c=w_1$. Jakýkoliv jiný vrchol $w \in V(G)$ již sousedí nanejvýše s jedním z vrcholů $u$ a $v$ (jinak by vznikl čtyřúhelník). Zopakováním předchozí úvahy pro vrchol se kterým $w$ nesousedí vidíme $deg(w) = deg(u) = deg(v)$. Nakonec $w_1$ nemůže podle předpokladu být spojen se všemi vrcholy, proto i pro něj platí $deg(w_1) = deg(u) = deg(v)$.

Všechny vrcholy tedy mají stejný stupeň a graf $G$ je $k$-regulární. Dokonce je silně regulární ($e=f=1$).

% délky cest
Nyní se podíváme na sledy délky $2$. Od každého vrcholu $x \in V$ jich vede $k^2$, protože $G$ je $k$-regulární. Navíc z vrcholu $x$ vede do každého jiného vrcholu $y \in V$ právě jedna cesta délky $2$ ($e=f=1$). Sledů z $x$ do $x$ je přesně $k$. Dostáváme tedy vztah, ze kterého vyjádříme počet vrcholů:
\begin{align}
k^2 &= (n-1) + k \\
n &= k^2 -k +1. \label{4-3:friendship-vrcholy}
\end{align}

% vlastní čísla
V dalším kroku zopakuje již známý postup pro hledání polynomiálního vztahu vlastních čísel. Matici souslednosti grafu $G$ označíme $A$.
Z rozboru sledů délky $2$ provedeném v předchozím kroku vidíme, že matice $A^2$ má na diagonále $k$ a všude mimo diagonálu jedničky. Dostáváme tedy vztah
\begin{align}
A^2 = J + (k-1)I.
\end{align}
Vlastními čísly matice $A^2$ jsou tedy $n+(k-1) = k^2$ s násobností $1$ a $k-1$ s násobností $n-1$. Vlastní čísla matice $A^2$ jsou druhými mocninami vlastních čísel matice $A$.  
Tudíž matice $A$ má vlastní čísla $k$ s násobností $1$ a $\pm \sqrt{k-1}$ s násobnostmi $m_1$, $m_2$.
Použitím vztahu pro stopu matice dostáváme:
\begin{align}
0 = \Tr(A) = k + (m_1 -m_2)\sqrt{k-1}.
\end{align}
To upravíme do tvaru
\begin{align}
k^2 = (m_2 -m_1)^2(k-1),
\end{align}
z něhož plyne, že $k-1|k^2$. To je však možné pouze pro $k = 1,2$. Jinak totiž $k-1$ dělí $k^2-1$, nemůže tedy dělit zároveň $k$. 

% závěr
Hodnotě $k=1$ odpovídá po dosazení do rovnice \ref{4-3:friendship-vrcholy} graf $K_1$.
Hodnotě $k=2$ odpovídá graf $K_3$. 
Oba splňují jak předpoklady, tak závěr věty. 
Pro jakýkoliv jiný graf nastává spor, tudíž musel vrchol $c$ sousedit se všemi ostatními.
\qed
