\df Vektory $v_1,\dots,v_n$ jsou {\it lineárně nezávislé}, jestliže neexistuje netriviální řešení 
rovnice $\sum_{i=1}^n \alpha_iv_i=0$.

\medskip
\subsection{Sudo-licho města a skorodisjunktní systémy podmnožin}


\df Buď $X$ $n$-prvková množina a $A_1,\dots,A_m$ systém jejích neprázdných podmnožin takový, že $A_i \ne A_j$ pro $i\neq j$. Úloha {\it A-B město} se ptá, jak velké může být $m$, je-li $|A_i|\sim B$ a $|A_i\cap A_j|\sim A$ pro všechna $i,j=1,\dots,m$, $i\neq j$.

\medskip
V~případě sudo-licho města tedy máme omezení na sudé průniky a liché velikosti.

\vt Pro sudo-licho město platí $m \leq n$.

\dk Podmnožinu množiny $X$ ztotožněme s~jejím charakteristickým vektorem délky $n$ a označme $A$ matici o~rozměrech $m\times n$, která má v~$i$-tém řádku vektor $A_i^T$. Platí
\begin{align}
A_i^TA_j\!\mod2=\begin{cases}1,&\text{je-li }i=j,\\0,&\text{je-li }i\neq j,\end{cases}
\end{align}
tedy nad $\GF(2)$ máme
\begin{align}
	AA^T = \left(\begin{matrix}A_1^T\\ A_2^T \\ \vdots \\ A_m^T \end{matrix}\right) 
	\cdot \left(\begin{matrix}A_1, A_2, \dots, A_m\end{matrix}\right) = I,
\end{align}
speciálně $\rank(AA^T) = m$. Jelikož každý sloupec matice $AA^T$ je lineární kombinací sloupců matice $A$, je $\rank(AA^T)\leq\rank(A)\leq n$. Tedy $m\leq n$.\qed

\vt Nechť pro $A_1,\dots,A_m\subseteq X$ platí $|A_i \cap A_j| = 1 $ a $A_i \neq A_j$, $i\neq j$. Potom $m \leq n$.

\dk Stejně jako v~předchozím důkazu označme $A$ matici charakteristických vektorů a podívejme se na součin $AA^T$, tentokrát však nad $\Q$:
\begin{align}
	AA^T = \left(\begin{matrix}
	a_1&&\smash{\bigone\quad}\\
	&\ddots&\\
	\smash{\quad\raisebox{5pt}{\bigone}}&&a_m
	\end{matrix}\right),\quad\text{kde }a_i=|A_i|. 
\end{align}
Dokážeme-li, že tato matice je regulární, získáme kýženou nerovnost $m\leq n$.

Všimněme si, že pro všechna $i$ je $a_i\geq1$, přičemž rovnost nastává nejvýše pro jedno z~nich. Můžeme tedy předpokládat, že pro $i\geq2$ platí $a_i\geq a_1\geq1$ (tedy $a_i\geq2$). Odečtením prvního řádku od všech ostatních získáme matici
\renewcommand{\arraystretch}{1.2}
\begin{align}
	B=\left(\begin{matrix}
	a_1 & 1 &1&\hdots &1 \\
	1-a_1& a_2-a_1 &&&\smash{\raisebox{-10pt}{\bigzero}\quad} \\
	1-a_1&&a_3-a_1&&\\
	\vdots&&&\ddots&\\
	1-a_1&\smash{\quad\raisebox{15pt}{\bigzero}}&&&a_m-a_1
	\end{matrix}\right),
\end{align}
jejíž determinant spočteme z~definice jako
\begin{align}
  \det(B)=a_1\cdot\prod_{i=2}^m(a_i-a_1)-(1-a_1)\cdot\sum_{i=2}^m\prod_{\substack{j=2\\j\neq i}}^m(a_j-a_1).
\end{align}
Protože $1-a_1\leq0$ a pro $i\geq2$ je $a_i-a_1>0$, dostáváme $\det(B)>0$. Tedy $B$ je regulární. Přičtení prvního řádku k~ostatním na regularitě zřejmě nic nezmění, a proto je i $AA^T$ regulární, což jsme chtěli dokázat.\qed

\medskip
Soubor podmnožin z~předchozí věty se nazývá {\it skorodisjunktní systém podmnožin}. Sudo-licho města a skorodisjunktní systémy podmnožin nyní využijeme ke konstrukci dolního odhadu Ramseyova čísla.

\vt (Ramsey) Pro každé $n\in\N$ existuje $N\in\N$ takové, že každý graf $G$ na aspoň $N$ vrcholech splňuje $\omega(G)\geq n$ nebo $\alpha(G)\geq n$.

\medskip
Víme, že $R_2(n) = N_{\min} \leq\binom{2n-2}{n-1}$. Ukážeme nerovnost $R_2(n)\geq\binom{n-1}3$.

\vt (Dolní odhad Ramseyova čísla) Existuje graf na $\binom{n-1}3$ vrcholech, který má kliku i nezávislou množinu velikosti nejvýše $n-1$.

\dk Buď $X$ množina, $|X|=n-1$. Sestrojíme graf
\begin{align}
G=\left(V=\binom X3, E=\{uv; |u\cap v|=1, u,v\in V\}\right).
\end{align}
Klika v~$G$ je skorodisjunktní systém podmnožin $X$, tedy $\omega(G) \le |X| = n-1$. Vrcholy jsou nezávislé, pokud $|a\cap b| \in \{0,2\}$, tedy nezávislá množina v~$G$ je sudo-licho město a $\alpha(G) \le |X| = n-1$.\qed
