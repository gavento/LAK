\subsection{Sudo-sudo města}


V~následující větě zachováme značení z~kapitoly o~lineární nezávislosti.

\vt Pro sudo-sudo město platí $m_{\max} = 2^{\lfloor n / 2 \rfloor}$.

\dk Nejprve sestrojíme sudo-sudo město o~velikosti $m=\lfloor\frac n2\rfloor$. Rozdělíme prvky množiny $X$ do dvojic (pokud jeden přebývá, odložíme ho stranou a dále se jím nebudeme zabývat) a za $A_1,\dots,A_m$ vezmeme všechny neprázdné podmnožiny množiny $X$, které obsahují z~každé dvojice buď oba prvky, nebo žádný. Takových podmnožin je $2^{\lfloor n / 2 \rfloor}$ a evidentně se jedná o~sudo-sudo město.

Nyní ukážeme nerovnost $m\leq\lfloor\frac n2\rfloor$. Nechť $M=\{A_1, A_2, \dots, A_m\}$ je co do inkluze maximální sudo-sudo město. Ztotožníme-li množiny $A_i$ s~jejich charakteristickými vektory, pak pro všechna $i,j\in\{1,\dots,n\}$ je $\sk{A_i,A_j} \text{ mod }2=0$. Tedy $M$ je vektorový prostor nad $\GF(2)$, neboť platí:
\begin{align*}
	&\emptyset \in M, \\
	&\forall u~\in M, \forall c\in \GF(2)\colon c\cdot u~\in M, \\
	&\forall x, \forall u,v\in M\colon \sk{x, u+v} = \sk{x,u} + \sk{x,v} = 0 + 0 = 0, \\
	&\forall u,v \in M\colon \sk{u+v,u+v} = \sk{u,u} + 2\sk{u,v} + \sk{v,v} = 0 + 0 + 0 = 0.
\end{align*}
Pokud $x\in M$, pak také $x\in M^\bot$, a tedy $M\subseteq M^\bot$. To znamená, že
\begin{align}
\dim M\leq\dim M^\bot=n-\dim M, \qquad\text{ekvivalentně } \dim M\leq\left\lfloor\frac n2\right\rfloor.
\end{align}
Jelikož $M \subseteq\GF(2)^n$ a $\dim M \leq\lfloor\frac n2\rfloor$, je $|M|=m \leq 2^{\lfloor n/2\rfloor}$. \qed


