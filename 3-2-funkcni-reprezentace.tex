\subsection{Funkční reprezentace grafu}

\df Nechť $G$ je graf, $X$ množina, $\Ft$ těleso a $\F\colon X\rightarrow\Ft$ systém funkcí. Pak pro vrchol $v$ mějme $c_v\in X$ a $f_v \in \F$, že
$f_v: X \to \Ft$ a platí:
\begin{enumerate*}
	\item $f_v(c_v) \neq 0$
	\item $uv \notin E_G \Rightarrow f_u(c_v) = 0$
\end{enumerate*}
Jinými slovy, pro funkci $f_a$ vrcholu $a$ platí, že vrchol dostane vždy nenulový prvek a jeho nesousedi vždy nulový. O sousedech nehovoříme nic.

\df Dimenzi $\F$ definujeme jako $\dim\calL(\{f_v\})$, tedy chápeme funkce
jako vektorový prostor.

\lm(O vztahu $\alpha$ a $\dim\F$) G má reprezentaci $\F$, pak $\alpha(G)
\leq \dim \F$.

\dk Nechť $A$ je nezávislá v $G$. Pak $\{f_a\}_{a\in A}$ je lineárně
nezávislá, stejně jako $\{c_a\}_{a\in A}$. Vyhodnotím reprezentující
funkci v bodech $A$.
\begin{align}
M = \left(
	\begin{matrix}
		f_1(c_1) & f_1(c_2) & \dots \\
		f_2(c_1) & f_2(c_2) & \dots \\
		\vdots &&\\
	\end{matrix}\right)
\end{align}

Matice $M$ bude mít na diagonále nenuly a všude jinde nuly. Tím pádem
jsou její řádky lineárně nezávislé a její dimenze je $|A|$. Navíc zjevně
$\dim M \le \dim\F$.
\qed


\lm(O dimenzi součinu reprezentací) Pokud $G_1$ má reprezentaci $\F_1$,
$G_2$ reprezentaci $\F_2$ nad stejným tělesem, pak $G = G_1 \boxtimes
G_2$ má reprezentaci $\F$,pro kterou platí $\dim\F \leq \dim\F_1 \cdot \dim\F_2$.

\dk Definujeme: 
\begin{align*}
& X = X_1 \times X_2 \\
& c_{(v_1,v_2)} = (c_{v_1}, c_{v_2}) \\
& f_{(v_1, v_2)}((x_1,x_2)) = f_{v_1}(x_1) \cdot f_{v_2}(x_2)
\end{align*}

Ověříme, že výše uvedené je funkční reprezentace a vezmeme si $B_1$ bázi
$\F_1$ a $B_2$ bázi $\F_2$. Pak $\{b_1 \otimes b_2\}_{b_1\in B_1, b_2\in
B_2}$ generuje celý prostor $\F$ a tudíž:
$$
	\dim\F \le |B_1|\cdot |B_2| = \dim\F_1 \cdot \dim\F_2
$$
\qed

\lm(O vztahu $\shn$ a $\dim\F$) G má reprezentaci $\F$, pak $\shn(G)
\leq \dim \F$.

\dk 
$$\shn(G) = \sup_i \alpha(G^i)^{1/i} \leq \sup_i(\dim f.r.(G^i))^{1/i} \leq \sup_i\dim f.r. (G) = \dim f.r.(G)$$

První nerovnost plyne z lemmatu o vztahu $\alpha$ a $\dim\F$, druhá z
lemmatu o dimenzi součinu reprezentací.
\qed


\vt Existuje $G, H$, že $\shn(G+H) > \shn(G) + \shn(H)$

\dk Zvolím $G$ takový, že $V_G={S\choose 3}$, $S = \{1, \dots, s\}$ a $E_G = \{ (A, B): |A\cap B| = 1\}$. 

Reprezentaci vytvoříme nad tělesem $\Ft = \Z_2$, $X = \Z_2^s$:
\begin{align*}
	c_A &= \text{charakteristický vektor } A \\ 
	f_A(x) &= \sum_{a\in A} x_a
\end{align*}

Ověříme, že se jedná o funkční reprezentaci a všimneme si, že každá
funkce $f_A$ je kombinace tří funkcí $b_i(x) = x_i$, přičemž počet funkcí
$b_i$ je $s$.
$$
\dim f.r.(G) \le s \qquad\Rightarrow\qquad \shn(G) \le s
$$

Dále pro $H = \overline G$ zvolíme reprezentaci pro $\Ft = \R$, $X =
\R^s$:
\begin{align*}
	c_A &= \text{charakteristický vektor } A \\ 
	f_A(x) &= (\sum_{a\in A} x_a) - 1
\end{align*}

Opět ověříme, že se jedná o funkční reprezentaci.
$$
\dim f.r.(\overline G) \le s+1 \qquad\Rightarrow\qquad \shn(\overline G) \le s+1
$$

$$\shn(G + \overline G) \ge \sqrt{2{s\choose 3}} > 2s+1 \ge \shn(G) + \shn(\overline G)$$

První nerovnost platí z lemmatu o dvojité kapacitě a ostrou nerovnost
musíme splnit, aby věta platila. Zvolíme si tedy $s \ge 16$.
\qed

\df Obecná poloha vektorů množiny $\check N$ v  $\R^d$ je taková, že libovolná podmnožina velikosti $\leq d$ je lineárně nezávislá.

\df Lokálně obecná poloha vektorů reprezentace v $\R^d$ na grafu $G$ jsou takové vrcholy, že $\rho(\overline{N(v)})$ jsou lineárně nezávislé.

\vt Pro $G$ s $|G| = n$ jsou následující tvrzení ekvivalentní:
\begin{enumerate}
	\item $G$ má ortogonální reprezentaci v $\R^d$ v obecné poloze.
	\item $G$ má ortogonální reprezentaci v $\R^d$ v lokálně obecné poloze.
	\item $G$ je $(n-d)$-souvislý.
\end{enumerate}


