\documentclass[a4paper,12pt,titlepage]{article}
\usepackage[utf8]{inputenc}
\usepackage{a4wide}
\usepackage[czech]{babel}
\usepackage{amsfonts, amsmath, amsthm, amssymb}
\usepackage[small,compact]{titlesec}
\usepackage{anyfontsize}
\usepackage{rotating}
\usepackage{wrapfig}
\usepackage{subcaption}
\usepackage{mdwlist}
\usepackage{xcolor}
\usepackage{graphicx}
\usepackage{tikz} % knihovna pro kreslení obrázků

\newcommand*\circled[1]{\tikz[baseline=(char.base)]{
		  \node[shape=circle,draw,inner sep=1pt] (char) {#1};}}
\newcommand{\shn}{\Theta}
\newcommand{\lm}{\smallskip\noindent\bf Lemma\rm{} }
\newcommand{\dk}{\smallskip\noindent\bf Důkaz\rm{} }
\newcommand{\df}{\smallskip\noindent\bf Definice\rm{} }
\newcommand{\vt}{\smallskip\noindent\bf Věta\rm{} }
\newcommand{\pr}{\smallskip\noindent\bf Příklad\rm{} }
\newcommand{\poz}{\smallskip\noindent\bf Pozorování\rm{} }
\newcommand{\pzn}{\smallskip\noindent\bf Poznámka\rm{} }
\newcommand{\dsl}{\smallskip\noindent\bf Důsledek\rm{} }
\newcommand{\tv}{\smallskip\noindent\bf Tvrzení\rm{} }
\newcommand{\F}{\mathcal{F}}
\newcommand{\B}{\mathcal{B}}
\newcommand{\A}{\mathcal{A}}
\newcommand{\calL}{\mathcal{L}}
\newcommand{\Z}{\mathbb{Z}}
\newcommand{\Compl}{\mathbb{C}}
\newcommand{\R}{\mathbb{R}}
\newcommand{\Q}{\mathbb{Q}}
\newcommand{\N}{\mathbb{N}}
\newcommand{\GF}{\mathrm{GF}}
\newcommand{\Ft}{\mathbb{F}}
\newcommand{\xttt}{{\chi_T^\bot}^T}
\newcommand{\todo}[1]{{\color{red}{\bf TODO: \rm#1}}}
%\newcommand{\qed}{\hfill QED}
\DeclareMathOperator{\rank}{rank}
\DeclareMathOperator{\Sp}{Sp}
\DeclareMathOperator{\zz}{\circled{z}}
\DeclareMathOperator{\Tr}{Tr}
\DeclareMathOperator{\Ker}{Ker}
\newcommand\bigzero{\makebox(0,0){\text{\huge0}}}
\newcommand\bigone{\makebox(0,0){\text{\huge1}}}
\newcommand{\bigddots}[1]{\makebox(0,0){\rotatebox{-35}{\text{\xleaders\hbox{$\cdot$\hskip4pt}\hskip#1\kern0pt}}}}
\newcommand{\sk}[1]{{\langle #1\rangle}}
\newcommand{\diagdots}[3][-25]{%
  \rotatebox{#1}{\makebox[0pt]{\makebox[#2]{\xleaders\hbox{$\cdot$\hskip#3}\hfill\kern0pt}}}%
}

\newcommand{\vrchol}{node[circle, draw, fill=black!70,inner sep=0pt, minimum width=6pt] {}} % obrázky: styl pro vrcholy

\def\br#1{\left(#1\right)}
\def\set#1{\left\{#1\right\}}
\def\rmd{\,\textrm{d}}

\usepackage[pdftex,unicode]{hyperref}   % Must follow all other packages
\hypersetup{breaklinks=true}
\hypersetup{pdftitle={Lineární algebra v kombinatorice}}
\hypersetup{pdfauthor={Tomáš Gavenčiak (ed.)}}
\hypersetup{urlcolor=blue}

\title{Lineární algebra v kombinatorice\thanks{Tyto poznámky k predmětu \uv{Lineární algebra v kombinatorice} na MFF UK jsou rozpracované a mohou být nekompletní.
Budete-li se chtít do jejich tvorby zapojit, ozvěte se na {\tt gavento@ucw.cz}.
}}

\author{
Jan Bok\\
Pavel Dvořák\\
Jan Horáček\\
Radek Hušek\\
Karel Král\\
Ladislav Láska\\
Tomáš Masařík\\
Honza Musílek\\
Stanislava Tlustá\\
Martina Vaváčková\\
\\
Tomáš Gavenčiak (ed.)\\}

\begin{document}

\maketitle
\newpage
\tableofcontents
\newpage



\section{Lineární nezávislost}

\subsection{Sudo-licho města}


\df Nechť $|X|=n$ a $A_1, \dots, A_m \subseteq X$ $A_i \ne A_j$  jsou neprázdné podmnožiny.  
Úloha A-B město se ptá, jak velké může být $m$, pokud $|A_i| \sim B$ a $|A_i\cap 
A_j|\sim A$ (tedy pro sudo-licho město máme omezení na liché velikosti a sudé 
průniky).

\vt Pro úlohu sudo-licho město platí $m \leq n$.

\dk Počítejme nad $GF(2)$. Matice $A$ nechť je charakteristická matice dimenze 
$n \times m$. Podívejme se na součin $AA^T$, tedy na matici skalárních součinů:
\begin{align}
	AA^T = \left(\begin{matrix}A_1\\ A_2 \\ \vdots \\ A_m \end{matrix}\right) 
	\cdot \left(\begin{matrix}A_1, A_2, \dots, A_m\end{matrix}\right) =
	\left(\begin{matrix}
	1 & &\bigzero & \\
	& \ddots && \\
	&\bigzero& \ddots & \\
	&&  &1
	\end{matrix}\right)
\end{align}
Tedy víme, že $\rank(AA^T) = m$ a $\rank(A) \leq n$. Z vlastností ranku již 
snadno získáme nerovnost $m=\rank(AA^T) \leq \rank(A) \leq n$. \todo{důkaz 
rankové nerovnosti obrázkem pomocí zobrazení} \qed

\vt Nechť $|X|=n$ a $A_1, \dots, A_m \subseteq X$ že platí $|A_i \cap A_j| = 1 $ 
a $A_i \neq A_j$. Potom $m \leq n$.
\dk Podobně jako v předchozím příkladě vezměme matici charakteristických vektorů 
$A$ a podívejme se na součit $AA^T$, tentokrát již nad $Q$:
\begin{align}
	AA^T = \left(\begin{matrix}
	|A_1| & &\bigone & \\
	& \ddots && \\
	&\bigone& \ddots & \\
	&&  &|A_m|
	\end{matrix}\right)
\end{align}
Dále označme $a_i := |A_i|$. Můžeme předpokládat, že $a_1 \leq a_2 \leq \dots 
\leq a_m$. Zřejmě také $a_2 > 1$ (jinak $A_1 = A_2$). Nyní bychom chtěli 
dokázat, že je matice regulární -- proto se podíváme na determinant této matice:
\begin{align}
	|AA^T| = \left|\left(\begin{matrix}
	a_1 & &\bigone & \\
	& \ddots && \\
	&\bigone& \ddots & \\
	&&  &a_m
	\end{matrix}\right)\right|
	%= \left|\begin{matrix}\text{{\fontsize{120}{120}\selectfont 1}}\end{matrix}
	= \left|\begin{matrix}\text{\bf\Huge 1}\end{matrix}
	+\left(\begin{matrix}
	a_1-1 & &\bigzero & \\
	& \ddots && \\
	&\bigzero& \ddots & \\
	&&  &a_m-1
	\end{matrix}\right)\right|
\end{align}
Zatímco matice jedniček je singulární \todo{Pochopit proč se to dá spočítat, ale 
determinant vyjde kladně}.  \qed



\medskip
\subsection{Dvouvzdálenostní množiny}


\vt $P_1, P_2, \dots, P_m$ jsou body v $\R^n$ a $\exists \alpha, \beta \in \R$ t. že $\|P_iP_j\| \in {\alpha, \beta}$. Pak $m(n) \leq {(n+1)(n+4)\over 2}$.

\dk
\begin{align}
F(x,y) = (\|x,y\|^2-\alpha^2)(\|x,y\|)-\beta^2)&\qquad F: (\R^n \rightarrow \R) \\
f_i(x) = F(x, P_i)&\qquad f_i: \R^n \rightarrow \R
\end{align}

Když jsou $f_1, f_2, \dots, f_m$ lineárně nezávislé, pak $m \leq \dim($prostor funkcí $\R^n \rightarrow \R)$. Lineární kombinace $\sum_{i=1}^m \gamma_if_i(x) = 0$.

\begin{align}
f_i(P_j) &= \alpha^2\beta^2&\qquad {\rm pro}\ i=j\\
f_i(P_j) &= 0&\qquad {\rm pro}\ i\neq j
\end{align}

\begin{align}
\forall j: \sum_{i=1}^m \gamma_if_i(P_j) = \alpha^2\beta^2\gamma_j = 0 \qquad\Rightarrow\qquad \forall j: \gamma_j = 0
\end{align}

Z toho plyne, že funkce $f_1, f_2, \dots f_m$ jsou lineárně nezávislé.

\begin{align}
f_i(x) &= ((x_1-p_1)^2+\dots+(x_n-p_n)^2-\alpha^2) ((x_1-p_1)^2+\dots+(x_n-p_n)^2-\beta^2) \\
&= (x_1^2+\dots+x_n^2-2p_1x_1-\dots-2p_nx_n-\alpha^2) (x_1^2+\dots-2p_1x_1-\dots-\beta^2)
\end{align}

$p_i^2$ se ztratí do $\alpha$ a $\beta$. Následuj rozbor případů po roznásobení:

\begin{align}
	&(x_1^2+\dots+x_n^2)(x_1^2+\dots+x_n^2) &\qquad 1\\
	&(x_1^2+\dots+x_n^2)x_i &\qquad n\\
	&x_i^2 &\qquad n\\
	&x_ix_j &\qquad {n \choose 2}\\
	&x_i &\qquad n\\
	&1 &\qquad 1\\
\end{align}

Případ $(x_1^2+\dots+x_n^2)$ není potřeba, vyjádříme ho jako kombinaci $x_i^2$. Velikost lineárního obalu:

$${n\choose 2} + 3n + 2 = {n(n-1) \over 2} + {6n\over 2} + {4\over 2} = {n^2-5nn+4 \over 2} = {(n+1)(n+4)\over 2}$$ \qed


\vt Pro dvouvzdálenostní množinu na kouli platí: $${n(n+1)\over 2} \leq m_{sf}(n) \leq {n(n+3)\over 2}$$

\dk

{\bf Horní odhad} (ostatní řádky nepotřebujeme, $(x_1^2 + \dots + x_n^2)$ se na kouli posčítá na konstantu):
\begin{align}
	&x_i^2 &\qquad n\\
	&x_ix_j &\qquad {n \choose 2}\\
	&x_i &\qquad n\\
\end{align}

$${n\choose 2} + 2n = {n(n-1)\over 2} + {4n\over 2} = {n^2+3n\over 2} = {n(n+3) \over 2}$$

{\bf Dolní odhad} (konstrukce 2-vzdálenostní množiny v $\R^n$):

Body budou všechny vektory délky $n$ s dvěma jedničkovými souřadnicemi. Vzdálenost dvou bodů s $1$ na různých souřadnicích je $2$, zatímco vzdálenost bodů které se v jedné souřadnici shodují je $\sqrt 2$.

Uvažujme nyní body v $\R^{n+1}$ místo v $\R^n$. Takových je $n+1 \choose 2$. 

\bigskip
$\sum x_i^2 = 2 \Rightarrow$ všechny body leží na sféře\footnote{$x_i$ je $i$-tá souřadnice bodu $x$}\\
\indent$\sum x_i = 2 \Rightarrow$ všechny body leží v nadrovině \\

$$\left\{x | \sum x_i = 2 \right\} \cap \R^{n+1} \simeq \R^n$$

Tedy máme 2-vzdálenostní množinu $n+1 \choose 2$ bodů na kouli v $\R^n$.



\medskip
\subsection{Fišerova nerovnost}


\vt Nechť máme graf $K_n$ a jeho hranově disjunktní rozklad na $m$ úplných 
bipartitních grafů. Potom $m \geq n-1$.

\dk Označme si úplné bipartitní grafy $B_1, \ldots, B_m$ a $X_k$, $Y_k$ jejich 
partity, přičemž jednotlivý $B_i$ nemusí být pokrývat všechny vrcholy $K_n$.  
Mějme matici $A_k$ pro graf $B_k$ velikosti $n \times n$ definovanou:
\begin{align}
	a_{ij} = \left\{\begin{array}{ll}1 & \text{pokud } i \in X_k\text{ a }j \in 
	Y_k \\ 0 & \text{jinak} \end{array}\right.
\end{align}
Protože v každém nenulovém řádku jsou jedničky právě pro sousedy daného vrcholu 
v druhé partitě, jsou všechny nenulové řádky stejné (sousedství jsou stejná), 
$A_k$ má tedy hodnost $1$.

Nyní uvažme matici $A=A_1 + \ldots + A_m$. Hodnost součtu je nanejvýš rovna 
součtu hodností, proto $\rank(A) \leq m$. Nyní budeme chtít dokázat, že 
$\rank(A) \geq n-1$:

Protože každá hrana grafu náleží právě jednomu $B_k$, je jednička právě na 
jednom z míst $a_{ij}$ nebo $a_{ji}$ (pozor, matice nejsou matice sousednosti -- 
rozlišují partitu!). Na diagonále $A$ jsou pak samé nuly. Sečtením $A+A^T$ 
získáme matici incidence $K_n$, tedy $A+A^T=J_n - I_n$.

Dále pro spor předpokládejme, že $\rank(A) \leq n-2$. Připíšeme k matici jeden 
řádek samých jedniček, čímž hodnost zvýšíme nanejvýš o $1$. Protože ale $A$ nemá 
plnou hodnost, existuje netriviální lineární kombinace sloupců, která dává 
$\vec{0}$ -- nechť jsou její koeficienty zaznamenány ve vektoru $\vec{x} \in 
\R^n$ a tedy $A\vec{x}=\vec{0}$. Zároveň protože poslední řádek jsou samé 
jedničky, platí $\sum x_i \cdot 1 = 0$ a tedy také $J_n\vec{x} = 0$. Počítejme 
dvěmi způsoby:

\begin{align}
	&x^T(A + A^T) x = x^T(J_n - I_n)x = x^T(J_nx) - x^T(I_nx) = 0 - x^T x = - 
	\sum x_i^2 < 0 \\
	&x^T(A+A^T) x = x^TA^Tx + x^TAx = 0^Tx + x^T0 = 0
\end{align}

což dává spor. \qed





\section{Skalární součin}

\df (Skalární součin) Dvěma vektorům $x,y$ z~vektorového prostoru $V = \mathbb{F}^n$
přiřadíme skalární součin $\sk{x,y} = \sum x_iy_i$ \quad(případně $\sk{x,y} = \sum
x_i\overline{y_i}$ pro $\mathbb{F} = \C$).

\subsection{Ortogonální doplněk}

\df $M \subseteq \mathbb{F}^n$ \quad $M^\bot = \{x \mid \forall a\in M\colon \sk{x,a} = 0\}$ je
ortogonální doplněk $M$.

\poz $\dim M^\bot = n - \dim \L M$

\df (Součet podprostorů) $\L M + \L N = \left\{ u + v \mid u \in \L M, v \in \L N \right\} = \L{(M\cup N)}$

\poz $(A \cap B)^\bot = A^\bot + B^\bot$

\poz $(A + B)^\bot = A^\bot \cap B^\bot$

\poz $\dim(M+M^\bot) = n$ a také $M + M^\bot = \mathbb{F}^n$

\poz ${(M^\bot)}^\bot = \L M$

\dk \uv{$\supseteq$} jednoduché, \uv{$\subseteq$} přes dimenze\quad $n-(n-k) = k =
\dim \L M$ \qed

\poz $\dim\left(\L M + \L N\right) + \dim\left(\L M \cap \L N\right) = \dim \L M + \dim \L N$

Pozor, pro tři podprostory už předchozí pozorování neplatí! Například v~rovině tři
přímky $U,V,W$ procházející počátkem, pak $\dim(U + V + W) \neq \dim(U) + \dim(V) +
\dim(W) - \dim(U \cap V) - \dim(U \cap W) - \dim(V \cap W) + \dim(U \cap V \cap W)$
(čísly $2 \neq 1 + 1 + 1 - 0 - 0 - 0 + 0$).

\dsl Mějme podprostory $M,N \ll \mathbb{F}^n$, ve kterých platí $\dim M + \dim N > n$,
pak $\dim M \cap N \ge 1$, tedy $\exists u \neq 0, u \in M\cap N$.

\dsl Navíc pro tělesa, ve kterých standardní skalární součin je opravdu skalárním
součinem, tedy $\sk{x,x} \neq 0$ pro $x\neq 0$, máme:  $M\cap M^\bot = \{0\}$.

Například $\GF(2)^2$ předchozí podmínku nesplňuje a dokonce platí $\sk{(1,1), (1,1)} =
0$, tedy vektor $(1,1)$ je kolmý sám na sebe v~$\GF(2)^2$.



\medskip
\subsection{Sudo-sudo města}


V~následující větě zachováme značení z~kapitoly o~lineární nezávislosti.

\vt Pro sudo-sudo město platí $m_{\max} = 2^{\lfloor n / 2 \rfloor}$.

\dk Nejprve sestrojíme sudo-sudo město o~velikosti $m=\lfloor\frac n2\rfloor$. Rozdělíme prvky množiny $X$ do dvojic (pokud jeden přebývá, odložíme ho stranou a dále se jím nebudeme zabývat) a za $A_1,\dots,A_m$ vezmeme všechny neprázdné podmnožiny množiny $X$, které obsahují z~každé dvojice buď oba prvky, nebo žádný. Takových podmnožin je $2^{\lfloor n / 2 \rfloor}$ a evidentně se jedná o~sudo-sudo město.

Nyní ukážeme nerovnost $m\leq\lfloor\frac n2\rfloor$. Nechť $M=\{A_1, A_2, \dots, A_m\}$ je co do inkluze maximální sudo-sudo město. Ztotožníme-li množiny $A_i$ s~jejich charakteristickými vektory, pak pro všechna $i,j\in\{1,\dots,n\}$ je $\sk{A_i,A_j} \text{ mod }2=0$. Tedy $M$ je vektorový prostor nad $\GF(2)$, neboť platí:
\begin{align*}
	&\emptyset \in M, \\
	&\forall u~\in M, \forall c\in \GF(2)\colon c\cdot u~\in M, \\
	&\forall x, \forall u,v\in M\colon \sk{x, u+v} = \sk{x,u} + \sk{x,v} = 0 + 0 = 0, \\
	&\forall u,v \in M\colon \sk{u+v,u+v} = \sk{u,u} + 2\sk{u,v} + \sk{v,v} = 0 + 0 + 0 = 0.
\end{align*}
Pokud $x\in M$, pak také $x\in M^\bot$, a tedy $M\subseteq M^\bot$. To znamená, že
\begin{align}
\dim M\leq\dim M^\bot=n-\dim M, \qquad\text{ekvivalentně } \dim M\leq\left\lfloor\frac n2\right\rfloor.
\end{align}
Jelikož $M \subseteq\GF(2)^n$ a $\dim M \leq\lfloor\frac n2\rfloor$, je $|M|=m \leq 2^{\lfloor n/2\rfloor}$. \qed




\medskip
\subsection{Eulerovské a úplné bipartitní podgrafy}


$G = (V,E)$ je souvislý graf. $V_G = \{$ spanning\footnote{Česky též \uv{napnuté} --
podgrafy obsahující všechny vrcholy grafu $G$ (i kdyby některé z~nich byly
izolované).} podgrafy $G$ $\}$

\tv $V_G$ je vektorový prostor nad $\GF(2)$, místo sčítání vektorů je symetrická
diference. $V_H \in \GF(2)^E$.

\bigskip
\todo Obrázek se symetrickou diferencí podgrafů.
\bigskip

\df $\varepsilon_G = \{ $ eulerovské podgrafy $\equiv
\forall $ stupně sudé $ \}$. Součtem dvou eulerovských podgrafů je eulerovský
podgraf, tvoří tedy podprostor $V_G$.

\lm $\dim \varepsilon_G = |E| - n + 1$

\dk Vybereme si libovolnou kostru $T$ grafu $G$. Pro každou hranu, která není
v~kostře existuje právě jedna elementární kružnice $K_e$ určená touto hranou. $\{
K_e\ |\ e \in E(G) - E(T) \}$ tvoří lineárně nezávislé vektory. Lze dokázat,
že tvoří bázi $\varepsilon_G$.

Z~toho $\dim \varepsilon_G = |E| - n + 1$, což je počet hran mimo kostru. \qed

\df $\beta_G = \{$ úplné bipartitní spanning podgrafy $G$ $\}$. $\beta_G$ je
prostor všech řezů v~$G$.

\lm $\beta_G \ll V_G$, $\beta_G = \langle\{$ hvězdy $\}\rangle$

\dk Každý úplný bipartitní podgraf lze zapsat jako symetrickou diferenci hvězd.
Vezmeme hvězdy ze všech vrcholů v~jedné z~partit. Mezi těmito vrcholy se hrany
vyruší, mezi vrcholy z~druhé partity žádné nevedou a všude jinde ano. 

Mám-li dva různé úplné bipartitní podgrafy, rozepíšu si je na součet hvězd a
výsledkem musí být dle výše uvedeného opět úplný bipartitní podgraf.

\vt $\varepsilon_G^\bot = \beta_G$. Tedy eulerovské podgrafy jsou ortogonálním
doplňkem úplných bipartitních podgrafů.

\dk Vezmeme si $H \in \varepsilon_G$ eulerovský podgraf a $u \in V(G)$. $H_u$
označíme hvězdu z~vrcholu $u$. Platí $\sk{H,H_u} = \deg_H u$, neboť hvězda
obsahuje všechny hrany jdoucí z~$u$ a žádné jiné. Protože v~$H$ vychází
z~každého vrcholu sudý počet hran a počítáme nad $\GF(2)$:
\begin{align*}
\forall u: \sk{H,H_u} = 0 \quad\Rightarrow\quad \forall B\in \beta_G: \sk{H,B} = 0 \quad\Rightarrow\quad H \in \beta_G^\bot \quad\Rightarrow\quad \varepsilon_G \subseteq \beta_G^\bot
\end{align*}

Naopak, každý podgraf $H$, který je kolmý na všechny hvězdy je nutně eulerovský:
\begin{align*}
\forall u: \sk{H,H_u} = 0 \quad\Rightarrow\quad \forall u: \deg_H u~\equiv 0 \mod 2 \quad\Rightarrow\quad H \in \varepsilon_G \quad\Rightarrow\quad \beta_G^\bot \subseteq \varepsilon_G
\end{align*}

Tedy $\varepsilon_G = \beta_G^\bot$, protože $\varepsilon_B^\bot = {\left(\beta_G^\bot\right)}^\bot = \beta_G$.
\qed

\dsl $\dim \varepsilon_G = \dim \beta_G^\bot = |E| - n + 1$.


\vt $M \subseteq \GF(2)^n \quad\Rightarrow\quad (1,1,\dots,1) \in \sk M + M^\bot = \sk{M\cup M^\bot}$

\dk $\sk M \cap M^\bot$
\begin{enumerate}
\item[(a)] $\dim(\sk M \cap M^\bot) = 0 \quad\Rightarrow\quad \dim(\sk M + M^\bot) = k+n-k = n \Rightarrow \sk M + M^\bot = \GF(2)^n \Rightarrow (1,1,\dots,1)\in \sk M + M^\bot$
\item[(b)] $\dim(\sk M \cap M^\bot) > 0 \quad\Rightarrow\quad \exists u\in \sk M \cap M^\bot$ \\ 
$\forall u\in \sk M \cap M^\bot: \sk{u,u} = 0 \Rightarrow \sum u_i^2 \equiv 0 \mod 2 \Rightarrow \sum u_i = \sk{u, (1,1,\dots,1)}$ \\
nad $\GF(2)$ platí $u_i^2 = u_i$ \\
$\Rightarrow (1,1,\dots,1) \in {(\sk M \cap M^\bot)}^\bot = {\sk M}^\bot + {(M^\bot)}^\bot = M^\bot + \sk M$
\end{enumerate}
\qed


\vt $\forall G \ \exists V_1,V_2,\ V_1\overset{.}{\cup} V_2 = V(G)$ takové, že
$G[V_1]$ i $G[V_2]$ mají všechny stupně sudé.

\dk $M = \varepsilon_G \ll V_G$

$G = (1,1,\dots,1) \in \varepsilon_G + \varepsilon_G^\bot = \varepsilon_G + \beta_G$
\qed

\dsl $\exists H \in \varepsilon_G\ \exists B\in \beta_G: G = H + B$ (tedy každý graf lze zapsat jako symetrickou diferenci eulerovského podgrafu a hranového řezu).





\section{Shannonova kapacita a Lovászova $\vartheta$ funkce}

\subsection{Shannonova kapacita}


\df Domečkový součin grafů $G$ a $H$ je graf $G \boxtimes H$ takový, že:
\begin{align*}
	&V(G \boxtimes H) = \{ (u,v) \ |\  u\in V(G), v\in V(H) \} \\
	&E(G \boxtimes H) = \{ ((u_1,v_1),(u_2,v_2))\} \left\{\begin{matrix}
		&u_1 = u_2, v_1 \sim v_2 &\text{(sousedí)} \\
		&v_1 = v_2, u_1 \sim u_2 \\
		&v_1 \sim v_2, u_1 \sim u_2
		\end{matrix}\right.
\end{align*}

Motivací ke zkoumání Shannonovy kapacity grafu může být posílání zpráv.
Potřebujeme-li kód, který opraví jednu chybu, můžeme na $C_5$ najít pouze dvě
kódová slova ($\alpha(C_5) = 2$). Naproti tomu, $\alpha(C_5 \boxtimes C_5) = 5
> 2^2$. Posílání zpráv ve větších blocích tedy může být efektivnější.

\df Shannonova kapacida grafu:
$$\shn(G) = \sup_{i\ge 1}(\alpha(G^i))^{1/i}$$

\lm $\shn(G\boxtimes H) \ge \shn(G) \cdot \shn(H)$

\dk Vezměme si maximální nezávislou množinu v $G$ a maximální nezávislou
množinu v $H$. Z vlastností domečkového součinu plyne, že mezi vrcholy
$G\boxtimes H$ zkombinovanými z těchto množin nepovede žádná hrana a tudíž
budou tvořit nezávislou množinu velikosti alespoň $\alpha(G)\cdot\alpha(H)$.

\poz $\shn(G^i) \ge \shn(G)^i$

\dk Postupnou iterací lemmatu.

\df Ortonormální reprezentace grafu $G$ je funkce $\rho: V(G) \rightarrow \R^d$,
$\|\rho(v)\| = 1$. Pro každé $(u,v) \not\in E(G)$ platí $\rho(u)\bot\rho(v)$,
neboli $\sk{\rho(u), \rho(v)} = 0$.

\df Lovászova theta funkce:
$$\vartheta(G,\rho) = \max_{v\in V(G)} {1\over \sk{\rho(v),e_1}^2}$$

Vezmeme si reprezentaci grafu $C_5$ ta se skládá z pěti vektorů $v_1, \dots,
v_5$ a jednoho speciálního vektoru $e_1$, vůči kterému budeme ostatní
vztahovat. Protože se jedná o ortonormální reprezentaci, musí každé dva
nesousední vrcholy z $C_5$ svírat pravý úhel. Představíme si \uv{paraplíčko},
kde vektor $e_1$ tvoří držadlo a vektory $v_1, \dots, v_5$ jsou okolo něj a
tvoří dráty deštníku. Představme si dále, že deštník roztahujeme, dokud nebudou každé dva nesousední dráty svírat pravý úhel. Pak můžeme spočíst úhel mezi dráty a držadlem, který vyjde $\sk{\rho(v),e_1} = 5^{-{1\over 4}}$. Z toho:
$$
\vartheta(C_5,\rho) = \sqrt 5
$$

\df $\vartheta(G) = \min\limits_{\rho\ \text{ONR}} \vartheta(G,\rho)$

Z toho plyne $\vartheta(C_5)\le \sqrt 5$. Kdybychom ještě znali vztah mezi
$\shn(G)$ a $\vartheta(G)$, měli bychom vyhráno. Tuto charakterizaci přináší
následující věta.

\vt $\shn(G) \le \vartheta(G)$

\dk K důkazu věty budeme potřebovat dvě pomocná lemmata.


\lm (O vztahu $\vartheta$ a $\alpha$) Nechť $H$ je graf a $\rho$ nějaká jeho ortonormální reprezentace. Pak $\alpha(H) \leq \vartheta(H, \rho)$.

\dk Nechť $A$ je nějaká nezávislá množina $H$. Zřejmě vektory $\rho(v)$ pro $v \in A$ tvoří ortonormální systém vektorů. Přáli bychom si odhadnout, jak velký bude skalární součin $\sk{\rho(v),e_1}^2$, z čehož nám vztah vyplyne.

Nechť $u$ je libovolný vektor a $b_i$ jsou vektory ortonormální báze. Chceme-li vyjádřit vektor $u$ proti bázi $b_i$, získáme $i$-tou souřadnici skalárním součinem $\sk{b_i,u}$ (můžeme si to představovat tak, že z vektorů $b_i$ složíme matici předhocu). Použijeme-li Pythagorovu větu, získáme:
\begin{align}
	||u||^2 = \sum_{i=1}^n \sk{b_i,u}^2
\end{align}
Pokud aplikujeme tento poznatek na vektory $\rho{v}$ rozšířené na bázi (což jistě lze), a vektor $e_1$, rovnost se změní na nerovnost (nezajímají nás přidané vektory) a s vědomím, že všechny vektory máme ortonormální, získáme:
\begin{align}
	1=||u||^2 \geq \sum_{v\in A}\sk{\rho(v),e_1}^2
\end{align}
Tedy existuje alespoň jeden vrchol $w$, že $\sk{\rho(w),e_1}^2 \leq 1/|A|$ a dosadíme-li do zlomku z definice $\vartheta$, získáme odhad $ \alpha(G) = |A| \leq\vartheta(H,\rho)$, což jsme chtěli dokázat. \qed



\lm (O součinu $\vartheta$) Nechť $H_1$ a $H_2$ jsou grafy, a $\rho_i$ jejich 
ortonormální reprezentace. Potom existuje ortonormální reprezentace $\rho$ 
silného součinu $H_1 \boxtimes H_2$, pro niž platí $\vartheta(H_1 \boxtimes H_2, 
\rho) = \vartheta(H_1, \rho_1) \cdot \vartheta(H_2,\rho_2)$.

\dk Zadefinujme si funkci $\rho$ pro vrcholy $v_i$ následovně:
\begin{align}
	\rho(v) = \rho_1(v_1) \otimes \rho_2(v_2)
\end{align}
Kde operace $\otimes$ je tenzorový součin vektorů, tedy pro $x \in \R^n$ a $y 
\in \R^m$ je výsledek vektor $z \in \R^{mn}$, který obsahuje všechny součiny 
$x_iy_i$. 

Zbývá pouze ověřit, že dělá správnou věc. Podívejme se tedy nejdříve na skalární 
součin:
\begin{align}
	\sk{x \otimes y , x' \otimes y'} = \sk{x | x'} \cdot \sk{y | y'}
\end{align}
Pokud levou a pravou stranu zvlášť rozepíšeme, je vidět, že roznásobením sum 
napravo získáme sumu nalevo a rovnost tedy platí:
\begin{align}
	\sum_{ij} (x_iy_j) \cdot (x_i' y_j') = \left( \sum_i x_ix_i'\right) \left(\sum_j 
	y_jy_j' \right)
\end{align}
Zde již jednoduchou úvahou zjistíme, že $\rho$ je stále ortonormální 
reprezentace: zjevně pro kolmé vektory jsou opět kolmé, a všechny vektory si 
zachovají délku $1$. Nyní se stačí podívat, co se stane s $\vartheta$ funkcí, 
rozepišme si ji ted z definice:
\begin{align*}
	\vartheta(H_1 \boxtimes H_2, \rho) &= \max_{v\in V(H_1 \boxtimes H_2)} { 1 
	\over \sk{\rho(v) , e_1}^2} \\
	&= \max_{v\in V(H_1 \boxtimes H_2)} { 1 \over \sk{\rho_1(v_1)\otimes \rho_2(v_2) , e_{11} \otimes e_{12}}^2} \\
	&= \max_{v\in V(H_1 \boxtimes H_2)} { 1 \over \sk{\rho_1(v_1) , e_{11}}^2
		\cdot \sk{\rho_2(v_2) , e_{12}}^2} \\
	&= \max_{v_1\in V(H_1)} { 1 \over \sk{\rho_1(v_1) , e_{11}}^2} \cdot
	  \max_{v_2\in V(H_2)} { 1 \over \sk{\rho_2(v_2) , e_{12}}^2} 
	&= \vartheta(H_1,\rho_1) \cdot \vartheta(H_2, \rho_2)
\end{align*}
A lemma je dokázáno. \qed


\dk (Věty o vztahu $\Theta$ a $\vartheta$)  $$\alpha(G^i) \le \vartheta(G^i) \le \vartheta(G)^i$$
První nerovnost plyne z lemma o vztahu $\vartheta$ a $\alpha$. Druhá plyne z
opakovaného použití lemma o součinu $\vartheta$.
\qed


\lm (O dvojité kapacitě) $\shn(G + \overline{G}) \ge \sqrt{2|G|}$

\dk Ukážeme, že $\alpha ((G+\overline G)^2) \geq 2|G|$.
\begin{align*}
	V_{G+\overline G} = \{ v_1, ..., v_n, v_1', ..., v_n'\}
\end{align*}

Vezeme graf $(G+\overline G)^2$ a najdeme v něm nezávislou množinu $A$:
\begin{align*}
	A = \left\{\begin{matrix}
		(v_1, v_1'), (v_2, v_2'), \dots \\
		(v_1', v_1), (v_2', v_2), \dots
		\end{matrix}\right\}
\end{align*}

Velikost $A$ je zřejmě $2|G|$ a z definice Shannonovy kapacity dostaneme:
$$\shn(G + \overline{G}) \ge \sqrt{2|G|}$$
\qed



\subsection{Funkční reprezentace grafu}

\df Nechť $G$ je graf, $X$ množina, $\Ft$ těleso a $\F\colon X\rightarrow\Ft$ systém funkcí. Pak pro vrchol $v$ mějme $c_v\in X$ a $f_v \in \F$, že
$f_v: X \to \Ft$ a platí:
\begin{enumerate*}
	\item $f_v(c_v) \neq 0$
	\item $uv \notin E_G \Rightarrow f_u(c_v) = 0$
\end{enumerate*}
Jinými slovy, pro funkci $f_a$ vrcholu $a$ platí, že vrchol dostane vždy nenulový prvek a jeho nesousedi vždy nulový. O sousedech nehovoříme nic.

\df Dimenzi $\F$ definujeme jako $\dim\calL(\{f_v\})$, tedy chápeme funkce
jako vektorový prostor.

\lm(O vztahu $\alpha$ a $\dim\F$) G má reprezentaci $\F$, pak $\alpha(G)
\leq \dim \F$.

\dk Nechť $A$ je nezávislá v $G$. Pak $\{f_a\}_{a\in A}$ je lineárně
nezávislá, stejně jako $\{c_a\}_{a\in A}$. Vyhodnotím reprezentující
funkci v bodech $A$.
\begin{align}
M = \left(
	\begin{matrix}
		f_1(c_1) & f_1(c_2) & \dots \\
		f_2(c_1) & f_2(c_2) & \dots \\
		\vdots &&\\
	\end{matrix}\right)
\end{align}

Matice $M$ bude mít na diagonále nenuly a všude jinde nuly. Tím pádem
jsou její řádky lineárně nezávislé a její dimenze je $|A|$. Navíc zjevně
$\dim M \le \dim\F$.
\qed


\lm(O dimenzi součinu reprezentací) Pokud $G_1$ má reprezentaci $\F_1$,
$G_2$ reprezentaci $\F_2$ nad stejným tělesem, pak $G = G_1 \boxtimes
G_2$ má reprezentaci $\F$,pro kterou platí $\dim\F \leq \dim\F_1 \cdot \dim\F_2$.

\dk Definujeme: 
\begin{align*}
& X = X_1 \times X_2 \\
& c_{(v_1,v_2)} = (c_{v_1}, c_{v_2}) \\
& f_{(v_1, v_2)}((x_1,x_2)) = f_{v_1}(x_1) \cdot f_{v_2}(x_2)
\end{align*}

Ověříme, že výše uvedené je funkční reprezentace a vezmeme si $B_1$ bázi
$\F_1$ a $B_2$ bázi $\F_2$. Pak $\{b_1 \otimes b_2\}_{b_1\in B_1, b_2\in
B_2}$ generuje celý prostor $\F$ a tudíž:
$$
	\dim\F \le |B_1|\cdot |B_2| = \dim\F_1 \cdot \dim\F_2
$$
\qed

\lm(O vztahu $\shn$ a $\dim\F$) G má reprezentaci $\F$, pak $\shn(G)
\leq \dim \F$.

\dk 
$$\shn(G) = \sup_i \alpha(G^i)^{1/i} \leq \sup_i(\dim f.r.(G^i))^{1/i} \leq \sup_i\dim f.r. (G) = \dim f.r.(G)$$

První nerovnost plyne z lemmatu o vztahu $\alpha$ a $\dim\F$, druhá z
lemmatu o dimenzi součinu reprezentací.
\qed


\vt Existuje $G, H$, že $\shn(G+H) > \shn(G) + \shn(H)$

\dk Zvolím $G$ takový, že $V_G={S\choose 3}$, $S = \{1, \dots, s\}$ a $E_G = \{ (A, B): |A\cap B| = 1\}$. 

Reprezentaci vytvoříme nad tělesem $\Ft = \Z_2$, $X = \Z_2^s$:
\begin{align*}
	c_A &= \text{charakteristický vektor } A \\ 
	f_A(x) &= \sum_{a\in A} x_a
\end{align*}

Ověříme, že se jedná o funkční reprezentaci a všimneme si, že každá
funkce $f_A$ je kombinace tří funkcí $b_i(x) = x_i$, přičemž počet funkcí
$b_i$ je $s$.
$$
\dim f.r.(G) \le s \qquad\Rightarrow\qquad \shn(G) \le s
$$

Dále pro $H = \overline G$ zvolíme reprezentaci pro $\Ft = \R$, $X =
\R^s$:
\begin{align*}
	c_A &= \text{charakteristický vektor } A \\ 
	f_A(x) &= (\sum_{a\in A} x_a) - 1
\end{align*}

Opět ověříme, že se jedná o funkční reprezentaci.
$$
\dim f.r.(\overline G) \le s+1 \qquad\Rightarrow\qquad \shn(\overline G) \le s+1
$$

$$\shn(G + \overline G) \ge \sqrt{2{s\choose 3}} > 2s+1 \ge \shn(G) + \shn(\overline G)$$

První nerovnost platí z lemmatu o dvojité kapacitě a ostrou nerovnost
musíme splnit, aby věta platila. Zvolíme si tedy $s \ge 16$.
\qed

\df Obecná poloha vektorů množiny $\check N$ v  $\R^d$ je taková, že libovolná podmnožina velikosti $\leq d$ je lineárně nezávislá.

\df Lokálně obecná poloha vektorů reprezentace v $\R^d$ na grafu $G$ jsou takové vrcholy, že $\rho(\overline{N(v)})$ jsou lineárně nezávislé.

\vt Pro $G$ s $|G| = n$ jsou následující tvrzení ekvivalentní:
\begin{enumerate}
	\item $G$ má ortogonální reprezentaci v $\R^d$ v obecné poloze.
	\item $G$ má ortogonální reprezentaci v $\R^d$ v lokálně obecné poloze.
	\item $G$ je $(n-d)$-souvislý.
\end{enumerate}





\section{Vlastní čísla grafu}

\subsection{Vlastní čísla matic}


\df Nechť $A$ je čtvercová matice. Potom pokud pro nějaké $\lambda$ a $x$ netriviální platí, že $Ax=\lambda x$ říkáme, že $\lambda$ je vlastní číslo $A$ a $x$ je vlastní vektor příslušící k $\lambda$.

\df Spektrum matice $A$ je množina jejích vlastních čísel. Značíme $\Sp(A) = \{ \lambda_1, \ldots, \lambda_n \}$.

\df Podprostorem generovaným vlastním číslem $\lambda$ rozumíme $V_\lambda = \{u | Au = \lambda u \}$. Geometrická násobnost $\lambda$ je poté dimenze tohoto prostoru $V_\lambda$.

\tv $V_\lambda$ je vektorový prostor.

\dk Stačí dokázat uzavřenost. Pro $u,v\in V_\lambda$ počítejme:
\begin{align}
	A(u+v) = Au + Av = \lambda u + \lambda v = \lambda(u+v)
\end{align}
Tedy i $u+v\in V_\lambda$. \qed

\tv Vlastní čísla matice $A$ lze vypočítat jako kořeny rovnice $\det(A - \lambda \cdot E) = 0$.

\dk Z definice počítejme:
\begin{align}
	Au &= \lambda u \\
	Au - \lambda u &= \vec0 \\
	(A-\lambda) u &= \vec0 \\
	\det(A - \lambda E) &= 0 
\end{align}
Přičemž v posledním kroku využíváme faktu, že pro součin netriviálního vektoru s maticí musí být matice singulární, aby mohl vyjít nulový vektor a tudíž můžeme přejít k determinantu. \qed

\df Polynomu $P_A(\lambda) = \det(A - \lambda \cdot E)$ říkáme charakteristický polynom.

\df Násobnosti kořene $\lambda$ v polynomu $P_A$ říkáme {\it algebraická násobnost}.

\vt Nechť $GN(\lambda)$ a $AN(\lambda)$ značí geometrickou, resp. algebraickou násobnost $\lambda$. Potom platí:
\begin{align}
&GN(\lambda) \geq 1 \Leftrightarrow \lambda \in \Sp(A) \Leftrightarrow AN(\lambda) \geq 1\\
\text{a}\qquad &GN(\lambda) \leq AN(\lambda)
\end{align}

\dk {\it (bez důkazu)}

\df Hermitovská transpozice matice $A$ je matice $A^*$, taková, že $A_{ij}^* = \overline{A_{ji}}$.

\df Matice $A \in \C^{n\times n}$ je {\it normální}, pokud $AA^* = A^*A$.

\vt Matice $A$ má ortonormální bázi složenou z vlastních vektorů právě tehdy, když je $A$ normální.

\dk \begin{description}
	\item \uv{$\Rightarrow$} Nechť $x_i$ jsou vlastní vektory příslušející vlastním číslům $\lambda_i$ tvořící ortonormální bázi. Z ortonormality plyne, že $XX^*=E$, kde $X$ má ve sloupcích $x_i$. Podívejme se nyní jak vypadá matice $X^* A X$:
	\begin{align}
		X^* A X = 
		\underbrace{
		\left(\begin{array}{ccc} &\vdots &\\ \hline \hspace{1cm}& x_j^* & \hspace{1cm}\\ \hline&\vdots& \end{array}\right)}_{=X^*} 
		\underbrace{\left(\begin{array}{c|c|c} \raisebox{5mm}{\ } & &\\ \dots& \lambda_i x_i & \dots\\\raisebox{5mm}{\ }  && \end{array}\right)}_{=AX}
		= \left(\begin{array}{ccc}\lambda_1 &&\bigzero \\ &\ddots & \\ \bigzero&&\lambda_n\end{array}\right)
	\end{align}
	Přičemž druhá matice vznikla ze vztahu $Ax=\lambda x$, přičemž jsme vynásobili všechny vektory naráz díky tomu, že byly v matici. Poslední rovnost plyne z pozorování, že na pozici $ij$ nalezneme výraz $x_j^*\lambda_i x_i = x_j^* x_i \lambda_i$ a protože vektory $x_l$ tvoří ortonormální bázi, jsou nula pokud je $i\neq j$ a jedna jinak.

	Nyní víme, že $X^*AX=D$, kde $D$ je nějaká (konkrétní) diagonální matice. Nyní již snadno vypočteme elementárními úpravami:
	\begin{align*}
		X^* A X = D \quad \Rightarrow \quad AX &= XD \quad \Rightarrow A = XDX^* \\
		A\cdot A^* = XD\underbrace{X^*\cdot X}_ED^*X^* = XDD^*X^*
		&= XD^*DX^* = XD^*\underbrace{X^*\cdot X}_EDX^* = A^*\cdot A
	\end{align*}
	Přičemž jediná finta, kterou jsme použili je, že $DD^* = D^*D$, což je zřejmě pravda, protože jsou to diagonální matice.

	\item \uv{$\Leftarrow$} \todo{gavento} byl jen pochybny naznak
\end{description}

\vt Nechť $A_i \in \C^{n\times n}$ a $\forall i,j$ jsou $A_i$ a $A_j$ normální a $A_iA_j = A_j A_i$. Potom existuje společná ortonormální báze z vlastních vektorů.

\dk \todo{Gavento}

\vt Nechť $A$ je hermitovská matice, tedy $A = A^*$. Potom všechna její vlastní čísla jsou reálná.

\dk Víme, že existuje nějaké $D$ diagonální s vlastními čísly na diagonále a $X$, že $X^*AX = D$. Dále počítáme:
\begin{align}
	D^* = (X^*(AX))^* = (AX)^*X = X^*A^*X = X^*AX = D
\end{align}
A komplexní sdružení tedy nesmí udělat žádnou operaci, tedy jsou vlastní čísla reálná. \qed



\subsection{Mooreovy grafy}
% Stanislava Tlustá

% úvod
% % motivace
V této části využijeme znalostí o vlastních číslech k přiblížení toho, jak mohou vypadat regulární grafy bez krátkých cyklů.

% % Mooreova podmínka
Nejprve si ukažme kolik může mít takový $r$-regulární graf bez krátkých cyklů (troj- a čtyřúhelníků) vrcholů. 
Vezměme jeden vrchol. Ten má $r$ sousedů, z nichž žádné dva spolu nesousedí (vznikl by trojúhelník). Každý z nich má $r-1$ nových sousedů. Ti musí být různí (vznikl by čtyřúhelník).

\begin{align}
\label{4-2:graf-Petersen}
\begin{tikzpicture}[thick,scale=1.1]
% obrázek: Petersenův graf (jako Mooreův)
% 1. vrstva
\draw (0,0) -- (3,2) (0,0) -- (3,0) (0,0) -- (3,-2);
\draw (0,0) \vrchol;
% 2. vrstva - 1. vrchol
\draw (3,2) -- (6,2.5) \vrchol (3,2) -- (6,1.5) \vrchol;
\draw (3,2) \vrchol;
% 2. vrstva - 2. vrchol
\draw (3,0) -- (6,0.5) \vrchol (3,0) -- (6,-0.5) \vrchol;
\draw (3,0) \vrchol;
% 2. vrstva - 3. vrchol
\draw (3,-2) -- (6,-1.5) \vrchol (3,-2) -- (6,-2.5) \vrchol;
\draw (3,-2) \vrchol;
% 3. vrstva - přerušované čáry
\draw[black!50, dashed] (6,0.5) to[bend right] (6,2.5);
\draw[black!50, dashed] (6,-2.5) to[out=20,in=340] (6,2.5);
\draw[black!50, dashed] (6,-0.5) to[bend right] (6,1.5);
\draw[black!50, dashed] (6,-1.5) to[out=20,in=340] (6,1.5);
\draw[black!50, dashed] (6,-1.5) to[bend right] (6,0.5);
\draw[black!50, dashed] (6,-2.5) to[bend right] (6,-0.5);
% popisky (dole)
\draw (0,-3) node {1. vrchol};
\draw (3,-3) node {2. vrstva};
\draw (3,-3.5) node {r sousedů};
\draw (6,-3) node {3. vrstva};
\draw (6,-3.5) node {r(r-1) vrcholů};
%% alternativa
%%% popisky (nahoře)
%\draw (0,3.5) node {1. vrchol};
%\draw (3,3.5) node {2. vrstva};
%\draw (3,3) node {r sousedů};
%\draw (6,3.5) node {3. vrstva};
%\draw (6,3) node {r(r-1) vrcholů};
\end{tikzpicture}
\end{align}
Dostáváme tedy odhad:
\begin{align}
%\label{mooreova-podminka}
|V| \geq 1 + r + r(r-1) = r^2 + 1
\end{align}

% % definice
\df Mooreův graf je takový $r$-regulární graf na $r^2 + 1$ vrcholech, který neobsahuje troj- a čtyřúhelníky.

Mooreovy grafy jsou tedy nejmenší možné regulární grafy bez krátkých cyklů. Podle následující věty není však až na výjimky možno tohoto ideálu dosáhnout.

\vt Nechť $G$ je $r$-regulární Mooreův graf. Pak $r \in \set{1, 2, 3, 7, 57}$.

\pzn \\
Pro $r = 1$ je hledaným grafem jedna hrana. \\
Pro $r = 2$ je to pětiúhelník. \\
Pro $r = 3$ je to Petersenův graf (viz obrázek \ref{4-2:graf-Petersen}). \\
Pro $r = 7$ je to takzvaný Hoffman-Singletonův graf. \\
Pro $r = 57$ není zatím známo, zda lze takový graf skutečně sestrojit.

\dk
% jak to budeme dělat
Pomocí poznatků o vlastních číslech získaných v předchozí části sestavíme rovnici, která nám přesně vymezí, co musí $r$ splňovat.

Matici souslednosti grafu $G$ označme $A$. Její druhá mocnina zachycuje počet sledů délky 2. Má tedy na diagonále stupeň ($r$). Ukážeme, že mimo diagonálu má oproti matici $A$ prohozené nuly a jedničky.
Je-li $uv \in E(G)$, pak mezi $u$ a $v$ nemůže existovat cesta délky $2$ (vznikl by trojúhelník).
Pokud $uv \not\in E(G)$, tak se podíváme na konstrukci na obrázku \ref{4-2:graf-Petersen}. Bez újmy na obecnosti můžeme předpokládat, že $u$ je 1. vrchol. Vidíme, že existuje cesta délky $2$. Ta může být jen jedna, jinak by vznikl čtyřúhelník.

Dostáváme tedy:
\begin{align}
A + A^2 = rE + (J - E),
\end{align}
kde $E$ je jednotková matice a $J$ je matice samých jedniček. Úpravou dostaneme polynomiální vztah
\begin{align}
p(A) = A^2 + A + (1-r)E = J.
\end{align}
% spektra
Dále platí, že pro $\lambda \in \Sp(A)$ je $p(\lambda) \in \Sp(J)$. Spektrum $J$ známe: obsahuje $n = r^2 + 1$ s násobností $1$ a $0$ s násobností $n-1$.

% % \lambda = r
Protože $G$ je $r$-regulární, tak je $r$ také jeho vlastním číslem. Dosazením do $p$ dostaneme:
\begin{align}
p(r)= r^2 + r + (1-r) = r^2 + 1 = n.
\end{align}
% % p(x)=0
Nyní zbývá vyřešit případ, kdy $p(\lambda)=0$.
Řešení této kvadratické rovnice je 
\begin{align}
\lambda_{1,2} = \frac{-1\pm\sqrt{4r-3}}{2}.
\end{align}
Násobnosti těchto kořenů označíme $m_1, m_2$. Jejich součet je zřejmě roven $n-1 = r^2$.
% % stopa
Využitím faktu, že stopa matice je suma vlastních čísel 
včetně násobností, získáme rovnici
\begin{align}
	0 = \Tr(A) = r+m_1\lambda_1 + m_2\lambda_2.
\end{align}
Její snadnou úpravou již získáme hledanou podmínku pro $r$
\begin{align}
2r - r^2 + \sqrt{4r-3}(m_1 - m_2) = 0 \label{4-2:rovnice-pro-r}
\end{align}

% % rozbor odmocniny
Řešení rozdělíme na dva případy.
\begin{enumerate}
% % % \not\in \Q
\item $\sqrt{4r-3} \not\in \Q$: potom $m_1 = m_2$ a tedy $r = 2$.
% % %  \in \Q
\item $\sqrt{4r-3} = s \in \Q$, což implikuje
\footnote{Odmocnina z přirozeného čísla je vždy přirozené číslo, či iracionální číslo, nikdy zlomek.} 
 $s \in \N$.
  Substitucí $4r - 3 = s^2$ do rovnice \ref{4-2:rovnice-pro-r} získáme
  \begin{align}
  -s^4 + 2s^2 + 16(m_1 - m_2)s +15 =0.
  \end{align}
  Tudíž $s|15$. Pro jednotlivé hodnoty	$s \in \set{1, 3, 5, 15}$, dostáváme $r \in \set{1, 3, 7, 57}$.
\end{enumerate}
\qed

\subsection{Silně regulární grafy}
% Stanislava Tlustá

% úvod
Dalším typem regulárních grafů jsou silně regulární grafy.

\df $r$-regulární graf $G$ se nazývá silně regulární, pokud existují $e, f \in \N$ taková, že:
\begin{itemize}
	\item každá hrana $uv \in E(G)$ se vyskytuje právě v $e$ trojúhelnících (tj. $|N(u) \cap N(v)| = e$) a zároveň
	\item každá nehrana $uv \not\in E(G)$ se vyskytuje právě v $f$ třešničkách (tj. $|N(u) \cap N(v)| = f$).
\end{itemize}

\pzn Abychom mohli zanedbat triviální případy, dodáváme $f>0$ a $G\neq K_n$.
Příkladem silně regulárního grafu je úplný bipartitní graf se stejně velkými
partitami ($e=0$, $f$ velikost partity). Nejmenším nebipartitním silně regulárním grafem je
pětiúhelník ($e=0$, $f=1$).

\vt Nechť $G$ je silně regulární graf s parametry $(r,e,f)$ na $n$ vrcholech. Potom:
\begin{enumerate}
	\item[(a)] $f-e = 1$, $n = 2r +1$,  $r = 2f$ nebo
%	\item[] {\it nebo}
	\item[(b)] $\exists s \in \Z$ takové, že platí $(e-f)^2-4(f-r) = s^2$ \\
	a výraz \ ${r\over 2fs}((r-1+f-e)(s+f-e)-2f)$ je přirozené číslo.
\end{enumerate}

\dk
% jak na to
Technika tohoto důkazu je stejná jako u předchozí věty o Mooreových grafech.

% sestavení polynomu
Matici souslednosti grafu $G$ označíme $A$. Její druhá mocnina má na diagonále $r$. Mimo diagonálu má buď hodnotu $e$ pro případ kdy v $A$ byla jednička ($e$ trojúhelníků), nebo hodnotu $f$, pokud v $A$ byla nula ($f$ třešniček). Vidíme tedy vztah:
\begin{align}
A^2 = rI + eA + f(J-I-A),
\end{align}
kde $E$ je jednotková matice a $J$ je matice samých jedniček. Úpravou dostaneme polynomiální vztah
\begin{align}
p(A) = A^2 + (f-e)A + (f-r)E = fJ.
\end{align}

% spektra
Dále platí, že pro $\lambda \in \Sp(A)$ je $p(\lambda) \in Sp(fJ)$. Spektrum $fJ$ známe: obsahuje $fn$ s násobností $1$ a $0$ s násobností $n-1$.

% % \lambda = r
Protože $G$ je $r$-regulární, tak je $r$ také jeho vlastním číslem. Dosadíme tedy $r$ do $p$:
\begin{align}
p(r) &= r^2 + (f-e)r + (f-r) \\
p(r) &= r^2 +fr -er +f -r +1 -1 \\
p(r) &= (r^2 -er +1) + f(r+1) -(r+1) \\
p(r) &= (r^2 -er +1) + (r+1)(f-1).
\end{align}
Protože $f>0$, tak platí $(r+1)(f-1) \geq 0$.
Navíc zřejmě $e<r$, tudíž $r^2 -er +1 >0$. Jediné vlastní číslo, které toto splňuje je $fn$, proto
\begin{align}
\label{4-3:vztah-pro-vrcholy}
p(r) = fn
\end{align}

%%%%%%%%
% % p(x)=0
Nyní zbývá vyřešit případ, kdy $p(\lambda)=0$.
Řešení této kvadratické rovnice je 
\begin{align}
\lambda_{1,2} = {e-f \pm s \over 2}, \qquad s = \sqrt{(f-e)^2 -4(f-r)}.
\end{align}
Násobnosti těchto kořenů označíme $m_1, m_2$. Jejich součet je zřejmě roven $n-1$.
% % stopa
Využitím faktu, že stopa matice je suma vlastních čísel 
včetně násobností, získáme rovnici
\begin{align}
	0 = \Tr(A) = r + m_1\lambda_1 + m_2\lambda_2,
\end{align}
kterou upravíme
\begin{align}
0 &= r + {m_1 \over 2}(e -f +s) + {m_2 \over 2}(e -f -s) \\
0 &= 2r + (e-f)(m_1 +m_2) +s(m_1 -m_2). \label{4-3:rovnice-pro-s}
\end{align}

% % rozbor odmocniny
Řešení rozdělíme na dva případy.
\begin{enumerate}
% % % \not\in \Q
\item $s \not\in \Q$: potom $m_1 = m_2$. Potom se rovnice zjednoduší
\begin{align}
0 &= 2r +2m_{1}(e-f) \\
m_1 &= {r \over f-e}.
\end{align}
Z toho vidíme, že
\begin{align}
\label{4-3:rovnice-ef}
	(f-e)|r, \quad f-e > 0, \quad n = 1 + 2m_1 = 1 + {2r\over f-e}.
\end{align}
Pokud $f-e = 1$, tak jsme hotovi. \\
Pokud $f-e = 2$, pak $n = 1+r$ a $G = K_{r+1}$, ale úplné grafy jsme si zakázali. \\
Pokud $f-e > 2$, pak $n < 1+r$, což je nesmysl.

Po dosazení $f-e=1$ do poslední rovnice \ref{4-3:rovnice-ef} vidíme, že $n = 2r+1$. \\
Po dosazení téhož do polynomu $p$ a použitím vztahu \ref{4-3:vztah-pro-vrcholy} dostáváme
\begin{align}
r^2 + r + (f-r) = f(2r+1).
\end{align}
Z toho již snadnou úpravou získáme hledané rovnosti
\begin{align}
r = 2f, \qquad n = 4f+1.
\end{align}

%%%%%%
% % %  \in \Q
\item $s \in \Q$, což implikuje $s \in \N$.
Vezmeme vztah pro násobnosti vlastních čísel a vztah \ref{4-3:vztah-pro-vrcholy} pro $n$
\begin{align}
m_2 = n - 1 - m_1 = {r^2 -er +1 +(r+1)(f-1) \over f} -1 -m_1.
\end{align}
Dosadíme do rovnice \ref{4-3:rovnice-pro-s}. Dostaneme:
\begin{align}
0 = 2r +m_1(e-f+s) + ({r^2 -er +1 +(r+1)(f-1) \over f} -1 -m_1)(e-f-s) 
\end{align}
Což dále upravíme
\begin{align}
& m_1(-(e-f+s)+(e-f-s)) = 2r + {1 \over f}((r^2 -er +rf -r +f)-f)(e-f-s), \\
& m_1(-2s) = 2r +{1 \over f}(r^2 -er +fr -r)(e-f-s), \\
& m_1 = {1 \over 2sf}(-2rf +r(r-e+f-1)(-e+f+s)), \\
& m_1 = {r \over 2sf}(r-1+f-e)(s+f-e) -2f).
\end{align}
Protože $m_1$ je násobnost vlastního čísla $\lambda_1$, tak se jedná o přirozené číslo.
\end{enumerate}
\qed

\vt(Friendship theorem) Nechť každí dva lidé mají právě jednoho společného známého. Pak existuje jeden (starosta), který zná všechny.

Neboli: nechť pro každé dva různé vrcholy $u, v \in V(G)$ platí $|N(u) \cap N(v)| = 1$. Potom existuje vrchol $c \in V$ takový, že $N(c) \cup \{c\} = V$.

\pzn Friendship theorem tvrdí, že takový graf musí vypadat jako
mlýn (hromádka trojúhelníků, které se stýkají v jednom centrálním vrcholu), viz obrázek \ref{4-3:obr-mlyn}.
\begin{align}
\label{4-3:obr-mlyn}
\begin{tikzpicture}[thick,scale=1.1]
% obrázek: šestilopatkový mlýn
\draw \foreach \x in {15,75,...,315}
    {
        (0,0) -- (\x:2) \vrchol -- (\x+30:2) \vrchol -- cycle
    };
\draw (0,0) \vrchol;
% popisek
\draw (5,0) node {šestilopatkový mlýn};
\end{tikzpicture}
\end{align}

\dk
% převzato z:
% http://math.mit.edu/~fox/MAT307-lecture20.pdf
% jak na to - sporem
Pro spor předpokládejme, že takový vrchol $c$ neexistuje. Nejprve si všimněme, že podmínka na množství sousedů implikuje, že $G$ neobsahuje čtyřúhelník. 

% G regulární
% % u,v nesousedí
Nejdříve ukážeme, že $G$ je regulárním grafem. Vezměme libovolné dva vrcholy $u$, $v$, které spolu nesousedí a označme $w_1, ..., w_k$ sousedy vrcholu $u$. Každý vrchol $w_i$ musí mít jednoho společného souseda $z_i$ s vrcholem $v$. Vrcholy $z_i$ musí být různé, jinak by vznikl čtyřúhelník ($u, w_i,z_i =z_j, w_j$). Vrchol $v$ má tedy také alespoň $k$ sousedů. Symetrickou úvahou pak dostáváme rovnost $deg(u) = deg(v)$. 

% % ostatní vrcholy
Vrcholy $u$ a $v$ mají právě jednoho společného souseda $c$.
Bez újmy na obecnosti můžeme předpokládat, že je to $c=w_1$. Jakýkoliv jiný vrchol $w \in V(G)$ již sousedí nanejvýše s jedním z vrcholů $u$ a $v$ (jinak by vznikl čtyřúhelník). Zopakováním předchozí úvahy pro vrchol se kterým $w$ nesousedí vidíme $deg(w) = deg(u) = deg(v)$. Nakonec $w_1$ nemůže podle předpokladu být spojen se všemi vrcholy, proto i pro něj platí $deg(w_1) = deg(u) = deg(v)$.

Všechny vrcholy tedy mají stejný stupeň a graf $G$ je $k$-regulární. Dokonce je silně regulární ($e=f=1$).

% délky cest
Nyní se podíváme na sledy délky $2$. Od každého vrcholu $x \in V$ jich vede $k^2$, protože $G$ je $k$-regulární. Navíc z vrcholu $x$ vede do každého jiného vrcholu $y \in V$ právě jedna cesta délky $2$ ($e=f=1$). Sledů z $x$ do $x$ je přesně $k$. Dostáváme tedy vztah, ze kterého vyjádříme počet vrcholů:
\begin{align}
k^2 &= (n-1) + k \\
n &= k^2 -k +1. \label{4-3:friendship-vrcholy}
\end{align}

% vlastní čísla
V dalším kroku zopakuje již známý postup pro hledání polynomiálního vztahu vlastních čísel. Matici souslednosti grafu $G$ označíme $A$.
Z rozboru sledů délky $2$ provedeném v předchozím kroku vidíme, že matice $A^2$ má na diagonále $k$ a všude mimo diagonálu jedničky. Dostáváme tedy vztah
\begin{align}
A^2 = J + (k-1)I.
\end{align}
Vlastními čísly matice $A^2$ jsou tedy $n+(k-1) = k^2$ s násobností $1$ a $k-1$ s násobností $n-1$. Vlastní čísla matice $A^2$ jsou druhými mocninami vlastních čísel matice $A$.  
Tudíž matice $A$ má vlastní čísla $k$ s násobností $1$ a $\pm \sqrt{k-1}$ s násobnostmi $m_1$, $m_2$.
Použitím vztahu pro stopu matice dostáváme:
\begin{align}
0 = \Tr(A) = k + (m_1 -m_2)\sqrt{k-1}.
\end{align}
To upravíme do tvaru
\begin{align}
k^2 = (m_2 -m_1)^2(k-1),
\end{align}
z něhož plyne, že $k-1|k^2$. To je však možné pouze pro $k = 1,2$. Jinak totiž $k-1$ dělí $k^2-1$, nemůže tedy dělit zároveň $k$. 

% závěr
Hodnotě $k=1$ odpovídá po dosazení do rovnice \ref{4-3:friendship-vrcholy} graf $K_1$.
Hodnotě $k=2$ odpovídá graf $K_3$. 
Oba splňují jak předpoklady, tak závěr věty. 
Pro jakýkoliv jiný graf nastává spor, tudíž musel vrchol $c$ sousedit se všemi ostatními.
\qed

\subsection{Rayleighův princip a proplétání}


\vt (Rayleighův princip) Nechť $A$ je matice $n\times n$ s ortonormální bazí z vlastních 
vektorů $x_i$ a vlastními čísly $\lambda_1 \geq \lambda_2 \geq\dots\geq\lambda_n$. Potom:
\begin{enumerate}
\item $x \in\sk{x_1,\dots,x_k} \Rightarrow x^*Ax\ge \lambda_kx^*x$
\item $x \in\sk{x_k,\dots,x_n} \Rightarrow x^*Ax\le \lambda_kx^*x$
\end{enumerate}

\dk $x \in\sk{x_1,\dots,x_k} \Rightarrow x = \sum_{i=1}^k \alpha_ix_i$
\begin{align*}
	x^*Ax &= x^*(Ax) = x^*\left(A\cdot\sum_{i=1}^k\alpha_ix_i\right) = x^*\left(\sum_{i=1}^k\alpha_iAx_i\right) = x^*\left(\sum_{i=1}^k\alpha_i\lambda_ix_i\right) = \\
	&= \sum_{i=1}^k\alpha_i\lambda_ix^*x_i = \sum_{i=1}^k\alpha_i\lambda_i\left(\sum_{j=1}^k \alpha_jx_j\right)^*x_i = \sum_{i=1}^k \alpha_i\lambda_i(\alpha_ix_i)^*x_i = \\
	&= \sum_{i=1}^k \lambda_i\underbrace{\alpha_i\overline{\alpha_i}}_{\ge 0} \ge \sum_{i=1}^k \lambda_k\alpha_i\overline{\alpha_i} = \lambda_k\sum_{i=1}^k \alpha_i\overline{\alpha_i} = \lambda_kx^*x
\end{align*}

Poslední rovnost plyne z následujícího:
\begin{align*}
	 \lambda_kx^*x = \left(\sum_{i=1}^k \alpha_ix_i\right)^*\left(\sum_{i=1}^k \alpha_ix_i\right) = \sum_{i=1}^k \alpha_i\overline{\alpha_i}
\end{align*}

Druhou nerovnost dokážeme analogicky. \qed


\vt (Věta o proplétání) Nechť $A$ a $B$ jsou matice takové, že $B$ vznikla z $A$ 
vymazáním nějakého řádku a sloupce. Potom pro vlastní čísla $\lambda_i,\mu_i$ 
matic $A,B$ platí:
\begin{align}
	\lambda_1 \geq \mu_1 \geq \lambda_2 \geq \dots\geq \mu_{n-1} \geq \lambda_n
\end{align}

\dk Dokazujeme $\lambda_k \geq \mu_k \geq \lambda_{k+1}$. Označme $x_i$ 
a $y_i$ vlastní vektory matic $A$ a $B$.  Zaveďme následující vektorové 
podprostory $\Compl^n$ (ačkoli druhý z nich nemá dostatek složek, můžeme mu jednu 
nulovou přidat a nic se nestane):
\begin{align}
S_1 := \calL\{x_k, \dots, x_n\} \subseteq \Compl^n \\
S_2 := \calL\{y_1, \dots, y_k\} \subseteq \Compl^n
\end{align}
Zřejmě $\dim(S_1) + \dim(S_2) = (n-k+1) + k > n$, tedy $\exists x \in S_1\cap S_2$. Použijeme 
Reileighův princip pro oba prostory a máme:
\begin{align}
	\mu_k \leq \frac{y^*By}{y^*y} = \frac{x^*Ax}{x^*x} \leq \lambda_k
\end{align}
Stačí ukázat, že $\mu_k \geq \lambda_{k+1}$ -- to je ale snadné, stačí vzít $-A$ 
a $-B$, čímž se obrátí znaménka vlastních čísel a nerovnosti. \qed

\vt (Věta o proplétání při násobení maticí) Nechť $A$ je symetrická čtvercová matice 
s vlastními čísly a vektory $\lambda_i$ a $x_i$, $S$ reálná matice, že $S^TS=I$.  
Definujeme $B := S^TAS$ a označíme vlastní čísla a vektory matice $B$ jako 
$\mu_i$ a $y_i$. Potom $\mu_i$ proplétají $\lambda_i$ a pokud navíc $\mu_i = 
\lambda_i$ pro nějaké $i$, tak $Sy_i$ je vlastní vektor $A$ příslušící vlastnímu 
číslu $\lambda_i$.

\dk Použijeme Rayleighův princip podobně jako v předchozím tvrzení. Všimneme 
si, že:
\begin{align}
	x \in \calL\{ S^Tx_k, \dots, S^Tx_{n}\}^\perp \Leftrightarrow
	Sx \in \calL\{ x_k, \dots, x_{n}\}^\perp
\end{align}
Stačí si opět vzít vhodný prvek $x$ z průniku:
\begin{align}
	x \in \calL\{ S^Tx_k, \dots, S^Tx_{n}\}^\perp \cap \calL\{y_1, \dots, y_k\}
\end{align}
A můžeme použít Reileighův princip:
\begin{align}
	\lambda_k \geq \frac{(Sx)^TASx}{(Sx)^TSx} = \frac{x^TBx}{x^Tx} \geq \mu_k \\
\end{align}
Navíc platí, že pokud $\lambda_i = \mu_i$, potom:
\begin{align}
	\frac{x^TBx}{x^Tx} = \lambda_i \quad\Rightarrow\quad x^TBx=x^Tx\lambda_i 
	\quad\Rightarrow\quad Bx = \lambda_i x
\end{align}
A $x$ je vlastní vektor příslušící $\lambda_i$, jak jsme chtěli dokázat.\qed

\df $A$ je bloková matice s bloky velikosti $x_1, \dots, x_m$. Kvocient $A$ je matice $B^{m\times m}$, kde $b_{i,j} = $ průměr hodnot $A_{i,j}$.
\begin{align*}
	A = \left(\begin{matrix}
		A_{1,1} & A_{1,2} & \dots \\
		A_{2,1} & A_{2,2} & \dots \\
		\vdots & \vdots & \ddots 
		\end{matrix}\right)
	\qquad
	\qquad
	B = \left(\begin{matrix}
		b_{1,1} & b_{1,2} & \dots \\
		b_{2,1} & b_{2,2} & \dots \\
		\vdots & \vdots & \ddots 
		\end{matrix}\right)
\end{align*}

\vt (Věta o proplétání kvocientu) Pokud $B$ je kvocient $A$, pak vlastní čísla
$B$ proplétají vlastní čísla $A$.

\dk Mějme $\widetilde S$ matici incidence blokové $A$:
\begin{align*}
	\widetilde S = \left(\begin{array}{llll}
		\framebox[3em][c]{1} & & & \hfil\bigzero \\
		& \framebox[4em][c]{1} & & \\
		\bigzero & & \framebox[2em][c]{1} & \\
		& & & \framebox[1em][c]{1}
		\end{array}\right)
\end{align*}

\begin{align*}
	& \widetilde S\cdot\widetilde S^T = \text{diagonální matice } (x_1, x_2, \dots, x_m) = D \\
	& S := \widetilde S \cdot D^{-{1\over2}} \\
	& \widetilde B = S^TAS
\end{align*}

Kromě toho platí:
\begin{align*}
	& S^TS = I
	& B = D^{-{1\over2}}\widetilde BD^{-{1\over2}}
\end{align*}

Tedy $B$ je matice podobná $\widetilde B$ a má stejná vlastní čísla. Matice
$\widetilde B$ proplétá matici $A$, což plyne z věty o proplétání při násobení
maticí. \qed



\section{Náhodné procházky}

\subsection{Markovovské řetězce}


\df Markovovský řetězec je orientovaný graf s váženými hranami takový, že
výstupní stupeň každého vrcholu je 1. Markovoský řetězec často reprezentujeme
maticí přechodu $P$, kde $P_{ij}$ udává pravděpodobnost, že ze stavu $i$
přejdeme do stavu $j$.

\df Distribuce $\pi$ je vektor, jehož součet je 1 a kde $p_i$ určuje
pravděpodobnost, že se nacházíme ve stavu $i$.

\pzn Máme-li distribuci $\pi$ a provedeme jeden krok na Markovovském řetězci s
maticí přechodu $P$, dostaneme novou distribuci $\pi\cdot P$.

\df Markovovský řetězec je reversibilní, existuje-li distribuce $\pi$
t. že $\pi_i\cdot P_{ij} = \pi_j\cdot P_{ji}$.

\lm Markovovský řetězec je reversibilní $\Leftrightarrow$ je odvozen z váženého neorientovaného grafu.

\dk 
\begin{itemize}

\item[\uv{$\Leftarrow$}]
Zvolíme si $\pi$ následovně a ukážeme, že splňuje reversibilní podmínku:
\begin{align*}
& \pi_v = {\deg v\over \sum_{u\in V(G)} \deg u} & P_{ij} = {w_G(i,j)\over \deg i} \\ 
\end{align*}
\begin{align*}
\pi_i P_{ij} &= \pi_i {w_G(i,j)\over \deg i} = {w_G(i,j)\over \sum_{u\in V(G)}\deg u} \\
\pi_j P_{ji} &= \pi_j {w_G(j,i)\over \deg j} = {w_G(j,i)\over \sum_{u\in V(G)}\deg u}
\end{align*}

\item[\uv{$\Rightarrow$}]
Zvolíme váhu $w(i,j) = P_{i,j}\pi_i = P_{j,i}\pi_j = w(j,i)$ a dostaneme vážený
neorientovaný graf.
\qed
\end{itemize}

\df $\pi$ je stabilní distribuce\footnote{Někdy též zvaná \uv{stacionární}.},
je-li $\pi\cdot P = \pi$. Jinak řečeno, stabilní distribuce se po provedení
kroku nezmění.

\vt Pro $G$ neorientovaný souvislý platí: $\forall \rho$ počáteční distribuci
$\set{ \rho P_G^k }_k$ konverguje $\Leftrightarrow$ $G$ není bipartitní.

\dk \begin{description}
	\item \uv{$\Rightarrow$} Pokud je $G$ bipartitní, stačí jako protipříklad vzít distribuci, která začíná jenom v jedné partitě. Pak každým pronásobením matice se celá distribuce přesune do druhé partity, protože nemá kam jinam. Zjevně tedy nekonverguje k jedinému rozložení.
	\item \uv{$\Leftarrow$} Prvně si vyjádříme distribuci jako lineární kombinaci vlastních vektorů matice $P_G$ (to lze, protože tvoří ortonormální bázi). Tedy $\rho = \sum_i a_ip_i$. Dále si vyjádříme distribuci po $k$ iteracích:
  \todo{distribucí násobíme zleva, dále (levým) vlastním vektorem 1 není vektor jedniček,
        ale stabilní distribuce $\pi$ \dots}
	\begin{align}
		P_G^k\rho = P_G^k\sum_ia_ip_i = \sum_iP_G^ka_ip_i
	\end{align}
	Protože $p_i$ je vlastní vektor $P_G$, tak $P_Gp_i = \lambda_ip_i$:
	\begin{align}
		\sum_i\lambda_i^ka_ip_i
	\end{align}
  \todo{$P$ není matice sousednosti\dots}
	Nyní si všimneme, že protože graf není bipartitní, tak $\lambda_1 \neq -\lambda_n$ a největší vlastní číslo distribuce je $1$, protože matice $P_G$ má řádkové i sloupcové součty konstantní $1$ a zároveň je $1$ má vlastní vektor samých jedniček. Tedy pro $i > 1$ platí $|\lambda_i| < 1$. Dejme nyní výraz do limity a všimneme si, že suma jde k nule díky tomu, že jediný člen závislý na $k$ je $\lambda_i$:
	\begin{align}
		\lim_{k\to \infty}\left(\lambda_1^ka_1p_1 + \sum_{i>1}\lambda_i^ka_ip_i\right) = a_1p_1 = \pi
	\end{align}
	Tedy máme stabilní distribuci, protože $a_1p_1$ jsou po celou dobu konstantní.
\end{description}

\vt Nechť $\rho$ je distribuce na vrcholech grafu a $\mu=\max\{\lambda_i,-\lambda_n\}$. Pak po $t$ krocích platí, že $\|P_G^t\rho - \pi\|_1 \leq \mu^t\sqrt{n}$, tedy distribuce konverguje relativně rychle.

\dk Z předchozího důkazu víme, že $\rho = p_ia_i + \sum_{i>1} \lambda_i^ta_ip_i$ a \todo{vec}.

Pusťme se do odhadu naší odchylky, prozatím však v $L_2$ normě.

\begin{align}
	\|P_G^t\rho - \pi\|_2^2 = \left\|\sum_{i > 1} \lambda_i^ta_ip_i\right\|_2^2 = \sum_{i>1}\lambda_i^{2t}\|a_ip_i\|_2^2
\end{align}
Nyní si zjednodušíme práci a do sumy zahrneme i první člen. Navíc odhadneme $\lambda_i$ největšším vlastním číslem $\mu$ (mocnina u $\lambda_i$ je sudá!).
\begin{align}
	\leq \mu^{2t}\sum_i\|a_ip_i\|_2^2 = \mu^{2t}\|\rho\|_2^2 \leq \mu^{2t}
\end{align}
Nyní stačí výraz odmocnit a vzpomenout si na analýzu, čímž víme, že $\|x\|_1 \leq \|x\|_2 \cdot \sqrt n$ a máme nerovnost:
\begin{align}
	\|P_G^t\rho - \pi\|_2 &\leq \mu^{t} \\
	\|P_G^t\rho - \pi\|_1 &\leq \mu^{t} \sqrt n
\end{align}
Což jsme chtěli dokázat. \qed


\subsection{Stabilní distribuce a konvergence}





\section{Expandéry}

\subsection{Expanze}


\df
\begin{itemize}
	\item $E(S,T) = \{$ hrany mezi $S$ a $T$ $\}$
	\item $e(S,T) = |E(S,T)|$
	\item $e(S) = $ počet hran uvnitř $S$
	\item vrcholová expanze $h_v(G) = \min\limits_{S\subseteq V, |S|\le {n\over 2}} {|N(S) \over |S|}$
	\item hranová expanze $h(G) = \min\limits_{S\subseteq V, |S|\le {n\over 2}} {e(S,\bar S) \over |S|}$
\end{itemize}

\poz $h_v(G) \le h(G) \le d . h_v(G)$

\df 
\begin{itemize}
	\item Rodina expanderů $\{G_i\}_\infty$\quad$2^i \ge |G_i| \ge i: h(G_i) \ge \varepsilon$, $G_i$ je $d$-regulární.
	\item Spectral gap $= d - \max\{\lambda_2,-\lambda_n\}$
	\item Spektrální expanze $= d - \lambda_2$
	\item $\lambda = \max\{\lambda_2,-\lambda_n\}$
\end{itemize}

\vt ${1\over 2}(d-\lambda_2) \le h(G) \le \sqrt{d(d-\lambda_2)}$ (G je $d$-regulární graf).

\dk (Jen první nerovnost, druhá je bez důkazu). Sporem: nechť $S$ je množina
vrcholů s malou hranovou expanzí.

Pro $x \bot (1,1,\dots,1)$ platí $\lambda_2 \ge {x^TAx\over x^Tx}$ (Raileighův
princip). Zvolíme $x = (n-s)1_S - s1_{\bar S}$, kde $s = |S|$ a $1_S$ je
charakteristický vektor množiny $S$.

$$x^Tx = (n-s)^2s + s^2(n-s) = s(n-s)n$$
$$x^TAx = \sum_{(a,b)\in E} 2x_ax_b = 2(n-s)^2e(S)-2s(n-s)e(S,\bar S) + 2s^2e(\bar S)$$

Platí $ds = 2e(S) + e(S,\bar S)$, neboť $ds$ odpovídá počtu konců hran v $S$. Analogicky $d(n-s) = 2e(\bar S) + e(S,\bar S)$ pro $\bar S$. Z toho si vyjádříme $e(S)$ a $e(\bar S)$ a dosadíme do rovnice výše:

$$x^TAx = -e(S,\bar S)n^2 + (n-s)ds(n-s+s) = (n-s)dsn - e(S,\bar S)n^2$$
$$\lambda_2 \ge {(n-s)dsn - e(S,\bar S)n^2 \over s(n-s)n} = d - {n\over n-s}\cdot{e(S,\bar S)\over s}$$
$$d-\lambda_2 \le {n\over n-s} \cdot {e(S,\bar S)\over s} \le 2\cdot{e(S,\bar S)\over s} = 2h(G)$$
\qed

\lm Pro náhodný d-regulární graf skoro jistě platí $\lambda \le 2\sqrt{d-1} + O(1)$. Bez důkazu.



\input{6-2-vlastnosti-expanderu.tex}
\subsection{Konstrukce expandérů}
%úvod


Ačkoliv například úplný graf je dobrý expandér ve smyslu, že má vysokou 
expanzi, bohužel s expanzí a velikostí roste neúnosně stupeň. Následující 
konstrukce vytváří grafy, které mají vyokou expanzi, ale konstantně malý 
stupeň.

\vt (Randomizovaná) Mějme $2n$ vrcholů. Pro $d$-regulární expandér zvolíme $d$ 
uniformě náhodných prefektních párování nad těmito vrcholy. Sjednocení těchto 
párování dává $d$-regulární graf, který je navíc dobrý expandér.

\vt (Prvočíselná) Nechť $p$ je prvočíslo a $V:=Z_p$. Definujeme $G_p=(V,E)$ s 
hranami $(x, x\pm 1)$ a $(x,x^{-1})$. Pak $G_p$ je rodina dobrých expandérů.

\vt (Margulis) Nechť $G_m=(V,E)$ a $V=\Z_m \times \Z_M$. Definujeme hrany pro 
$4$-regulární graf jako: $(x\pm y, y)$ a $(x,y\pm x)$. Pak $G_m$ je rodina 
dobrých expandérů.

%zig-zag součin
\df (Zig-Zag) Nechť $G$ je $g$-regulární graf a $H$ je $h$-regulární graf na $g$ 
vrcholech. Definujeme $G \zz H$ následujícím způsobem:
\begin{enumerate}
	\item Říkejme, že hrany z $G$ jsou {\bf\color{red}červené} a hrany z $H$ 
		jsou {\bf\color{blue}modré}.
	\item Vrcholy $G$ nahradíme grafy $H$ tak, že každému vrcholu $H$ přiřadíme 
	jednu hranu $G$. Protože $G$ je $g$-regulární a $H$ má právě $g$ vrcholů, 
	nyní každý vrchol má právě jednu hranu z $G$.
	\item Pro každou cestu tvaru 
		{\bf\color{blue}modrá}-{\bf\color{red}červená}-{\bf\color{blue}modrá} 
		vytvoříme {\bf\color{magenta}růžovou} hranu z prvního vrcholu do 
		posledního vrcholu.
	\item Odstraníme z grafu {\bf\color{blue}modré} a {\bf\color{red}červené} 
		hrany.
\end{enumerate}
Ukázka konstrukce je na obrázku \ref{zigzag-konstrukce}. Pro názornost není
graf $H$ regulární (nepodařilo se mi najít lepší příklad). Příklad je převzat z
webu.\footnote{\url{http://math.stackexchange.com/questions/454162/clearing-doubt-over-a-definition}}

\begin{figure}
\centering
\begin{subfigure}{7.5cm}{\includegraphics[width=\textwidth]{img/zigzag1.png}}\caption{Vložení 
$H$ do $G$}\end{subfigure}
\begin{subfigure}{7.5cm}{\includegraphics[width=\textwidth]{img/zigzag2.png}}\caption{Napojení 
hran $G$ na vrcholy $H$}\end{subfigure}
\begin{subfigure}{7.5cm}{\includegraphics[width=\textwidth]{img/zigzag3.png}}\caption{Tvorba 
růžových hran}\end{subfigure}
\begin{subfigure}{7.5cm}{\includegraphics[width=\textwidth]{img/zigzag4.png}}\caption{Výsledek}\end{subfigure}
\caption{Příklad Zig-Zag součinu. Pro jednoduchost není $H$ regulární, na funkci
to ale nic nemění.}
\label{zigzag-konstrukce}
\end{figure}

\poz Díky této konstrukci máme graf, který zachovává velikost $G$ (dokonce ji 
zvětšuje), ale dědí stupeň po grafu $H$ (pokud stupeň byl $h$, nyní bude $h^2$ 
-- to sice není úplně dobré, ale mohlo by to být horší).


 %Nechť $H$ je "dobrý" $d$-regulární expandér, $G_1 := H^2$. Potom 
%rodina grafů $G_{i+1} := G^2_i\zz H$ jsou "dobré" expandéry.
Pokud $H$ i $G$ jsou \uv{dobré} expandéry, potom také jejich zig-zag součin bude \uv{dobrý} expandér

\vt (Zig-Zag zachovává spectralní expanzi) 

Mějme $g$-regulární graf $G$ na $n$ vrcholech s v absolutní hodnotě druhým největším vlastním číslem $\lambda_G$ a $H$ bude $h$-regulární graf na $g$ vrcholech s v absolutní hodnotě druhým největším vlastním číslem $\lambda_H$, potom jejich zig-zag součin ($G\zz H$) bude $h^2$-regulární graf na $ng$ vrcholech s v absolutní hodnotě druhým největším vlastním číslem menším než $\lambda_G+\lambda_H+\lambda_H^2$.

%http://math.stackexchange.com/questions/454162/clearing-doubt-over-a-definition
%https://lucatrevisan.wordpress.com/2011/03/07/cs359g-lecture-17-the-zig-zag-product/#more-2203

%TODO Vzdálenostní množiny.



\section{Perfektní kódy}

Perfektní kódy jsou v jistém smyslu ty nejlepší samoopravné kódy, konkrétně mají vlastnost, že žádná slova z abecedy nezůstávají nevyužita. Cílem našeho snažení bude ukázat větu, která tyto kódy charakterizuje ve smyslu, při jakých parametrech může být kód perfektní. Začneme připomenutím základních pojmů, vyslovíme a dokážeme Lloydovu větu o nutné podmínce a z ní následně dokážeme (v současné podobně spíše nastíníme) kýženou charakterizaci.

\medskip
\subsection{Samoopravné kódy}



\df Samoopravný kód $C$ s parametry $(n, M)_q$ nad abecedou $A$ je podmnožina $A^n$, kde $|A|=q$ a $|C|=M$. Prvkům množiny $C$ říkáme kódová slova.

Nejčastěji $A$ je konečné těleso $GF(q)$ o $q$ prvcích nebo $A=\{0, \dots, q-1\}$. Pokud $C$ je vektorový podprostor nad tělesem $A$, pak $C$ nazýváme lineárním kódem. Množinu $A^n$ navíc vybavíme Hammingovou metrikou $d(~,~)$. Pro dvě slova $x, y \in A^n$ se složkami $x_i$ resp. $y_i$ platí $$d(x, y)=|\{i~|~x_i\neq y_i\}|.$$ Minimální vzdálenost kódu je pak definována jako $d=\min_{x \neq y \in C} d(x, y)$. Mluvíme pak o $(n, M, d)_q$ kódu. 

Chceme, aby kód měl co největší minimální vzdálenost (při co největší mohutnosti). To souvisí s tím, že pokud vysíláme kódové slovo $c \in C$, může během přenosu dojít k chybám (uvažujeme pouze změnu složky nikoliv zkrácení délky) a druhá strana přijme slovo $y= c + e$, kde $e$ je chybové slovo. Příjemce se pak snaží chybu detekovat a případně opravit $y$ na nejbližší kódové slovo $c' \in C$ (vše je měřeno Hammingovou metrikou). Pokud kódová slova budou co nejdále od sebe, je detekce a oprava $y$ na správné kódové slovo $c$ (tj. $c=c'$) více pravděpodobná. Přesněji pokud počet chyb (což je počet nenulových složek chybového slova $e$) je $\leq d-1$ je možné chybu detekovat (přijmeme-li nekódové slovo, víme, že nastala chyba). Pokud pokud počet chyb je $\leq \lfloor\frac{d-1}{2}\rfloor=:t$ je možné chybu opravit. Označme $N_t(c)=\{x\in A^n~|~d(c, x)\leq t\}$ okolí slova $c$ do vzdálenosti $t$. Vidíme, že okolí $N_t(c)$ pro všechna $c\in C$ jsou disjunktní, kde $C$ má minimální vzdálenost $d$ a $t$ je definováno výše. 





%\df Kód $C$ je $t$-perfektní, pokud opravuje $t$ chyb a navíc úplně pokrývá svou nosnou množinu $M$.

\tv(Hammingův odhad) Mějme $(n, M, d)_q$ kód $C$. Označme $t=\lfloor\frac{d-1}{2}\rfloor$. Pak
\begin{align*}
	|C| \leq { q^n \over \sum_{i=0}^t \binom{n}{i} (q-1)^i }
\end{align*}

\dk Stačí si uvědomit, že okolí jsou disjunktní a obsahují všechny stejně slov (zde nezáleží na středu okolí). Dostáváme tak: 
$$q^n\geq \sum_{c\in C} |N_t(c)|=M\cdot \sum^{t}_{i=0} \binom{n}{i}(q-1)^i.$$
\qed

\df Kódy, která nabývají rovnosti v Hammingově odhadu nazýváme perfektními.

Nyní si ukážeme základní příklady perfektních kódů. Mezi ty triviální patří totální $(n, q^n, 1)$ kód obsahující všechna slova z $A^n$, opakovací $(n, 2, n )$ kód pro lichou délku $n$ a jednoprvkový kód. 

Každý lineární kód můžeme popsat jeho bází. Generující matice $G$ o rozměrech $k \times n$ lineárního $(n, q^k)$ kódů $C$ nad $GF(q)$ má v řádcích zapsanou jeho bázi. Kontrolní matice lineárního kódu $C$ je taková matice $H$ o rozměrech $(n-k) \times n$, že $c\in C \Leftrightarrow Hc^T=0$. Platí $HG^T=0$. Lineární kód můžeme jednoznačně popsat jeho generující nebo kontrolní maticí.

\df Hammingův kód $\mathcal{H}(r, q)$ je určen svojí kontrolní matici o rozměrech $r \times \frac{q^r-1}{q-1}$, která obsahuje ve sloupcích všechny po dvou lineárně nezávislé vektory nad $GF(q)$ délky $r$. Kód $\mathcal{H}(r, q)$ je $1$-perfektní $(\frac{q^r-1}{q-1}, \frac{q^r-1}{q-1}-r, 3)_q$ lineární kód.

\df Uvažme matici $G'=(I_{12}~|~Q)$, kde $Q$ je doplněk matice sousednosti dvacetistěnu. Matice $G'$ generuje $(24, 12, 8)_2$ kód $\mathcal{G}_{24}$ nad $GF(2)$. Vynecháním libovolné fixní souřadnice kódových slov v $\mathcal{G}_{24}$ obdržíme $(23, 12, 7)_2$ kód $\mathcal{G}_{23}$. Kód $\mathcal{G}_{23}$ se nazývá binární Golayův $3$-perfektní kód.

\df Uvažme matici $G=(I_6~|~Q)$, kde 
\begin{displaymath}
Q= \left(
\begin{array}{ccccc}
1& 1 & 1 & 1 &  1\\
0& 1 & 2 & 2 &  1\\
1& 0 & 1 & 2 &  2\\
2& 1 & 0 & 1 &  2\\
2& 2 & 1 & 0 &  1\\
1& 2 & 2 & 1 &  0\\
\end{array}
\right).
\end{displaymath}

Matice $G$ generuje $(11, 6, 5)_3$ kód $\mathcal{G}_{11}$ nad $GF(3)$. Kód $\mathcal{G}_{11}$ se nazývá ternární Golayův $2$-perfektní kód. 

\subsection{Lloydova věta}
Nyní směřujeme k charakterizující větě, která říká, že ve skutečnosti žádné jiné perfektní kódy než výše uvedené nad abecedou mohutnosti mocniny prvočísla neexistují. Důkaz, který uvedeme je kombinatorický. Jádro důkazu spočívá v důkazu Lloydovy věty, která dává silné omezení na existenci perfektních kódů. 



\vt Definujme Lloydův polynom v proměnné $x$ stupně $t$

\begin{displaymath}
	L_t(x) = \sum_{j=0}^t(-1)^j(q-1)^{t-j}\binom{x-1}{j}\binom{n-x}{t-j}
\end{displaymath}
Pokud existuje $t$-perfektní kód délky $n$ nad abecedou mohutnosti $q$, pak $L_t(x)$  má $t$ různých celočíselných kořenů mezi $1$ a $n$.

K důkazu Lloydovy věty budeme potřebovat vlastnosti vzdálenostně regulárních grafů. 

\subsection{Vzdálenostně regulární grafy}


\df Uvažme graf $\Gamma=(V,E)$, že $V(G) = A^n$ a hrana mezi vrcholy $u,v$ vede právě tehdy, když $d(u, v) = 1$, tedy liší se právě v jedné souřadnici. Kód v grafu $\Gamma$ příslušející kódu $C$ je pak podmnožina vrcholů, které odpovídají kódovým slovům $C$.

Graf $\Gamma$ je speciálním případem vzdálenostně regulárního grafu. Poznatky z této sekce na závěr aplikujeme právě na $\Gamma$. Po celou dobu této podkapitoly pracujeme pouze s vzdálenostně regulárními grafy.

\df Graf $G$ je vzdálenostně regulární, pokud existují konstanty $s_{hij}$ tak, že  pro $\forall u,v\in V(G), d(u,v) = j$ je $$|\{w: d(u,w) = h, d(w,v) = i\}| = s_{hij}.$$

\poz $|i-j| > h \Rightarrow s_{hij} = 0$ (plyne z trojúhelníkové nerovnosti), $k = s_{110}$ je počet sousedů libovolného vrcholu v $k$-regulárním grafu. 

\lm Platí $$z_{mi} = z_{m-1,i-1} \cdot s_{1,i-1,i} + z_{m-1,i} \cdot s_{1,i,i} + z_{m-1,i+1} \cdot s_{1,i+1,i},$$ kde $z_{mi}$ značí počet sledů délky $m$ mezi vrcholy ve vzdálenosti $i$.

\dk $z_{00} = 1$, jinak $z_{0i} = 0$. Dále dokážeme indukcí pro $m \ge 1$ a $i
\ge 1$. $s_{1,i,j}$ je nenulové pouze pro $i \in \{j-1,j,j+1\}$ (z trojúhelníkové
nerovnosti). V rovnici sčítáme vrcholy sousedící s $u$, které jsou ve
vzdálenosti $i-1$, $i$ a $i+1$ od $v$.

\df Mějme matici sousednosti $A = A_G$. Označme $\A(G) = \{p(A): p(x) \in \Compl[x]\}$. $\A(G)$ je
vektorový prostor nad $\Compl$.



\df Definujme vzdálenostní matice $A_0=I, A_1=A, A_2, \dots, A_d$ grafu $G$. Sloupce a řádky jsou číslovány vrcholy grafu. \\
$$(A_i)_{uv} = \left\{\begin{matrix}
1\quad & d(u,v) = i  \\
0\quad & \text{jinak} \\
\end{matrix}\right.$$


\vt $\dim \A(G) = d+1$, kde $d$ je průměr $G$.\footnote{Průměr grafu je maximální nejkratší vzdálenost přes všechny dvojice vrcholů.} Bází $\A(G)$ jsou výše definované matice $A_0, A_1, A_2, \dots, A_d$. 

\dk Platí $A^m = \sum_{i=0}^d z_{mi}A_i$ pro libovolné $m\in \mathbb{N}$. Matice $A_0, A_1, A_2, \dots, A_d$ tedy generují celý prostor $\A(G)$ a zároveň jsou lineárně nezávislé a proto $\dim \A(G) = d+1$.
%$i > m \Rightarrow Z_{mi} = 0$ \\
%$A^0 = Z_{0,0} \cdot A_0 = A_0$\\
%$A^1 = Z_{1,0} \cdot A_0 + Z_{1,1} \cdot A_1 = A_1$\\
%$A^2 = Z_{2,0} \cdot A_0 + Z_{2,1} \cdot A_1 + Z_{2,2} \cdot A_2$\\
%$\vdots$\\
%$A^d = Z_{d,0} \cdot A_0 + Z_{d,1} + \dots + Z_{d,d}\cdot A_d$
 
\qed

%\poz $\widetilde \A = \{A_0, A_1, \dots, A_d\}$ tvoří bázi $\A(G)$.


\df Matice $B_h$ je velikosti $(d+1)\times (d+1)$ a definujme ji předpisem
$$
	(B_h)_{ij} := s_{hij}
$$
Navíc označme $B=B_1$.

\lm Existuje homomorfismus vektorových prostorů $\widehat{~~}: \A(G) \to \widehat{\A(G)}$ takový, že $\widehat{A_h} = B_h$ pro $h=0, \dots, d$.

\dk %Z předchozího lemmatu již máme bázi $\widetilde{\A}$ prostoru $\A$. Ukážeme si tedy, že můžeme přejít k bázi z menších matic $B$. 
Nejdříve si všimněme, co se děje v následujícím součinu matic:

$$
	(A_hA_i)_{uv} = \sum_w(A_h)_{uw} \cdot (A_i)_{wv} = s_{hid(u,v)}
$$

V sumě je přičtena $1$ pokaždé, když pro vrchol $w$ platí, že $d(u,w)=h$ a $d(w,v) = i$, což je přesně definice $s_{hij}$ pro $j = d(u,v)$. Máme tedy:

$$
	A_hA_i = \sum_{j=0}^d s_{hij} A_j
$$

Což je vlastně lineární kombinace prvků z báze s koeficienty $s_{hij}$.
Matici $B_h$ obsahuje v $i$-tém řádku souřadnice $A_hA_i$ vzhledem k bázi $\{A_0, A_1, \dots, A_d\}$, čili vektor $(s_{hi0}, \dots, s_{hid})$. Hledaný homomorfismus je tedy transpozice regulární reprezentace levého násobení v $\A(G)$ vzhledem k $\{A_0, A_1, \dots, A_d\}$.
\qed



\lm $B=B_1$ je tridiagonální matice. Všechny sloupcové součty jsou stejné a jsou rovny $k=s_{110}$. Navíc $s_{100}=0$ a $s_{101}=1$.
$$
B= \left(\begin{matrix}
& & & & & & & & \bigzero & \\
& & & & & & & & & \\
& \bigzero & & & & {\smash{\raisebox{.75\normalbaselineskip}{\diagdots{9em}{.5em}}}} & {\smash{\raisebox{1.2\normalbaselineskip}{\diagdots{6.5em}{.5em}}}} & \\
& & & & {\smash{\raisebox{1.3\normalbaselineskip}{\diagdots{6.5em}{.5em}}}} & & \\
\end{matrix}\right)$$

\dk Matice je tridiagonální, protože $s_{1,i,j}$ dává smysl jen pro $i \in \{j-1,j,j+1\}$ (z trojúhelníkové nerovnosti). Navíc v $j$-tém sloupci je $s_{1,j-1,j} + s_{1,j,j} + s_{1,j+1,j}$, což zahrnuje všechny sousedy $u$, kterých je $k$.
\qed

Následující známy výsledek z teorie matic uvádíme bez důkazu.

\pzn $B$ je tridiagonální matice $\Rightarrow$ $\forall$ její vlastní čísla jsou různá.

\subsection{Charakteristické polynomy}


\df Definujme polynomy $v_i \in \Q[\lambda]$ takové, že $\deg v_i(\lambda) = i$ tak, že
\begin{enumerate} 
	\item $v_0(\lambda) = 1$
	\item $v_1(\lambda) = \lambda$
	\item pro $i \in \{ 2, \dots, d-1\}$ induktivně, aby splňovaly rovnici 
	$$
		s_{1,i,i-1} v_{i-1}(\lambda) + s_{1,i,i}v_i(\lambda) + s_{1,i,i+1}v_{i+1}(\lambda) = \lambda v_i(\lambda)
	$$
\end{enumerate}


\subsection{Lloydova věta}


\vt Pokud existuje $t$-perfektní kód s parametry $(n,q)$, pak $L_t(x)$ (definice níže) má $t$ různých celočíselných kořenů mezi $0$ a $n$.

\begin{align}
	L_t(x) = \sum_{j=0}^t(-1)^j(q-1)^{t-j}\binom{x-1}{j}\binom{n-x}{t-j}
\end{align}

\dk Důkaz bude plynout touto sekcí a obsahuje spoustu pomocných lemmat a konceptů. Pro pochopení a reprodukci důkazu bude potřeba pochopit všechno mezi tímto místem a a sekcí označující samotný důkaz. Nechť práce započne.



\subsection{Vzdálenostně regulární grafy}


\df Vzdálenostně regulární graf je regulární a $\exists s_{hij}$ takové, že $\forall u,v\in V(G), d_G(u,v) = j:$ $|\{w: d_G(u,w) = h, d_G(w,v) = i\}| = s_{hij}$.

\poz $|h-j| > j \Rightarrow s_{hij} = 0$ (plyne z $\Delta$ nerovnosti), $k = s_{110}$ (počet sousedů vrcholu $u = v$ v $k$-regulárním grafu)

\lm $Z_{mi} = Z_{m-1,i-1} \cdot s_{1,i-1,i} + Z_{m-1,i} \cdot s_{1,i,i} + Z_{m-1,i+1} \cdot s_{1,i+1,i}.$ $Z_{mi}$ značí počet sledů délky $m$ mezi vrcholy ve vzdálenosti $i$.

\dk $Z_{00} = 1$, jinak $Z_{0i} = 0$. Dále dokážeme indukcí pro $m \ge 1$ a $i
\ge 1$. $s_{1,i,j}$ je nenulové pouze pro $i \in \{j-1,j,j+1\}$ (z $\Delta$
nerovnosti). V rovnici sčítáme vrcholy sousedící s $u$, které jsou ve
vzdálenosti $i-1$, $i$ a $i+1$ od $v$.

\df Matice sousednosti $A = A_G$. $\A(G) = \{p(A): p(x) \in \Compl[x]\}$. $\A(G)$ je
vektorový prostor.

\df Vzdálenostní matice $A_1, A_2, \dots, A_d$ grafu $G$: \\
\indent $(A_i)_{uv} = \left\{\begin{matrix}
1\quad & d_G(u,v) = i \hfill & \hspace{4cm} A_0 = I \\
0\quad & \text{jinak} \hfill & \hspace{4cm} A_1 = A \\
\end{matrix}\right.$



\subsection{Reprezentace vzdálenostně regulárních grafů polynomy}


\vt $\dim \A(G) = d+1$, kde $d$ je průměr $G$.\footnote{Průměr grafu je maximální nejkratší vzdálenost přes všechny dvojice vrcholů.}

\dk $A^m = \sum_{i=0}^d Z_{mi}A_i$ \\
$i > m \Rightarrow Z_{mi} = 0$ \\
$A^0 = Z_{0,0} \cdot A_0 = A_0$\\
$A^1 = Z_{1,0} \cdot A_0 + Z_{1,1} \cdot A_1 = A_1$\\
$A^2 = Z_{2,0} \cdot A_0 + Z_{2,1} \cdot A_1 + Z_{2,2} \cdot A_2$\\
$\vdots$\\
$A^d = Z_{d,0} \cdot A_0 + Z_{d,1} + \dots + Z_{d,d}\cdot A_d$

Generujeme celý vektorový prostor polynomů $A$ $\deg \le d$, tedy $\dim \A(G)
\le d+1$. Zároveň ale $A_0, A_1, \dots, A_d$ jsou lineárně nezávislé a proto
$\dim \A(G) = d+1$. 
\qed

\poz $\widetilde \A = \{A_0, A_1, \dots, A_d\}$ tvoří bázi $\A(G)$.

\df Matice $B_h$ pro graf je velikosti $d\times d$, uchovávající parametry $s_{hij}$:
\begin{align}
	(B_h)_{ij} := s_{hij}
\end{align}
Maticí $B$ navíc rozumíme matici $B_1$.

\lm Existuje funkce $f: \A \to \A$, že $f(A_h) = B_h$ a tuto operaci značíme $\widehat A = B$.

\dk Z předchozího lemmatu již máme bázi $\widetilde{\A}$ prostoru $\A$. Ukážeme si tedy, že můžeme přejít k bázi z menších matic $B$. Nejdříve si všiměme, co se děje v následujícím součinu matic:

\begin{align}
	(A_hA_i)_{uv} = \sum_w(A_h)_{uw} \cdot (A_i)_{wv} = s_{hid(u,v)}
\end{align}

Kde zmíněná suma je rozpis maticového násobení pro jednu buňku součinu. Zřejmě přičtu $1$ pokaždé, když pro vrchol $w$ platí, že $d(u,w)=h$ a $d(w,v) = i$, což je přesně definice $s_{hij}$ pro $j = d(u,v)$. Jak takový prvek ještě můžeme vyjádřit (rozepsáním maticového násobení s použitím předchozího vzorce pro buňku)?

\begin{align}
	A_hA_i = \sum_{j=0}^d s_{hij} A_j
\end{align}

Což je vlastně lineární kombinace prvků z báze s koeficienty $s_{hij}$.
Vytvořme tedy novou bázi, například takovou, která bude obsahovat právě tyto
koeficienty. Do řádku $i$ matice $B'_h$ zapíšeme souřadnice součinu $A_hA_i$
vůdči bázi $\widetilde{A}$, tedy $s_{hij}$. Tím získáme matice $B'_h$, které
jsou bazí (vytvořili jsme je zapsáním souřadnic lineárně nezávislých prvků a
tak jsou lineárně nezávislé), která navíc splňuje žádané vlastnosti a tedy
$B'_h = B_h$. \qed



\lm (O sousedech) $B1 = \left(\begin{matrix}
& & & & & & & & \bigzero & \\
& & & & & & & & & \\
& \bigzero & & & & {\smash{\raisebox{.75\normalbaselineskip}{\diagdots{9em}{.5em}}}} & {\smash{\raisebox{1.2\normalbaselineskip}{\diagdots{6.5em}{.5em}}}} & \\
& & & & {\smash{\raisebox{1.3\normalbaselineskip}{\diagdots{6.5em}{.5em}}}} & & \\
\end{matrix}\right)$ je tridiagonální matice. Všechny sloupcové součty jsou stejné a jsou rovny $k$.

\dk Matice je tridiagonální, protože $s_{1,i,j}$ dává smysl jen pro $i \in \{j-1,j,j+1\}$ (z $\Delta$ nerovnosti). Navíc v $j$-tém sloupci je $s_{1,j-1,j} + s_{1,j,j} + s_{1,j+1,j}$, což zahrnuje všechny sousedy $u$, kterých je $k$.
\qed


\lm $B1 = \left(\begin{matrix}
& & & & & & & & \bigzero & \\
& & & & & & & & & \\
& \bigzero & & & & {\smash{\raisebox{.75\normalbaselineskip}{\diagdots{9em}{.5em}}}} & {\smash{\raisebox{1.2\normalbaselineskip}{\diagdots{6.5em}{.5em}}}} & \\
& & & & {\smash{\raisebox{1.3\normalbaselineskip}{\diagdots{6.5em}{.5em}}}} & & \\
\end{matrix}\right)$ je tridiagonální matice $\Rightarrow$ $\forall$ vlastní čísla jsou různá.



\subsection{Charakteristické polynomy}



\df Definujme polynomy $v_i \in \Q[\lambda]$ takové, že $\deg v_i(\lambda) = i$ tak, že
\begin{enumerate} 
	\item $v_0(\lambda) = 1$
	\item $v_1(\lambda) = \lambda$
	\item pro $i \in \{ 2, \dots, d-1\}$ induktivně, aby splňovaly rovnici 
	$$
		s_{1,i,i-1} v_{i-1}(\lambda) + s_{1,i,i}v_i(\lambda) + s_{1,i,i+1}v_{i+1}(\lambda) = \lambda v_i(\lambda)
	$$
\end{enumerate}

\lm (O charakteristickém polynomu) Nechť $\lambda_1,\ldots,\lambda_d \in \Sp(B_1)$ takové, že jsou různá od $k$ . Potom pro $i=1, \dots, d$ platí:
$$
v_o(\lambda_i) + \ldots + v_d(\lambda_i)=0
$$ neboli $v_o(\lambda) + \ldots + v_d(\lambda) = c \cdot (\lambda - \lambda_1) \cdot \ldots \cdot (\lambda - \lambda_d)$.

\dk Vytvořme vektor $\vec{v} = (v_1(\lambda), \ldots, v_d(\lambda))$ a uvažme systém rovnic $B\vec{v} = \lambda \vec{v}$. Ten umíme řešit po řádcích (známe první dva členy vektoru a celou matici obsahující potřebné koeficienty), známe tedy vlastní čísla (kořeny této rovnice) a jejich vlastní vektory (obsahují složky $v_i(\lambda)$.

Nejprve si ukážeme, že jedno z vlastních čísel je $k$ (všimněte si, že v předpokladech používáme $d$ vlastních čísel, ale dimenze matice $B$ je $d+1$). Vezměme si výše používaný systém rovnic a sečtěme levé a pravé strany. Podle Lemma o sousedech jsou sloupcové součty matice $B$ rovny $k$, získáme tedy rovnici $k(v_0(\lambda) + \ldots + v_d(\lambda)) = \lambda (v_0(\lambda) + \ldots + v_d(\lambda))$, z čehož po úpravě plyne, že $\lambda = k$.

\todo{Rovnost s char. polynomem}

\lm Pro polynomy $v_i$ platí, že $v_i(A) = A_i$ a $v_i(B) = B_i$.

\dk $$AA_i=\sum^{d}_{j=0}s_{1ij}A_j=s_{1,i,i-1} A_{i-1} + s_{1,i,i}A_i + s_{1,i,i+1}A_{i+1}$$
Tj. $v_i(A)=A_i$. Po aplikaci homomorfismu $\widehat{~~}$ dostáváme $v_i(B)=B_i$.


\df Zafixujme $z\in V(G)$. Definujme $T\in\{0,1\}^{(d+1) \times n}$ předpisem
$$ T_{i,u} = \left\{\begin{matrix}
1\quad & d(u,z) = i\hfill \\
0\quad & \text{jinak}\hfill
\end{matrix}\right.$$

\lm (O zastřešování) $X\in\A(G), z\in V(G) \Rightarrow TX = \widehat XT$

\dk 
\begin{align*}
(TA)_{iu} &= \sum_w T_{iw}A_{wu} = s_{i,1,d(u,z)} \\
(BT)_{iu} &= \sum_j B_{ij}T_{ju} = s_{1,i,d(u,z)}=s_{i,1,d(u,z)} \\
TA = BT \quad&\Rightarrow\quad TA^2 = BTA = B^2T \quad\Rightarrow\quad TA^m = B^mT \\
Tp(A) = p(B)T \quad&\Rightarrow\quad TX = \widehat XT
\end{align*}
\qed

\df Definujme si pomocné polynomy: 
\begin{align*}
	x_i(\lambda) &= v_0(\lambda) + \dots + v_i(\lambda) \\
  S_t &= x_t(A) = A_0 + A_1 + \dots + A_t
\end{align*}
Kde $S_t$ je matice, která označuje dvojce vrcholů jedničkou, pokud jsou vzdálené nanejvýš $t$ (je to součet vzdálenostních matic do $t$).

\lm Nechť $C$ je perfektní kód v grafu $G$. Ať $c$ je jeho charakteristický vektor $C$. Pak $S_t\cdot c = \vec 1$.

\dk $(S_t\cdot c)_u = |\{w: w\in C, d(w,u) \le t\}| = 1$, což plyne přímo z definice perfektního kódu.
\qed

\lm $G$ obsahuje $t$-perfektní kód $C$ $\Rightarrow$ $\dim \Ker \widehat S_t \ge t$

\dk Mějme $z_0 = z \in C$ a $z_1,z_2,\dots,z_t\in C$ takové, že $d(z,z_i) = i$ pro $i = 1, 2, \dots, t$.
Platí
$(T_{z_i} \cdot c)_j = \delta_{ij}$ (Kroneckerovo delta $= 1$ pro $i=j$, $0$ jinak). Tedy vektory $T_{z_i} \cdot c$ pro $i = 0, 1, \dots, t$ jsou lineárně nezávislé. Navíc dostáváme, že

\begin{align*}
	&\widehat S_t(T_{z_i}\cdot c) = (\widehat S_t \cdot T_{z_i}) \cdot c \overset{1}{=} T_{z_i} \cdot S_t \cdot c \overset{2}{=} T_{z_i}\cdot \vec 1 = \left(\begin{matrix}
		k_0 \\ \vdots \\ k_d
	\end{matrix}\right) \\
\end{align*}

$\overset{1}{=}$ plyne z lemma o zastřešování, $\overset{2}{=}$ plyne z předchozího lemmatu. Výsledný vektor je pro všechny volby $z_i$ stejný, protože jeho položky je počet sousedů s pevnými vzdálenostmi, a protože je to vzdálenostně regulární graf, jsou to nějaké hodnoty $s_{hij}$ se stejným $hij$ pro řádek. Pišme

\begin{align*}
	&u_i = T_{z_i}\cdot c - T_{z_0}\cdot c\qquad i = 1, 2, \dots, t \\
	&\widehat S_t u_i = \widehat S_t T_{z_i}\cdot c - \widehat S_t T_{z_i}\cdot c = \left(\begin{matrix}k_0 \\ \vdots \\ k_d\end{matrix}\right) - \left(\begin{matrix}k_0 \\ \vdots \\ k_d\end{matrix}\right) = \vec 0 \quad\Rightarrow\quad u_i\in\Ker \widehat S_t\\
\end{align*}

Vektory $u_1,\dots, u_t$ tvoří $\Ker \widehat S_t$ a jsou lineárně nezávislé. Tedy $\dim\Ker\widehat S_t \ge t$.
\qed

\subsection{Důkaz Lloydovy věty}


Zde začnou věci dávat větší smysl. Nejdříve dokážeme pomocí výše zmíněných lemat pomocné tvrzení, který dá podobný polynom, následně si s ním pohrajeme a získáme polynom Lloydův, tak jak byl zadefinován na začátku.

\vt (Lloydův prototyp) Pokud existuje $t$-perfektní kód v $G$, potom $x_t(\lambda)\backslash x_d(\lambda)$.

\dk Nejprve si všimněme, že $\widehat{S_t} = \widehat{X_t(A)} = \widehat{\sum_i^t A_i} = \sum_i^t B_i = X_t(B)$.
Dále se podívejme na spektra $B$ a $\widehat{S_t}$:
\begin{align}
	&\Sp(B) = \{ k, \lambda_1, \ldots, \lambda_d \} \\
	&\Sp(\widehat{S_t}) = \{ x_t(k), x_t(\lambda_1), \ldots, x_t(\lambda_d) \}
\end{align}

\todo{proc a zbytek...}



\subsection{Charakterizace perfektních kódů}

\vt Nechť $q=p^r$, a $p$ je prvočíslo. Pak existují právě následující netriviální perfektní kódy (tedy s $|C| \geq 2$ a pokud $|C| = 2$, tak to není kód $q=2$ a $n=2t+1$):
\begin{description}
	\item $1$-perfektní kód $n={q^k-1 \over q-1}$ pro libovolné $k$ a $q$ (Hammingův)
	\item $2$-perfektní kód $q=3$ a $n=11$ (Golayův)
	\item $3$-perfektní kód $q=2$ a $n=23$ (Golayův)
\end{description}
Dál $q$ složené neexistují perfektní kódy pro $t \geq 3$ a pro $t = 1,2$ se to neví.

Důkaz je technicky náročný a budeme se jím zabývat po zbytek sekce. Základem je Lloydova věta a Sphere packing ukázaným na začátku sekce. Nejprve se pro malé hodnoty parametrů ukáže, zda pro dané hodnoty kódy existují či nikoli. Pak se pro obecný případ udělá horní odhad parametrů pomocí Lloydovy věty. A pro konečný počet případů, pro které by kódy mohly existovat, bylo dokázáno počítačem, že jiné perfektní kódy než Hammingovy a Golayovy neexistují.

\vt Pro $q=3$ existuje jenom $2$-perfektní kód.

\vt Pro $q=2$ neexistuje $2$-perfektní kód.

\dk Ze Sphere packingu dostaneme:
\begin{align}
	1 + n + {n \choose 2} &= q^\alpha  \\
	2 + 2n + n(n-1) &= q^{\alpha + 1} \label{eq:SpherePacking}\\
	7 + (2n + 1)^2 &=q^{\alpha + 3} \label{eq:SpherePacking2}
\end{align}

A dále pak z Lloydovy věty dostaneme:
\begin{align*}
L_2(x) &= {n - x \choose 2} - (x - 1)(n -x) + {x - 1 \choose 2} \\
2L_2(x) &= n^2 + n + 2 + 4x^2 - 2(n + 1)2x
\end{align*}

Provedeme substituci $y = 2x$ a za $n^2 + n + 2$ dosadíme $q^{\alpha + 1}$ (z rovnice~\ref{eq:SpherePacking}):
\begin{align*}
p(y) = y^2 - 2(n+1)y + 2^{\alpha + 1}
\end{align*}
Z Vietových vzorců dostaneme pro kořeny $y_1, y_2$ polynomu $p$:

\begin{align}
y_1y_2 &= 2^{\alpha + 1} \\
y_1 + y_2 &= 2n + 2 \label{eq:Viet}
\end{align}
Tedy $y_1 = 2^a, y_2 = 2^b$ pro nějaké $a,b \geq 0$, bez újmy na obecnosti $a \leq b$. Nyní rozebereme několik případů pro různé hodnoty $a$.

\begin{itemize}
\item[$a = 1$]
Tedy $y_1 = 2$ a $y_2 = 2n$. Po dosazení do polynomu $p$ dostaneme hodnoty $n = 1$ nebo $n = 2$, což jsou nesmyslné hodnoty pro kódy.
\item[$a = 2$]
Tedy $y_1 = 4$ a $y_2 = 2n -2$. Stejným způsobem jako v předchozím bodu dosadíme do $p$ a spočteme $n = 2$ nebo $n = 5$. Pro $n = 5$ dostaneme triviální opakovací kód.
\item[$a \geq 3$]
Z rovnice~\ref{eq:Viet} a faktu, že $a,b\geq 3$ dostaneme pro nějaké $k$:
\begin{align}
2n + 1 &= 2^a + 2^b - 1 \\
&= 8k - 1 \\
(2n + 1)^2 &= 64k^2 - 16k + 1 \label{eq:Mod1}
\end{align}

A dosadíme do rovnice~\ref{eq:SpherePacking2}:
\begin{align}
(2n + 1)^2 = 2^{\alpha + 3} - 7 \label{eq:Mod7}
\end{align}

Pravá strana rovnice~\ref{eq:Mod1} modulo 16 se rovná $1$, zatímco pravá strana rovnice~\ref{eq:Mod7} modulo 16 se rovná $-7$, což je spor, neboť by se pravé strany obou rovnic měly rovnat. Pro $a \geq 3$ tedy neexistuje žádný perfektní kód. \qed

\end{itemize}

\vt Pro $t \leq 3$ a $q > 2$ neexistuje $t$-perfektní kód nad abecedou s $q$ znaky.

Důkaz této části je nejnáročnější, proto ho rozdělíme do několika lemmátek. Hlavní roli budou mít kořeny Lloydova polynomu, které si označíme $\sigma_1, \dots, \sigma_t$ a pro které platí:

\begin{align*}
2 < \sigma_1 < \dots < \sigma_t < n
\end{align*}

Nerovnosti mezi kořeny máme z Lloydovy věty. Po dosazení čísel $0, 1$ a $2$ do Lloydova polynomu, zjistíme, že ani jedno z těchto čísel není kořenem, tedy $2 < \sigma_1$.

\lm $\prod\limits_{i = 1}^{t} \sigma_i = t!q^{\alpha - t}$

\dk Nejprve si vyjádříme součin kořenů pomocí koeficientů Lloydova polynomu.
\begin{align*}
L_t(x) &= a_t x^t + \dots + a_1x + a_0 \\
&= a_t (x^t + \frac{a_{t-1}}{a_t}x^{t-1} + \dots + \frac{a_0}{a_t} \\
&= a_t (x - \sigma_1)\dots(x - \sigma_t) \\
&= a_t\Bigl(x^t - \Bigl(\sum \sigma_i\Bigr)x^{t-1} + \dots + (-1)^t \prod \sigma_i\Bigr)
\end{align*}

Tedy máme:
\begin{align}
\prod\limits_{i = 1}^{t} \sigma_i = (-1)^t \frac{a_0}{a_t} \label{eq:ProdRoot}
\end{align}

Nyní spočítáme koeficienty $a_0$ a $a_t$:
\begin{align*}
a_0 &= L_t(0) = \sum\limits_{i = 0}^{t} {n \choose i}(q - 1)^i = q^\alpha \\
a_t &= \sum\limits_{i = 0}^{t} (-1)^i (q - 1)^{t - i} \frac{1}{i!}(-1)^{t-i} \frac{1}{(t-i)!} \\
&= \frac{(-1)^t}{t!}\sum\limits_{i = 0}^{t}(q-1)^{t-i} \frac{t!}{i!(t-i)!} \\
&= \frac{(-1)^t}{t!}\bigl((q - 1) + 1\bigr)^t
\end{align*}

Pro poslední rovnost jsme použili binomickou větu. Po dosazení do rovnice~\ref{eq:ProdRoot} dostaneme:
\begin{align*}
\prod\limits_{i = 1}^{t} \sigma_i = (-1)^t \frac{q^\alpha}{\frac{(-1)^t}{t!}q^t} = t!q^{\alpha-t}
\end{align*}
\qed

\lm $2 \sigma_1 \leq \sigma_t$

\dk Definujme si funkci $f(x) = k \Leftrightarrow x = p^hk$, kde pro $p$ platí $q = p^r$ a $p$ je nesoudělné s $k$. Aplikujme $f$ na součin kořenů:
\begin{align*}
f(\sigma_1)f(\sigma_2) \dots f(\sigma_t) &= f(\sigma_1\sigma_2 \dots \sigma_t) =\\
&= f(t!q^{a-t}) = f(t!) \leq t!
\end{align*}

Uvažme nyní 2 možnosti:
\begin{enumerate}
\item Nechť existují $i \neq j$ takové, že $f(\sigma_i) = f(\sigma_j) = k$, pak:
\begin{align*}
\sigma_i &= p^{h_i}k \\
\sigma_j &= p^{h_j}k \\
\end{align*}
Bez újmy na obecnosti platí $\sigma_i < \sigma_j$ a pak tedy $h_i < h_j$ a $p\sigma_i \leq \sigma_j$. Když všechny nerovnosti dáme dohromady:
\begin{align*}
\sigma_t \geq \sigma_j \geq p\sigma_i \geq 2\sigma_i \geq 2\sigma_1
\end{align*}
\item Nechť jsou tedy všechny $f(\sigma_i)$ různé. Jelikož je jejich součin menší než $t!$, pak se mezi $f(\sigma_1),\dots ,f(\sigma_t)$ vyskytují všechna čísla $1,\dots,t$. Jelikož $t \geq 3$ a $p$ je nesoudělné se všemi čísly $1,\dots,t$, tak $p > 3$. Evidentně musí existovat $i,j$ taková, že:
\begin{align*}
\sigma_i &= p^{h_i} \\
\sigma_j &= 2p^{h_j}
\end{align*}
Rozebereme 2 možnosti:
\begin{enumerate}
\item Nechť $h_j \geq h_i$, pak: $\sigma_t \geq \sigma_j = 2p^{h_j} \geq 2p^{h_i} = 2\sigma_i \geq 2\sigma_1$
\item Nechť $h_i > h_j$, pak: $\sigma_t \geq \sigma_i = p^{h_i} \geq p^{h_j + 1} = \frac{p}{2}\sigma_j \geq 2\sigma_j \geq 2\sigma_1$
\end{enumerate}
\end{enumerate}
\qed

\lm $\sigma_1\sigma_t \leq \frac{8}{9}(\frac{\sigma_1 + \sigma_t}{2})^2$

\dk Celou nerovnost vynásobíme $\sigma^2_1$:
\begin{align*}
\frac{\sigma_t}{\sigma_1} \leq \frac{8}{9}\Bigl(\frac{1 + \frac{\sigma_1}{\sigma_t}}{2}\Bigr)^2
\end{align*}
Provedeme substituci $x = \frac{\sigma_t}{\sigma_1}$:
\begin{align*}
x \leq \frac{8}{9}\Bigl(\frac{1 + x}{2}\Bigr)^2
\end{align*}
Po elementárních úpravách dostaneme:
\begin{align*}
0 \leq \Bigl(x - \frac{1}{2}\Bigr)\Bigl(x - 2\Bigr)
\end{align*}
Což platí, protože z předchozího lemmatu víme, že $x \geq 2$. \qed


\lm $\prod\limits_{i = 1}^t \sigma_i \geq \frac{n^t (q-1)^t}{q^t}\bigl(1 - \frac{t(t-1)}{2n}\bigr)$

\dk Víme, že:
\begin{align*}
\prod\limits_{i = 1}^t \sigma_i = t!q^{\alpha-t}
\end{align*}
Pravou stranu následně budeme upravovat:
\begin{align*}
\frac{t!}{q^t}q^\alpha &= \sum\limits_{i = 0}^{t}(q - 1)^i{n \choose i} \\
&\geq \frac{t!}{q^t} (q - 1)^t \frac{n(n-1)\dots(n-t+1)}{t!}  \\
&= \frac{(q-1)^t}{q^t} n^t \Bigl(1 - \frac{1}{n}\Bigr)\Bigl(1 - \frac{2}{n}\Bigr)\dots\Bigl(1 - \frac{t-1}{n}\Bigr)
\end{align*}
Nyní použijeme vzoreček, který platí pro $x_1,\dots,x_k \in (0,1)$:
\begin{align*}
\prod\limits_{i=1}^{k} (1 - x_i) \geq 1 - \sum\limits_{i=1}^{k} x_i
\end{align*}
Po aplikaci na $(1 - \frac{1}{n})(1 - \frac{2}{n})\dots(1 - \frac{t-1}{n})$, dostaneme:
\begin{align*}
\frac{t!}{q^t}q^\alpha &\geq \frac{n^t (q-1)^t}{q^t}\Bigl(1 - \sum\limits_{i=1}^{t-1} \frac{i}{n}\Bigr) \\
&\geq \frac{n^t (q-1)^t}{q^t}\Bigl(1 - \frac{t(t-1)}{2n}\Bigr) 
\end{align*}
\qed







\end{document}
