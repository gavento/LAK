\subsection{Mixing lemma}


\vt (Mixing lemma) $\forall G$ $d$-regulární, $\forall S,T \subseteq V, S\cap T = \emptyset: |e(S,T) - {d\cdot|S|\cdot|T|\over n}| \le \lambda d\cdot\sqrt{|S|\cdot|T|}$

\dk Buďte $\chi_S, \chi_T$ charakteristické vektory $S$ a $T$. $u = (1,1,\dots)$ je první vlastní vektor. $\chi_S^\bot$ značí vektor kolmý na $u$.

$${\sk{\chi_S \cdot u} \over \|u\|^2} = {|S|\over n} \qquad\Rightarrow\qquad \chi_S = u\cdot {|S|\over n} + \chi_S^\bot \qquad\qquad \chi_T = u\cdot {|T|\over n} + \chi_T^\bot$$

$$e(S,T) = \sum_{i\in S, j\in T} A_{ij} = \chi_T^TA\chi_S = \underbrace{{|S|\cdot |T| \over n^2} \underbrace{u^TAu}_{d\|u\|^2=dn}}_{d\cdot |S|\cdot |T|\over n} + \xttt A\chi_s^\bot$$

Zbývá dokázat, že $|\xttt A\chi_S^\bot| \le \lambda\cdot\sqrt{|S|\cdot |T|}$.

$$|\xttt A\chi_S^\bot| \le \|\xttt \| \cdot \|A\chi_S^\bot\| \le \|\xttt\|\cdot\lambda\cdot\|\chi_S^\bot\|$$

První nerovnost plyne z toho, že skalární součin dvou vektorů (tedy součin
jejich délek a sinu úhlu, který svírají) je vždy nejvýš roven součinu jejich
délek. Druhá nerovnost plyne z toho, že si $\chi_S^\bot$ můžu vyjádřit jako
lineární kombinaci vlastních vektorů $A$:

$$\chi_S^\bot = \sum_{i=2}^n y_i\alpha_i$$

Pro každý vlastní vektor $y_i$ můžu nahradit matici $A$ vlastním číslem
$\lambda_i$ (pak bude zachována rovnost) a tím spíš můžu nahradit matici $A$
největším vlastním číslem, což je v našem případě $\lambda =
\max\{\lambda_2,-\lambda_n\}$, abych zachoval nerovnost.

$$\|\chi_S\|^2 = |S| \qquad\Rightarrow\qquad \|\xttt\| \le \sqrt S$$
$$\|\chi_T\|^2 = |T| \qquad\Rightarrow\qquad \|\chi_S^\bot\| \le \sqrt T$$

$$|\xttt A\chi_S^\bot| \le \lambda\cdot\sqrt{|S|\cdot |T|}$$
\qed

