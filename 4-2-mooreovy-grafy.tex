\subsection{Mooreovy grafy}
% Stanislava Tlustá

% úvod
% % motivace
V této části využijeme znalostí o vlastních číslech k přiblížení toho, jak mohou vypadat regulární grafy bez krátkých cyklů.

% % Mooreova podmínka
Nejprve si ukažme kolik může mít takový $r$-regulární graf bez krátkých cyklů (troj- a čtyřúhelníků) vrcholů. 
Vezměme jeden vrchol. Ten má $r$ sousedů, z nichž žádné dva spolu nesousedí (vznikl by trojúhelník). Každý z nich má $r-1$ nových sousedů. Ti musí být různí (vznikl by čtyřúhelník).

\begin{align}
\label{4-2:graf-Petersen}
\begin{tikzpicture}[thick,scale=1.1]
% obrázek: Petersenův graf (jako Mooreův)
% 1. vrstva
\draw (0,0) -- (3,2) (0,0) -- (3,0) (0,0) -- (3,-2);
\draw (0,0) \vrchol;
% 2. vrstva - 1. vrchol
\draw (3,2) -- (6,2.5) \vrchol (3,2) -- (6,1.5) \vrchol;
\draw (3,2) \vrchol;
% 2. vrstva - 2. vrchol
\draw (3,0) -- (6,0.5) \vrchol (3,0) -- (6,-0.5) \vrchol;
\draw (3,0) \vrchol;
% 2. vrstva - 3. vrchol
\draw (3,-2) -- (6,-1.5) \vrchol (3,-2) -- (6,-2.5) \vrchol;
\draw (3,-2) \vrchol;
% 3. vrstva - přerušované čáry
\draw[black!50, dashed] (6,0.5) to[bend right] (6,2.5);
\draw[black!50, dashed] (6,-2.5) to[out=20,in=340] (6,2.5);
\draw[black!50, dashed] (6,-0.5) to[bend right] (6,1.5);
\draw[black!50, dashed] (6,-1.5) to[out=20,in=340] (6,1.5);
\draw[black!50, dashed] (6,-1.5) to[bend right] (6,0.5);
\draw[black!50, dashed] (6,-2.5) to[bend right] (6,-0.5);
% popisky (dole)
\draw (0,-3) node {1. vrchol};
\draw (3,-3) node {2. vrstva};
\draw (3,-3.5) node {r sousedů};
\draw (6,-3) node {3. vrstva};
\draw (6,-3.5) node {r(r-1) vrcholů};
%% alternativa
%%% popisky (nahoře)
%\draw (0,3.5) node {1. vrchol};
%\draw (3,3.5) node {2. vrstva};
%\draw (3,3) node {r sousedů};
%\draw (6,3.5) node {3. vrstva};
%\draw (6,3) node {r(r-1) vrcholů};
\end{tikzpicture}
\end{align}
Dostáváme tedy odhad:
\begin{align}
%\label{mooreova-podminka}
|V| \geq 1 + r + r(r-1) = r^2 + 1
\end{align}

% % definice
\df Mooreův graf je takový $r$-regulární graf na $r^2 + 1$ vrcholech, který neobsahuje troj- a čtyřúhelníky.

Mooreovy grafy jsou tedy nejmenší možné regulární grafy bez krátkých cyklů. Podle následující věty není však až na výjimky možno tohoto ideálu dosáhnout.

\vt Nechť $G$ je $r$-regulární Mooreův graf. Pak $r \in \set{1, 2, 3, 7, 57}$.

\pzn \\
Pro $r = 1$ je hledaným grafem jedna hrana. \\
Pro $r = 2$ je to pětiúhelník. \\
Pro $r = 3$ je to Petersenův graf (viz obrázek \ref{4-2:graf-Petersen}). \\
Pro $r = 7$ je to takzvaný Hoffman-Singletonův graf. \\
Pro $r = 57$ není zatím známo, zda lze takový graf skutečně sestrojit.

\dk
% jak to budeme dělat
Pomocí poznatků o vlastních číslech získaných v předchozí části sestavíme rovnici, která nám přesně vymezí, co musí $r$ splňovat.

Matici souslednosti grafu $G$ označme $A$. Její druhá mocnina zachycuje počet sledů délky 2. Má tedy na diagonále stupeň ($r$). Ukážeme, že mimo diagonálu má oproti matici $A$ prohozené nuly a jedničky.
Je-li $uv \in E(G)$, pak mezi $u$ a $v$ nemůže existovat cesta délky $2$ (vznikl by trojúhelník).
Pokud $uv \not\in E(G)$, tak se podíváme na konstrukci na obrázku \ref{4-2:graf-Petersen}. Bez újmy na obecnosti můžeme předpokládat, že $u$ je 1. vrchol. Vidíme, že existuje cesta délky $2$. Ta může být jen jedna, jinak by vznikl čtyřúhelník.

Dostáváme tedy:
\begin{align}
A + A^2 = rE + (J - E),
\end{align}
kde $E$ je jednotková matice a $J$ je matice samých jedniček. Úpravou dostaneme polynomiální vztah
\begin{align}
p(A) = A^2 + A + (1-r)E = J.
\end{align}
% spektra
Dále platí, že pro $\lambda \in \Sp(A)$ je $p(\lambda) \in \Sp(J)$. Spektrum $J$ známe: obsahuje $n = r^2 + 1$ s násobností $1$ a $0$ s násobností $n-1$.

% % \lambda = r
Protože $G$ je $r$-regulární, tak je $r$ také jeho vlastním číslem. Dosazením do $p$ dostaneme:
\begin{align}
p(r)= r^2 + r + (1-r) = r^2 + 1 = n.
\end{align}
% % p(x)=0
Nyní zbývá vyřešit případ, kdy $p(\lambda)=0$.
Řešení této kvadratické rovnice je 
\begin{align}
\lambda_{1,2} = \frac{-1\pm\sqrt{4r-3}}{2}.
\end{align}
Násobnosti těchto kořenů označíme $m_1, m_2$. Jejich součet je zřejmě roven $n-1 = r^2$.
% % stopa
Využitím faktu, že stopa matice je suma vlastních čísel 
včetně násobností, získáme rovnici
\begin{align}
	0 = \Tr(A) = r+m_1\lambda_1 + m_2\lambda_2.
\end{align}
Její snadnou úpravou již získáme hledanou podmínku pro $r$
\begin{align}
2r - r^2 + \sqrt{4r-3}(m_1 - m_2) = 0 \label{4-2:rovnice-pro-r}
\end{align}

% % rozbor odmocniny
Řešení rozdělíme na dva případy.
\begin{enumerate}
% % % \not\in \Q
\item $\sqrt{4r-3} \not\in \Q$: potom $m_1 = m_2$ a tedy $r = 2$.
% % %  \in \Q
\item $\sqrt{4r-3} = s \in \Q$, což implikuje
\footnote{Odmocnina z přirozeného čísla je vždy přirozené číslo, či iracionální číslo, nikdy zlomek.} 
 $s \in \N$.
  Substitucí $4r - 3 = s^2$ do rovnice \ref{4-2:rovnice-pro-r} získáme
  \begin{align}
  -s^4 + 2s^2 + 16(m_1 - m_2)s +15 =0.
  \end{align}
  Tudíž $s|15$. Pro jednotlivé hodnoty	$s \in \set{1, 3, 5, 15}$, dostáváme $r \in \set{1, 3, 7, 57}$.
\end{enumerate}
\qed
