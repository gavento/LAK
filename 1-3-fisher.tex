\subsection{Fišerova nerovnost}


\vt Nechť máme graf $K_n$ a jeho hranově disjunktní rozklad na $m$ úplných 
bipartitních grafů. Potom $m \geq n-1$.

\dk Označme si úplné bipartitní grafy $B_1, \ldots, B_m$ a $X_k$, $Y_k$ jejich 
partity, přičemž jednotlivý $B_i$ nemusí být pokrývat všechny vrcholy $K_n$.  
Mějme matici $A_k$ pro graf $B_k$ velikosti $n \times n$ definovanou:
\begin{align}
	a_{ij} = \left\{\begin{array}{ll}1 & \text{pokud } i \in X_k\text{ a }j \in 
	Y_k \\ 0 & \text{jinak} \end{array}\right.
\end{align}
Protože v každém nenulovém řádku jsou jedničky právě pro sousedy daného vrcholu 
v druhé partitě, jsou všechny nenulové řádky stejné (sousedství jsou stejná), 
$A_k$ má tedy hodnost $1$.

Nyní uvažme matici $A=A_1 + \ldots + A_m$. Hodnost součtu je nanejvýš rovna 
součtu hodností, proto $\rank(A) \leq m$. Nyní budeme chtít dokázat, že 
$\rank(A) \geq n-1$:

Protože každá hrana grafu náleží právě jednomu $B_k$, je jednička právě na 
jednom z míst $a_{ij}$ nebo $a_{ji}$ (pozor, matice nejsou matice sousednosti -- 
rozlišují partitu!). Na diagonále $A$ jsou pak samé nuly. Sečtením $A+A^T$ 
získáme matici incidence $K_n$, tedy $A+A^T=J_n - I_n$.

Dále pro spor předpokládejme, že $\rank(A) \leq n-2$. Připíšeme k matici jeden 
řádek samých jedniček, čímž hodnost zvýšíme nanejvýš o $1$. Protože ale $A$ nemá 
plnou hodnost, existuje netriviální lineární kombinace sloupců, která dává 
$\vec{0}$ -- nechť jsou její koeficienty zaznamenány ve vektoru $\vec{x} \in 
\R^n$ a tedy $A\vec{x}=\vec{0}$. Zároveň protože poslední řádek jsou samé 
jedničky, platí $\sum x_i \cdot 1 = 0$ a tedy také $J_n\vec{x} = 0$. Počítejme 
dvěmi způsoby:

\begin{align}
	&x^T(A + A^T) x = x^T(J_n - I_n)x = x^T(J_nx) - x^T(I_nx) = 0 - x^T x = - 
	\sum x_i^2 < 0 \\
	&x^T(A+A^T) x = x^TA^Tx + x^TAx = 0^Tx + x^T0 = 0
\end{align}

což dává spor. \qed


