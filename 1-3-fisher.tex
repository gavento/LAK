\subsection{Fisherova nerovnost}


\vt (Fisherova nerovnost) Mějme hranově disjunktní rozklad úplného grafu $K_n$ na $m$ úplných 
bipartitních grafů. Pak $m \geq n-1$.

\dk Označme $B_1, \ldots, B_m$ úplné bipartitní grafy v~rozkladu $K_n$. Dále označme $X_i$, $Y_i$ partity grafu $B_i$ a $A_i=((A_i)_{jk})$ matici o~rozměrech $n \times n$ indexovanou vrcholy grafu $K_n$ a definovanou následovně:
\begin{align}
	(A_i)_{jk} = \begin{cases}1, & \text{pokud } j \in X_i\text{ a }k \in Y_i, \\
	0 & \text{jinak.} \end{cases}
\end{align}
Nulové řádky matice $A_i$ odpovídají vrcholům grafu $K_n$ mimo partitu $X_i$, zatímco jedničky v~nenulových řádcích odpovídají sousedům vrcholů z~$X_i$. Protože $B_i$ je úplný bipartitní graf, mají všechny vrcholy z~$X_i$ stejné sousedy. Všechny nenulové řádky jsou tedy stejné a matice $A_i$ má hodnost $1$.

Položme $A=A_1 + \ldots + A_m$. Protože každá hrana grafu $K_n$ náleží právě jednomu $B_i$, má matice $A=(a_{jk})$ jedničku právě na 
jednom z~míst $a_{jk}$ nebo $a_{kj}$. Na diagonále $A$ jsou samé nuly. Z~předchozích pozorování vyplývá, že $A+A^T$ je matice incidence $K_n$, tedy $A+A^T=J-I$, kde $J$ je matice samých jedniček.

Ukážeme, že $\rank(A) \geq n-1$. Pro spor předpokládejme, že $\rank(A) \leq n-2$. Přidáním řádku samých jedniček k~matici $A$ vytvoříme matici $A'$, pro kterou platí $\rank(A')\leq n-1$. Protože $A'$ nemá plnou hodnost, existuje netriviální lineární kombinace jejích sloupců, která dává nulový vektor. Nechť jsou její koeficienty zaznamenány ve vektoru $x=(x_1,\dots,x_n)^T$. Tedy $A'x=0$ a rovněž $Ax=0$. Protože v~posledním řádku matice $A'$ jsou samé jedničky, platí $\sum_{i=1}^n 1\cdot x_i = 0$, a tedy i $Jx=0$. Počítejme dvěma způsoby:
\begin{align}
  x^T(A+A^T) x &= x^TAx + x^TA^Tx = x^T0+0^Tx= 0,\\
	x^T(A+A^T) x &= x^T(J - I)x = x^TJx - x^TIx = x^T0 - x^Tx = -\smash{\sum_{i=1}^n} x_i^2 < 0,
\end{align}
což je spor. Tedy $\rank(A) \geq n-1$.

Zároveň platí, že $\rank(A)\leq\rank(A_1)+\dots+\rank(A_m)$, protože prostor řádkových vektorů matice $A$ je generován řádkovými vektory matic $A_1,\dots,A_m$. To nám dává nerovnost $n-1\leq\rank(A)\leq m$.\qed