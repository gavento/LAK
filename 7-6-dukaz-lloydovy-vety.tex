\subsection{Důkaz Lloydovy věty}


Zde začnou věci dávat větší smysl. Nejdříve dokážeme pomocí výše zmíněných lemmat
pomocné tvrzení, který dá podobný polynom, následně si s ním pohrajeme a získáme
polynom Lloydův, tak jak byl zadefinován na začátku.

\vt (Lloydův prototyp) Pokud existuje $t$-perfektní kód v $G$, potom
$x_t(\lambda)\backslash x_d(\lambda)$.

\dk Nejprve si všimněme, že
$\widehat{S_t} = \widehat{x_t(A)} = \widehat{\sum_i^t A_i} = \sum_i^t B_i = x_t(B)$.
Dále se podívejme na spektra $B$ a $\widehat{S_t}$:
\begin{align}
	\Sp(B) &= \{ k, \lambda_1, \ldots, \lambda_d \} \\
	\Sp(\widehat{S_t}) &= \{ x_t(k), x_t(\lambda_1), \ldots, x_t(\lambda_d) \}
\end{align}
Z povídání o $B$ (je tridiagonální) víme, že její vlastní čísla jsou různá.
Dle pomocného lemmatu výše má $\widehat{S_t}$ dimenzi jádra alespoň $t$,
a tedy 0 je její alespoň $t$-násobné vlastní číslo -- tj. alespoň $t$ hodnot
z $\lambda_1, \ldots, \lambda_d$ jsou kořeny $x_t$ ($k$ není kořen $x_t$ \todo{proč?}).
Stupeň $x_t$ je nejvýše $t$ (z definice), takže toto jsou všechny jeho kořeny
\todo{proč není identicky nulový}.

Dohromady tedy kořeny $x_t$ jsou podmnožinou kořenů $x_d$ (a žádné kořeny nejsou
více\-ná\-so\-bné), takže $x_t$ dělí $x_d$. \qed


\vt (Lloyd) Pokud existuje $t$-perfektní kód s parametry $(n,q)$, pak $L_t(x)$
(definice níže) má $t$ různých celočíselných kořenů mezi $1$ a $n$.
\begin{align}
	L_t(x) = \sum_{j=0}^t(-1)^j(q-1)^{t-j}\binom{x-1}{j}\binom{n-x}{t-j}
\end{align}

\dk Nejprve definujme graf $\Gamma$: Jeho vrcholy jsou slova délky $n$ nad abecedou
velikosti $q$ a hrany spojují právě slova lišící se v jednom znaku. Tento graf
je vzdálenostně regulární s průměrem $d = n$.
Speciálně budou užitečné následující jeho parametry\footnote{
Hodnoty těchto parametrů lze odvodit rozebráním, které pozice ve slově odpovídajícím
vrcholu $u$ mohu změnit a jak, abych získal $w$.
}:
\begin{align}
  s_{1,i,i-1} &= (q-1)(n + 1 - i) \\
  s_{1,i,i}   &= (q-2)i \\
  s_{1,i,i+1} &= i + 1
\end{align}

Nyní bude naším cílem upočítat polynomy $x_i$. K tomu začneme od rekurzivního vzorce
pro polynomy $v_i$, z něj spočteme jejich vytvořující funkci\footnote{
Zde je dobré vědět, jak se to dělá, ale samotný výpočet je asi možné přeskočit.
}
$V(t) = \sum_{i = 0}^\infty v_i(\lambda) t^i$ a od té dojdeme
k vytvořující funkci $X(t) = \sum_{i = 0}^\infty x_i(\lambda) t^i$ pro $x_i$.
Začneme z definice $v_i$:
\begin{align}
	v_0(\lambda) &= 1 \\
	v_1(\lambda) &= \lambda \\
	s_{1,i,i-1}v_{i-1}(\lambda) + (s_{1,i,i} -\lambda) v_i(\lambda) +
    &s_{1,i,i+1}v_{i+1}(\lambda) = 0 \ \ \forall i \in \{1, \ldots, d\}
\end{align}
Dosadíme za $s_{1,i,*}$:
\begin{align}
  (q-1)(n + 1 - i)v_{i-1}(\lambda) + ((q-2)i - \lambda)v_i(\lambda) +
    (i+1) v_{i+1} = 0
\end{align}

Nyní bychom rádi dosadili $V$ za $v_i$, abychom $V$ upočítali, ale k tomu
se potřebujeme zbavit $i$. Z vlastností vytvořujících funkcí víme:
\begin{align}
  \sum_{i=0}^\infty v_{i+1}(\lambda) x^i &= \frac{V(x) - 1}{x} \\
  \sum_{i=0}^\infty i v_i(\lambda) x^i     &= x \frac{\rmd V(x)}{\rmd x} \\
  \sum_{i=0}^\infty (i+1) v_{i+1}(\lambda) x^i &= \frac{\rmd V(x)}{\rmd x} \\
  \sum_{i=0}^\infty (i+2) v_{i+2}(\lambda) x^i &=
    \frac{1}{x}\br{ \frac{\rmd V(x)}{\rmd x} - \lambda }
\end{align}
Přepíšeme vztah pro $v_i$, aby platil od $i = 0$, dosadíme a upravíme:
\begin{align}
  (q - 1)n v_i - (q - 1) i v_i + (q-2)(i+1)v_{i+1}
    - \lambda v_{i+1} +(i+2)v_{i+2} &= 0 \\
  (q - 1)n V(x) - (q - 1)x \frac{\rmd V(x)}{\rmd x}
    + (q-2)\frac{\rmd V(x)}{\rmd x} - \lambda \frac{V(x) - \lambda}{x}
    + \frac{1}{x}\br{\frac{\rmd V(x)}{\rmd x} - \lambda } &= 0 \\
  V(x) \br{ (q - 1)n - \frac{\lambda}{x} } -
    \frac{\rmd V(x)}{\rmd x} \br{ (1 - q)x + (q-2) + \frac{1}{x} }
    + \frac{\lambda}{x} - \frac{\lambda}{x} &= 0 \\
  V(x)\br{ (1 - q)nx - \lambda } - \frac{\rmd V(x)}{\rmd x} \br{
    (q - 1)x^2 + (q-2)x + 1
    } &= 0
\end{align}
Nyní přeskupíme a zintegrujeme pomocí rozkladu na parciální zlomky:
\begin{align}
  \int \frac{\rmd V}{V} &=
    \int \frac{(q - 1)nx - \lambda}{(1 - q)x^2 + (q-2)x + 1} \rmd x \\
  \ln V + c &= \int \frac{(n+\lambda)(q-1)}{q(1 + x(q-1))} \rmd x +
               \int \frac{\lambda - n(q - 1)}{q(1 - x)} \rmd x \\
   &= \frac{(n+\lambda)(q-1)}{q} \int \frac{1}{1 + x(q-1)} \rmd x +
               \frac{\lambda - n(q - 1)}{q} \int \frac{1}{1 - x} \rmd x \\
   &= \frac{(n+\lambda)(q-1)}{q} \frac{\ln (1 + x(q-1))}{q-1} +
               \frac{\lambda - n(q - 1)}{q} (-1) \ln (1 - x) \\
   &= \frac{n+\lambda}{q} \ln (1 + x(q-1)) -
               \frac{\lambda - n(q - 1)}{q} \ln (1 - x)
\end{align}
Nyní se zbavíme $c$ -- víme, že $V(x = 0) = 1$, takže $c = 0$ -- a logaritmů.
\begin{align}
  V = (1 + x(q-1))^{\frac{n + \lambda}{q}} (1 - x)^{\frac{n(q-1) - \lambda}{q}}
\end{align}
Hurá máme $V$, nyní chceme $X$ -- $x_i$ je posloupnost částečných součtů
$v_i$ a jak si všichni dobře pamatujeme posloupnost částečných součtů získáme
z vytvořující funkce jejím vynásobením vytvořující funkcí posloupnosti samých jedniček
$J(x) = \sum_{i=0}^\infty x^i = \frac{1}{1-x}$:
\begin{align}
  X = VJ =  (1 + x(q-1))^{\frac{n + \lambda}{q}} (1 - x)^{\frac{n(q-1) - \lambda}{q} - 1}
\end{align}
Provedeme substituci $y = n - \frac{n + \lambda}{q}$
a rozvineme dle zobecněné binomické věty, abychom měli explicitně vyjádřený
polynom $x_t$:
\begin{align}
  X(x) &= (1 + x(q-1))^{n - y} (1 - x)^{y - 1} \\
       &= \br{\sum_{i=0}^\infty {n - y \choose i} (q-1)^i x^i}
          \br{\sum_{i=0}^\infty {y - 1 \choose i} (-x)^i } \\
       &= \sum_{i=0}^\infty \br{\sum_{j=0}^i {n - y \choose j} (q-1)^j
                                {y - 1 \choose i - j} (-1)^{i-j}} x^i \\
%
  x_t(\lambda) &= \sum_{j=0}^t {n - y(\lambda) \choose j} (q-1)^j
                                {y(\lambda) - 1 \choose t - j} (-1)^{t-j}
\end{align}
Nyní zásadní trik: všimneme si, že když $y \in \set{1, \ldots, n}$, tak
$X$ (v uzavřeném tvaru) je polynom v $x$ stupně $\leq n - 1$ --
tedy koeficient před $x^t$ je 0
a $\lambda = n(q-1) - qy$ je kořenem $x_n$. Takto máme $n$ různých kořenů
$x_n$, a protože je to polynom stupně nejvýše $n$, jsou to všechny jeho kořeny.
Z toho a aplikace Lloydovy věty pro vzdálenostně regulární grafy plyne,
že $L_t(y)$ musí mít $t$ různých kořenů mezi čísly 1 až $n$.
\begin{align}
  L_t(y) &= \sum_{j=0}^t {n - y \choose j} (q-1)^j
                                {y - 1 \choose t - j} (-1)^{t-j}
\end{align}
\qed

