\df {\it Skalárním součinem} vektorů $x=(x_1,\dots,x_n)^T$ a $y=(y_1,\dots,y_n)^T$ z~vektorového prostoru $\Ft^n$ rozumíme číslo $\langle x,y\rangle=\sum_{i=1}^nx_iy_i\in\Ft$ (případně $\langle x,y\rangle=\sum_{i=1}^nx_i\overline{y_i}$ pro $\Ft=\C$).

\medskip
\subsection{Ortogonální doplněk}


\df Nechť $M \subseteq \Ft^n$. Množinu $M^\bot = \{x \mid \forall a\in M\colon \sk{x,a} = 0\}$ nazveme {\it ortogonálním doplňkem} $M$.

\df {\it Součet podprostorů} $\langle M\rangle$ a $\langle N\rangle$ (symbol $\langle X\rangle$ značí lineární obal $X$) definujeme jako
\begin{align}
\langle M\rangle + \langle N\rangle = \left\{ u~+ v~\mid u~\in \langle M\rangle, v~\in \langle N \rangle\right\} = \langle M\cup N\rangle.
\end{align}

\poz Pro $M,N\subseteq\Ft^n$ platí:
\begin{enumerate}\itemsep1pt \parskip0pt \parsep0pt
	\item[(i)] $\dim M^\bot = n - \dim \langle M\rangle$,
	\item[(ii)] ${(M^\bot)}^\bot = \langle M\rangle$,
	\item[(iii)] $(\langle M\rangle \cap \langle N\rangle)^\bot = M^\bot + N^\bot$,
	\item[(iv)] $(\langle M\rangle+\langle N\rangle)^\bot = M^\bot \cap N^\bot$,
	\item[(v)] $\langle M\rangle + M^\bot =\Ft^n$.
\end{enumerate}

\vt (Dimenze spojení a průniku) Pro podprostory $U,V\subseteq\Ft^n$ platí \begin{align}
\dim(U + V) + \dim(U \cap V) = \dim U+ \dim V.
\end{align}

Všimněme si, že pro tři podprostory už předchozí pozorování neplatí. Máme-li například v~rovině tři přímky $p,q,r$ procházející počátkem, pak $\dim(p + q + r) =2$, zatímco $\dim p + \dim q + \dim r - \dim(p \cap q) - \dim(p \cap r) - \dim(q \cap r) + \dim(p \cap q \cap r)=3$.

\dsl Nechť $U,V \subseteq \mathbb{F}^n$ jsou podprostory, pro které platí $\dim U + \dim V > n$. Pak $\dim U \cap V \ge 1$, tedy existuje $u \neq 0$, $u \in U\cap V$.

\dsl V~prostorech, ve kterých je skalární součin opravdu skalárním součinem, tedy $\sk{x,x} \neq 0$ pro $x\neq 0$, platí navíc $U\cap U^\bot = \{0\}$.

\medskip
Například v~$\GF(2)^2$ je $\sk{(1,1), (1,1)} = 0$, tedy $(1,1)\in\langle(1,1)\rangle\cap\langle(1,1)\rangle^\bot$.