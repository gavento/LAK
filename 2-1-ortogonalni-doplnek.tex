\df (Skalární součin) Dvěma vektorům $x,y$ z~vektorového prostoru $V = \mathbb{F}^n$
přiřadíme skalární součin $\sk{x,y} = \sum x_iy_i$ \quad(případně $\sk{x,y} = \sum
x_i\overline{y_i}$ pro $\mathbb{F} = \C$).

\subsection{Ortogonální doplněk}

\df $M \subseteq \mathbb{F}^n$ \quad $M^\bot = \{x \mid \forall a\in M\colon \sk{x,a} = 0\}$ je
ortogonální doplněk $M$.

\poz $\dim M^\bot = n - \dim \L M$

\df (Součet podprostorů) $\L M + \L N = \left\{ u + v \mid u \in \L M, v \in \L N \right\} = \L{(M\cup N)}$

\poz $(A \cap B)^\bot = A^\bot + B^\bot$

\poz $(A + B)^\bot = A^\bot \cap B^\bot$

\poz $\dim(M+M^\bot) = n$ a také $M + M^\bot = \mathbb{F}^n$

\poz ${(M^\bot)}^\bot = \L M$

\dk \uv{$\supseteq$} jednoduché, \uv{$\subseteq$} přes dimenze\quad $n-(n-k) = k =
\dim \L M$ \qed

\poz $\dim\left(\L M + \L N\right) + \dim\left(\L M \cap \L N\right) = \dim \L M + \dim \L N$

Pozor, pro tři podprostory už předchozí pozorování neplatí! Například v~rovině tři
přímky $U,V,W$ procházející počátkem, pak $\dim(U + V + W) \neq \dim(U) + \dim(V) +
\dim(W) - \dim(U \cap V) - \dim(U \cap W) - \dim(V \cap W) + \dim(U \cap V \cap W)$
(čísly $2 \neq 1 + 1 + 1 - 0 - 0 - 0 + 0$).

\dsl Mějme podprostory $M,N \ll \mathbb{F}^n$, ve kterých platí $\dim M + \dim N > n$,
pak $\dim M \cap N \ge 1$, tedy $\exists u \neq 0, u \in M\cap N$.

\dsl Navíc pro tělesa, ve kterých standardní skalární součin je opravdu skalárním
součinem, tedy $\sk{x,x} \neq 0$ pro $x\neq 0$, máme:  $M\cap M^\bot = \{0\}$.

Například $\GF(2)^2$ předchozí podmínku nesplňuje a dokonce platí $\sk{(1,1), (1,1)} =
0$, tedy vektor $(1,1)$ je kolmý sám na sebe v~$\GF(2)^2$.

