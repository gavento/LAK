\subsection{Vzdálenostně regulární grafy}


\df Vzdálenostně regulární graf je regulární a $\exists s_{hij}$ takové, že $\forall u,v\in V(G), d_G(u,v) = j:$ $|\{w: d_G(u,w) = h, d_G(w,v) = i\}| = s_{hij}$.

\poz $|h-j| > j \Rightarrow s_{hij} = 0$ (plyne z $\Delta$ nerovnosti), $k = s_{110}$ (počet sousedů vrcholu $u = v$ v $k$-regulárním grafu)

\lm $Z_{mi} = Z_{m-1,i-1} \cdot s_{1,i-1,i} + Z_{m-1,i} \cdot s_{1,i,i} + Z_{m-1,i+1} \cdot s_{1,i+1,i}.$ $Z_{mi}$ značí počet sledů délky $m$ mezi vrcholy ve vzdálenosti $i$.

\dk $Z_{00} = 1$, jinak $Z_{0i} = 0$. Dále dokážeme indukcí pro $m \ge 1$ a $i
\ge 1$. $s_{1,i,j}$ je nenulové pouze pro $i \in \{j-1,j,j+1\}$ (z $\Delta$
nerovnosti). V rovnici sčítáme vrcholy sousedící s $u$, které jsou ve
vzdálenosti $i-1$, $i$ a $i+1$ od $v$.

\df Matice sousednosti $A = A_G$. $\A(G) = \{p(A): p(x) \in \C[x]\}$. $\A(G)$ je
vektorový prostor.

\df Vzdálenostní matice $A_1, A_2, \dots, A_d$ grafu $G$: \\
\indent $(A_i)_{uv} = \left\{\begin{matrix}
1\quad & d_G(u,v) = i \hfill & \hspace{4cm} A_0 = I \\
0\quad & \text{jinak} \hfill & \hspace{4cm} A_1 = A \\
\end{matrix}\right.$


