\subsection{Samoopravné kódy}



\df Samoopravný kód $C$ s parametry $(n, M)_q$ nad abecedou $A$ je podmnožina $A^n$, kde $|A|=q$ a $|C|=M$. Prvkům množiny $C$ říkáme kódová slova.

Nejčastěji $A$ je konečné těleso $GF(q)$ o $q$ prvcích nebo $A=\{0, \dots, q-1\}$. Pokud $C$ je vektorový podprostor nad tělesem $A$, pak $C$ nazýváme lineárním kódem. Množinu $A^n$ navíc vybavíme Hammingovou metrikou $d(~,~)$. Pro dvě slova $x, y \in A^n$ se složkami $x_i$ resp. $y_i$ platí $$d(x, y)=|\{i~|~x_i\neq y_i\}|.$$ Minimální vzdálenost kódu je pak definována jako $d=\min_{x \neq y \in C} d(x, y)$. Mluvíme pak o $(n, M, d)_q$ kódu. 

Chceme, aby kód měl co největší minimální vzdálenost (při co největší mohutnosti). To souvisí s tím, že pokud vysíláme kódové slovo $c \in C$, může během přenosu dojít k chybám (uvažujeme pouze změnu složky nikoliv zkrácení délky) a druhá strana přijme slovo $y= c + e$, kde $e$ je chybové slovo. Příjemce se pak snaží chybu detekovat a případně opravit $y$ na nejbližší kódové slovo $c' \in C$ (vše je měřeno Hammingovou metrikou). Pokud kódová slova budou co nejdále od sebe, je detekce a oprava $y$ na správné kódové slovo $c$ (tj. $c=c'$) více pravděpodobná. Přesněji pokud počet chyb (což je počet nenulových složek chybového slova $e$) je $\leq d-1$ je možné chybu detekovat (přijmeme-li nekódové slovo, víme, že nastala chyba). Pokud pokud počet chyb je $\leq \lfloor\frac{d-1}{2}\rfloor=:t$ je možné chybu opravit. Označme $N_t(c)=\{x\in A^n~|~d(c, x)\leq t\}$ okolí slova $c$ do vzdálenosti $t$. Vidíme, že okolí $N_t(c)$ pro všechna $c\in C$ jsou disjunktní, kde $C$ má minimální vzdálenost $d$ a $t$ je definováno výše. 





%\df Kód $C$ je $t$-perfektní, pokud opravuje $t$ chyb a navíc úplně pokrývá svou nosnou množinu $M$.

\tv(Hammingův odhad) Mějme $(n, M, d)_q$ kód $C$. Označme $t=\lfloor\frac{d-1}{2}\rfloor$. Pak
\begin{align*}
	|C| \leq { q^n \over \sum_{i=0}^t \binom{n}{i} (q-1)^i }
\end{align*}

\dk Stačí si uvědomit, že okolí jsou disjunktní a obsahují všechny stejně slov (zde nezáleží na středu okolí). Dostáváme tak: 
$$q^n\geq \sum_{c\in C} |N_t(c)|=M\cdot \sum^{t}_{i=0} \binom{n}{i}(q-1)^i.$$
\qed

\df Kódy, která nabývají rovnosti v Hammingově odhadu nazýváme perfektními.

Nyní si ukážeme základní příklady perfektních kódů. Mezi ty triviální patří totální $(n, q^n, 1)$ kód obsahující všechna slova z $A^n$, opakovací $(n, 2, n )$ kód pro lichou délku $n$ a jednoprvkový kód. 

Každý lineární kód můžeme popsat jeho bází. Generující matice $G$ o rozměrech $k \times n$ lineárního $(n, q^k)$ kódů $C$ nad $GF(q)$ má v řádcích zapsanou jeho bázi. Kontrolní matice lineárního kódu $C$ je taková matice $H$ o rozměrech $(n-k) \times n$, že $c\in C \Leftrightarrow Hc^T=0$. Platí $HG^T=0$. Lineární kód můžeme jednoznačně popsat jeho generující nebo kontrolní maticí.

\df Hammingův kód $\mathcal{H}(r, q)$ je určen svojí kontrolní matici o rozměrech $r \times \frac{q^r-1}{q-1}$, která obsahuje ve sloupcích všechny po dvou lineárně nezávislé vektory nad $GF(q)$ délky $r$. Kód $\mathcal{H}(r, q)$ je $1$-perfektní $(\frac{q^r-1}{q-1}, \frac{q^r-1}{q-1}-r, 3)_q$ lineární kód.

\df Uvažme matici $G'=(I_{12}~|~Q)$, kde $Q$ je doplněk matice sousednosti dvacetistěnu. Matice $G'$ generuje $(24, 12, 8)_2$ kód $\mathcal{G}_{24}$ nad $GF(2)$. Vynecháním libovolné fixní souřadnice kódových slov v $\mathcal{G}_{24}$ obdržíme $(23, 12, 7)_2$ kód $\mathcal{G}_{23}$. Kód $\mathcal{G}_{23}$ se nazývá binární Golayův $3$-perfektní kód.

\df Uvažme matici $G=(I_6~|~Q)$, kde 
\begin{displaymath}
Q= \left(
\begin{array}{ccccc}
1& 1 & 1 & 1 &  1\\
0& 1 & 2 & 2 &  1\\
1& 0 & 1 & 2 &  2\\
2& 1 & 0 & 1 &  2\\
2& 2 & 1 & 0 &  1\\
1& 2 & 2 & 1 &  0\\
\end{array}
\right).
\end{displaymath}

Matice $G$ generuje $(11, 6, 5)_3$ kód $\mathcal{G}_{11}$ nad $GF(3)$. Kód $\mathcal{G}_{11}$ se nazývá ternární Golayův $2$-perfektní kód. 

\subsection{Lloydova věta}
Nyní směřujeme k charakterizující větě, která říká, že ve skutečnosti žádné jiné perfektní kódy než výše uvedené nad abecedou mohutnosti mocniny prvočísla neexistují. Důkaz, který uvedeme je kombinatorický. Jádro důkazu spočívá v důkazu Lloydovy věty, která dává silné omezení na existenci perfektních kódů. 



\vt Definujme Lloydův polynom v proměnné $x$ stupně $t$

\begin{displaymath}
	L_t(x) = \sum_{j=0}^t(-1)^j(q-1)^{t-j}\binom{x-1}{j}\binom{n-x}{t-j}
\end{displaymath}
Pokud existuje $t$-perfektní kód délky $n$ nad abecedou mohutnosti $q$, pak $L_t(x)$  má $t$ různých celočíselných kořenů mezi $1$ a $n$.

K důkazu Lloydovy věty budeme potřebovat vlastnosti vzdálenostně regulárních grafů. 

\subsection{Vzdálenostně regulární grafy}


\df Uvažme graf $\Gamma=(V,E)$, že $V(G) = A^n$ a hrana mezi vrcholy $u,v$ vede právě tehdy, když $d(u, v) = 1$, tedy liší se právě v jedné souřadnici. Kód v grafu $\Gamma$ příslušející kódu $C$ je pak podmnožina vrcholů, které odpovídají kódovým slovům $C$.

Graf $\Gamma$ je speciálním případem vzdálenostně regulárního grafu. Poznatky z této sekce na závěr aplikujeme právě na $\Gamma$. Po celou dobu této podkapitoly pracujeme pouze s vzdálenostně regulárními grafy.

\df Graf $G$ je vzdálenostně regulární, pokud existují konstanty $s_{hij}$ tak, že  pro $\forall u,v\in V(G), d(u,v) = j$ je $$|\{w: d(u,w) = h, d(w,v) = i\}| = s_{hij}.$$

\poz $|i-j| > h \Rightarrow s_{hij} = 0$ (plyne z trojúhelníkové nerovnosti), $k = s_{110}$ je počet sousedů libovolného vrcholu v $k$-regulárním grafu. 

\lm Platí $$z_{mi} = z_{m-1,i-1} \cdot s_{1,i-1,i} + z_{m-1,i} \cdot s_{1,i,i} + z_{m-1,i+1} \cdot s_{1,i+1,i},$$ kde $z_{mi}$ značí počet sledů délky $m$ mezi vrcholy ve vzdálenosti $i$.

\dk $z_{00} = 1$, jinak $z_{0i} = 0$. Dále dokážeme indukcí pro $m \ge 1$ a $i
\ge 1$. $s_{1,i,j}$ je nenulové pouze pro $i \in \{j-1,j,j+1\}$ (z trojúhelníkové
nerovnosti). V rovnici sčítáme vrcholy sousedící s $u$, které jsou ve
vzdálenosti $i-1$, $i$ a $i+1$ od $v$.

\df Mějme matici sousednosti $A = A_G$. Označme $\A(G) = \{p(A): p(x) \in \C[x]\}$. $\A(G)$ je
vektorový prostor nad $\C$.



\df Definujme vzdálenostní matice $A_0=I, A_1=A, A_2, \dots, A_d$ grafu $G$. Sloupce a řádky jsou číslovány vrcholy grafu. \\
$$(A_i)_{uv} = \left\{\begin{matrix}
1\quad & d(u,v) = i  \\
0\quad & \text{jinak} \\
\end{matrix}\right.$$


\vt $\dim \A(G) = d+1$, kde $d$ je průměr $G$.\footnote{Průměr grafu je maximální nejkratší vzdálenost přes všechny dvojice vrcholů.} Bází $\A(G)$ jsou výše definované matice $A_0, A_1, A_2, \dots, A_d$. 

\dk Platí $A^m = \sum_{i=0}^d z_{mi}A_i$ pro libovolné $m\in \mathbb{N}$. Matice $A_0, A_1, A_2, \dots, A_d$ tedy generují celý prostor $\A(G)$ a zároveň jsou lineárně nezávislé a proto $\dim \A(G) = d+1$.
%$i > m \Rightarrow Z_{mi} = 0$ \\
%$A^0 = Z_{0,0} \cdot A_0 = A_0$\\
%$A^1 = Z_{1,0} \cdot A_0 + Z_{1,1} \cdot A_1 = A_1$\\
%$A^2 = Z_{2,0} \cdot A_0 + Z_{2,1} \cdot A_1 + Z_{2,2} \cdot A_2$\\
%$\vdots$\\
%$A^d = Z_{d,0} \cdot A_0 + Z_{d,1} + \dots + Z_{d,d}\cdot A_d$
 
\qed

%\poz $\widetilde \A = \{A_0, A_1, \dots, A_d\}$ tvoří bázi $\A(G)$.


\df Matice $B_h$ je velikosti $(d+1)\times (d+1)$ a definujme ji předpisem
$$
	(B_h)_{ij} := s_{hij}
$$
Navíc označme $B=B_1$.

\lm Existuje homomorfismus vektorových prostorů $\widehat{~~}: \A(G) \to \widehat{\A(G)}$ takový, že $\widehat{A_h} = B_h$ pro $h=0, \dots, d$.

\dk %Z předchozího lemmatu již máme bázi $\widetilde{\A}$ prostoru $\A$. Ukážeme si tedy, že můžeme přejít k bázi z menších matic $B$. 
Nejdříve si všimněme, co se děje v následujícím součinu matic:

$$
	(A_hA_i)_{uv} = \sum_w(A_h)_{uw} \cdot (A_i)_{wv} = s_{hid(u,v)}
$$

V sumě je přičtena $1$ pokaždé, když pro vrchol $w$ platí, že $d(u,w)=h$ a $d(w,v) = i$, což je přesně definice $s_{hij}$ pro $j = d(u,v)$. Máme tedy:

$$
	A_hA_i = \sum_{j=0}^d s_{hij} A_j
$$

Což je vlastně lineární kombinace prvků z báze s koeficienty $s_{hij}$.
Matici $B_h$ obsahuje v $i$-tém řádku souřadnice $A_hA_i$ vzhledem k bázi $\{A_0, A_1, \dots, A_d\}$, čili vektor $(s_{hi0}, \dots, s_{hid})$. Hledaný homomorfismus je tedy transpozice regulární reprezentace levého násobení v $\A(G)$ vzhledem k $\{A_0, A_1, \dots, A_d\}$.
\qed



\lm $B=B_1$ je tridiagonální matice. Všechny sloupcové součty jsou stejné a jsou rovny $k=s_{110}$. Navíc $s_{100}=0$ a $s_{101}=1$.
$$
B= \left(\begin{matrix}
& & & & & & & & \bigzero & \\
& & & & & & & & & \\
& \bigzero & & & & {\smash{\raisebox{.75\normalbaselineskip}{\diagdots{9em}{.5em}}}} & {\smash{\raisebox{1.2\normalbaselineskip}{\diagdots{6.5em}{.5em}}}} & \\
& & & & {\smash{\raisebox{1.3\normalbaselineskip}{\diagdots{6.5em}{.5em}}}} & & \\
\end{matrix}\right)$$

\dk Matice je tridiagonální, protože $s_{1,i,j}$ dává smysl jen pro $i \in \{j-1,j,j+1\}$ (z trojúhelníkové nerovnosti). Navíc v $j$-tém sloupci je $s_{1,j-1,j} + s_{1,j,j} + s_{1,j+1,j}$, což zahrnuje všechny sousedy $u$, kterých je $k$.
\qed

Následující známy výsledek z teorie matic uvádíme bez důkazu.

\pzn $B$ je tridiagonální matice $\Rightarrow$ $\forall$ její vlastní čísla jsou různá.

\subsection{Charakteristické polynomy}


\df Definujme polynomy $v_i \in \Q[\lambda]$ takové, že $\deg v_i(\lambda) = i$ tak, že
\begin{enumerate} 
	\item $v_0(\lambda) = 1$
	\item $v_1(\lambda) = \lambda$
	\item pro $i \in \{ 2, \dots, d-1\}$ induktivně, aby splňovaly rovnici 
	$$
		s_{1,i,i-1} v_{i-1}(\lambda) + s_{1,i,i}v_i(\lambda) + s_{1,i,i+1}v_{i+1}(\lambda) = \lambda v_i(\lambda)
	$$
\end{enumerate}

