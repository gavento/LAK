\subsection{Připomenutí pojmů}


\df Samoopravný kód $C$ s parametry $(n,q)$ je pro nás systém množin $C \subseteq M = \{0, \ldots, q-1\}^n$ (prvkům této množiny říkáme kódová slova).

\df Grafem kódu rozumíme graf $G=(V,E)$, že $V(G) = \{0, \ldots, q-1 \}^n$ a hrana mezi vrcholy $u,v$ vede právě tehdy, když $d(u,v) = 1$, tedy liší se právě v jedné souřadnici ($d$ je hammingovská vzdálenost). Kód v takovém grafu je pak podmnožina vrcholů, které odpovídají kódovým slovům.

\df Kód opravuje $t$ chyb, pokud jsou $N_t(u)$ (okolí vrcholu $u$ do vzdálenosti $t$) disjunktní pro všechny dvojce kódových slov.

\df Kód $C$ je $t$-perfektní, pokud opravuje $t$ chyb a navíc úplně pokrývá svou nosnou množinu $M$.

\tv Pokud $C$ opravuje $t$ chyb, platí:
\begin{align*}
	|C| \leq { q^n \over \sum_{i=0}^t \binom{n}{i} (q-1)^i }
\end{align*}

\dk Okolíčka musí být disjunktní, stačí tedy spočítat, kolik může být kódových
slov, což je daný výraz: V čitateli je počet všech slov. Jmenovatel počítá
velikost každého $t$-okolí, tedy vybírá možné souřadnice ke změně a jejich
potenciální nové hodnoty.



