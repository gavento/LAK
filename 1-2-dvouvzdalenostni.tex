\subsection{Dvouvzdálenostní množiny}


\vt Nechť $a,b\in\R^+$ a $P_1, P_2, \dots, P_m$ jsou body v~$\R^n$ takové, že platí $|P_iP_j| \in\{a,b\}$, $i\neq j$. Pak $m \leq \frac{(n+1)(n+4)}2$.

\dk Pro každé $P_i$ definujme polynom $f_i\colon\R^n\rightarrow\R$ předpisem
\begin{align}
f_i(x) = (\|P_i-x\|^2-a^2)(\|P_i-x\|^2-b^2).
\end{align}
Platí
\begin{align}
f_i(P_j)=\begin{cases}a^2b^2,&\text{pokud }i=j,\\0&\text{jinak,}\end{cases}
\end{align}
a pro každé $j=1,\dots,m$ je
\begin{align}
\sum_{i=1}^m\alpha_if_i(P_j)=\alpha_ja^2b^2=0,\quad\text{právě když}\quad\alpha_j=0.
\end{align}
Polynomy $f_1,\dots,f_m$ jsou tedy lineárně nezávislé a $\dim \langle f_1,\dots,f_m\rangle=m$.

Je-li $x=(x_1,\dots,x_n)$ a $P_i=(p_{i1},\dots,p_{in})$, můžeme $f_i$ rozepsat jako
\begin{align}
f_i(x)&= \left(\smash{\underbrace{(x_1-p_{i1})^2+\dots+(x_n-p_{in})}_{\sum_{j=1}^nx_j^2-2\sum_{j=1}^np_{ij}x_j+\sum_{j=1}^np_{ij}^2}}^2-a^2\right)\left((x_1-p_{i1})^2+\dots+(x_n-p_{in})^2-b^2\right)\vphantom{\underbrace{x}_{\sum_{i=1}^n}}.
\end{align}
Generátory prostoru $\langle f_1,\dots,f_m\rangle$ jsou tedy také polynomy $(x_1^2+\dots+x_n^2)^2$, $(x_1^2+\dots+x_n^2)x_i$, $x_i^2$, $x_ix_j$, $x_i$ a $1$. Těchto polynomů je celkem
\begin{align}
1+n+n+\binom n2+n+1=\frac{(n+1)(n+4)}2,
\end{align}
tedy $m=\dim\langle f_1,\dots,f_m\rangle\leq\frac{(n+1)(n+4)}2$.\qed

\smallskip
Množina $\{P_1,\dots,P_m\}$ z~předchozí věty se nazývá {\it dvouvzdálenostní množina}.

\vt Nechť $\{P_1,\dots,P_m\}$ je dvouvzdálenostní množina v~$\R^n$ taková, že všechna $P_i$ leží na jedné sféře. Pak platí
\begin{align}
\frac{n(n+1)}2 \leq m_{\max} \leq \frac{n(n+3)}2.
\end{align}

\dk Nejprve ukážeme horní odhad. Definujme $f_i$ stejně jako v~důkazu předchozí věty. Opět platí $\dim\langle f_1,\dots,f_m\rangle=m$, ale za generující polynomy stačí vzít $x_i^2$, $x_ix_j$ a $x_i$, neboť na sféře je $x_1^2+\dots+x_n^2$ konstantní. Generujících polynomů je $n+\binom n2+n=\frac{n(n+3)}2$, tedy $m\leq\frac{n(n+3)}2$.

Nyní ukážeme vhodnou konstrukcí dolní odhad. Vezmeme ty body v~$\R^{n+1}$, které mají dvě souřadnice jedničkové a všechny ostatní nulové. Vzdálenost dvou bodů s~jedničkami na různých pozicích je $2$ a vzdálenost dvou bodů s~jednou jedničkou společnou je $\sqrt 2$. Skutečně se tedy jedná o~dvouvzdálenostní množinu.

Pro všechna $P_i$ platí
\begin{align}
\sum_{j=1}^{n+1}p_{ij}^2=2\quad\text{a}\quad\sum_{j=1}^{n+1}p_{ij}=2,
\end{align}
tedy body $P_1,\dots,P_m$ ($m=\binom{n+1}2=\frac{n(n+1)}2$) leží na jedné sféře v~$\R^{n+1}$ a zároveň v~jedné nadrovině dimenze $n$. Průnikem této sféry a této nadroviny je zřejmě hledaná sféra v~$\R^n$.\qed