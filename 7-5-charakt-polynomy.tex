\subsection{Charakteristické polynomy}



\df Definujme polynomy $v_i \in \Q[\lambda]$ takové, že $\deg v_i(\lambda) = i$ tak, že
\begin{enumerate} 
	\item $v_0(\lambda) = 1$
	\item $v_1(\lambda) = \lambda$
	\item pro $i \in \{ 2, \dots, d-1\}$ induktivně, aby splňovaly rovnici 
	$$
		s_{1,i,i-1} v_{i-1}(\lambda) + s_{1,i,i}v_i(\lambda) + s_{1,i,i+1}v_{i+1}(\lambda) = \lambda v_i(\lambda)
	$$
\end{enumerate}

\lm (O charakteristickém polynomu) Nechť $\lambda_1,\ldots,\lambda_d \in \Sp(B_1)$ takové, že jsou různá od $k$ . Potom pro $i=1, \dots, d$ platí:
$$
v_o(\lambda_i) + \ldots + v_d(\lambda_i)=0
$$ neboli $v_o(\lambda) + \ldots + v_d(\lambda) = c \cdot (\lambda - \lambda_1) \cdot \ldots \cdot (\lambda - \lambda_d)$.

\dk Vytvořme vektor $\vec{v} = (v_1(\lambda), \ldots, v_d(\lambda))$ a uvažme systém rovnic $B\vec{v} = \lambda \vec{v}$. Ten umíme řešit po řádcích (známe první dva členy vektoru a celou matici obsahující potřebné koeficienty), známe tedy vlastní čísla (kořeny této rovnice) a jejich vlastní vektory (obsahují složky $v_i(\lambda)$.

Nejprve si ukážeme, že jedno z vlastních čísel je $k$ (všimněte si, že v předpokladech používáme $d$ vlastních čísel, ale dimenze matice $B$ je $d+1$). Vezměme si výše používaný systém rovnic a sečtěme levé a pravé strany. Podle Lemma o sousedech jsou sloupcové součty matice $B$ rovny $k$, získáme tedy rovnici $k(v_0(\lambda) + \ldots + v_d(\lambda)) = \lambda (v_0(\lambda) + \ldots + v_d(\lambda))$, z čehož po úpravě plyne, že $\lambda = k$.

\todo{Rovnost s char. polynomem}

\lm Pro polynomy $v_i$ platí, že $v_i(A) = A_i$ a $v_i(B) = B_i$.

\dk $$AA_i=\sum^{d}_{j=0}s_{1ij}A_j=s_{1,i,i-1} A_{i-1} + s_{1,i,i}A_i + s_{1,i,i+1}A_{i+1}$$
Tj. $v_i(A)=A_i$. Po aplikaci homomorfismu $\widehat{~~}$ dostáváme $v_i(B)=B_i$.


\df Zafixujme $z\in V(G)$. Definujme $T\in\{0,1\}^{(d+1) \times n}$ předpisem
$$ T_{i,u} = \left\{\begin{matrix}
1\quad & d(u,z) = i\hfill \\
0\quad & \text{jinak}\hfill
\end{matrix}\right.$$

\lm (O zastřešování) $X\in\A(G), z\in V(G) \Rightarrow TX = \widehat XT$

\dk 
\begin{align*}
(TA)_{iu} &= \sum_w T_{iw}A_{wu} = s_{i,1,d(u,z)} \\
(BT)_{iu} &= \sum_j B_{ij}T_{ju} = s_{1,i,d(u,z)}=s_{i,1,d(u,z)} \\
TA = BT \quad&\Rightarrow\quad TA^2 = BTA = B^2T \quad\Rightarrow\quad TA^m = B^mT \\
Tp(A) = p(B)T \quad&\Rightarrow\quad TX = \widehat XT
\end{align*}
\qed

\df Definujme si pomocné polynomy: 
\begin{align*}
	x_i(\lambda) &= v_0(\lambda) + \dots + v_i(\lambda) \\
  S_t &= x_t(A) = A_0 + A_1 + \dots + A_t
\end{align*}
Kde $S_t$ je matice, která označuje dvojce vrcholů jedničkou, pokud jsou vzdálené nanejvýš $t$ (je to součet vzdálenostních matic do $t$).

\lm Nechť $C$ je perfektní kód v grafu $G$. Ať $c$ je jeho charakteristický vektor $C$. Pak $S_t\cdot c = \vec 1$.

\dk $(S_t\cdot c)_u = |\{w: w\in C, d(w,u) \le t\}| = 1$, což plyne přímo z definice perfektního kódu.
\qed

\lm $G$ obsahuje $t$-perfektní kód $C$ $\Rightarrow$ $\dim \Ker \widehat S_t \ge t$

\dk Mějme $z_0 = z \in C$ a $z_1,z_2,\dots,z_t\in C$ takové, že $d(z,z_i) = i$ pro $i = 1, 2, \dots, t$.
Platí
$(T_{z_i} \cdot c)_j = \delta_{ij}$ (Kroneckerovo delta $= 1$ pro $i=j$, $0$ jinak). Tedy vektory $T_{z_i} \cdot c$ pro $i = 0, 1, \dots, t$ jsou lineárně nezávislé. Navíc dostáváme, že

\begin{align*}
	&\widehat S_t(T_{z_i}\cdot c) = (\widehat S_t \cdot T_{z_i}) \cdot c \overset{1}{=} T_{z_i} \cdot S_t \cdot c \overset{2}{=} T_{z_i}\cdot \vec 1 = \left(\begin{matrix}
		k_0 \\ \vdots \\ k_d
	\end{matrix}\right) \\
\end{align*}

$\overset{1}{=}$ plyne z lemma o zastřešování, $\overset{2}{=}$ plyne z předchozího lemmatu. Výsledný vektor je pro všechny volby $z_i$ stejný, protože jeho položky je počet sousedů s pevnými vzdálenostmi, a protože je to vzdálenostně regulární graf, jsou to nějaké hodnoty $s_{hij}$ se stejným $hij$ pro řádek. Pišme

\begin{align*}
	&u_i = T_{z_i}\cdot c - T_{z_0}\cdot c\qquad i = 1, 2, \dots, t \\
	&\widehat S_t u_i = \widehat S_t T_{z_i}\cdot c - \widehat S_t T_{z_i}\cdot c = \left(\begin{matrix}k_0 \\ \vdots \\ k_d\end{matrix}\right) - \left(\begin{matrix}k_0 \\ \vdots \\ k_d\end{matrix}\right) = \vec 0 \quad\Rightarrow\quad u_i\in\Ker \widehat S_t\\
\end{align*}

Vektory $u_1,\dots, u_t$ tvoří $\Ker \widehat S_t$ a jsou lineárně nezávislé. Tedy $\dim\Ker\widehat S_t \ge t$.
\qed
